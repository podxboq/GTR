\documentclass{tstextbook}

\usepackage{amsmath,amsfonts,amssymb,amsthm,mathrsfs}
\usepackage{glossaries}
\usepackage{url}

\providecommand{\N}{\mathbb{N}}
\providecommand{\Z}{\mathbb{Z}}
\providecommand{\R}{\mathbb{R}}
\providecommand{\C}{\mathbb{C}}
\providecommand{\pescalar}[2]{\langle #1,#2 \rangle}
\providecommand{\braket}[2]{\left\langle#1\mid#2\right\rangle}
\providecommand{\bra}[1]{\left\langle#1\right\rvert}
\providecommand{\ket}[1]{\left\lvert#1\right\rangle}
\providecommand{\so}{\Rightarrow}
\providecommand*{\circled}[1]{\tikz[baseline=(char.base)]{\node[shape=circle,draw,inner sep=2pt] (char) {#1};}}
\providecommand{\by}[1]{\overset{\fbox{\tiny #1}}{=}}
\providecommand{\maps}[3]{#1:#2\longrightarrow #3}
\providecommand{\cartalocal}{(M^n,p,U,\varphi)\textmd{ una carta local}}
\providecommand{\doscartalocal}{(M,p,U,\varphi)\textmd{ y } (N,q,V,\psi)\textmd{ dos cartas locales}}
\providecommand{\indexdots}[3]{#1=#2,\ldots,#3}
\providecommand{\define}[2]{\textbf{#1}\label{def:#2}}
\providecommand{\avg}[1]{\left\langle#1\right\rangle}
\providecommand{\abs}[1]{\lvert#1\rvert}
\providecommand{\nor}[1]{\lVert#1\rVert}
\providecommand{\operatoravg}[3]{\left\langle#1|#2|#3\right\rangle}
\providecommand{\cinfinity}[1]{\mathscr{C}^\infty(#1)}
\providecommand{\mapsdef}[5]{ #1:\ #2 & \longrightarrow #3 \\ #4 & \longmapsto #5}
\providecommand{\glossarydef}[3]{\newglossaryentry{#1}{name={#2},description={#3}}\gls{#1}}
\makeglossaries
\begin{document}

    \tsbook{Apuntes en teorías de cuerdas}
    {Francisco Costa}
    {podxboq}
    {2021}
    {xxxxx}{xxx--xx--xxxx--xx--x}{0.1}
    {Autor independiente}
    {Murcia}

    \printglossaries

    \chapter{La chapa}\label{ch:la-chapa}
    Este libro quiere presentar matemáticamente los conceptos necesarios para el desarrollo de la Teoría General de la
Relatividad (GTR) de Einstein.

Recoger y unificar la terminología usada de forma diferente por diversos autores, ha sido el trabajo más importante
de este libro con respecto a otros libros sobre este tema.

Presentamos conceptos en geometría diferencial, variedades diferenciales, cálculo tensorial, etc.

Este documento puede ser usado libremente en las condiciones que establece la licencia GNU Free Documentation License
(www.gnu.org/copyleft/fdl.html).

Documento escrito en \LaTeX, con IntelliJ Idea como IDE y el plugin TeXiFy IDEA
de Hannah-Sten en un ordenador GNU/Linux (Manjaro).

Iniciado durante la pandemia del 2020.

\section{Notaci\'on}\label{ch:notacion}
\begin{itemize}
  \item Diremos que $V$ es un e-vectorial para indicar que es un espacio vectorial
  de dimensión finita y con un producto escalar $\pescalar{}{}$ sobre un cuerpo $K$.
  \item Si $V$ es un espacio vectorial, llamamos $V^*$ a su espacio dual.
  \item Diremos que una función de $\R^n$ en $\R^m$ es $(n,m)$-suave si se puede diferencial indefinidamente.
  \item Diremos que $M$ es una variedad diferenciable de dimensión $n$ con una métrica.
  \item Se llama a $\cartalocal$ si $M$ es un variedad diferenciable, $p\in U$ con $U$ abierto en $M$ y $\maps{\varphi}{U}{\R^n}$ es un homeomorfismo.
  \item Usaremos la función delta de kronecker con los índices en sus variantes
  $\delta_{\alpha\beta}=\delta_\alpha^\beta=\delta^{\alpha\beta}$.
  \item Dada una matriz $A$ describiremos a sus entradas con los índices en sus variantes
  $A=(A_{\alpha\beta})_{\alpha\beta}=(A_\alpha^\beta)_\alpha^\beta=(A^{\alpha\beta})^{\alpha\beta}$.
  \item Si $V$ es un espacio vectorial, denotaremos
  $V^{\otimes r}=V\otimes\overbrace{\cdots}^\text{r\ veces}\otimes V$.
\end{itemize}

\section{\'Indices}\label{sec:indices}
Principalmente para simplificar la notación al trabajar con índices, vamos
a introducir algunos conceptos y establecer una notación que haga más cómodo
trabajar con variables multi-indexadas.

\begin{notation}
  Denotaremos por:
  \begin{itemize}
    \item $I_n$ al conjunto de los naturales menores o iguales a $n$.
    \item $I_n^m$ al producto cartesiano $m$-veces de $I_n$.
    \item $P_n^m=\{(i_1,\cdots,i_m)\in I_n^m\ \mid \ i_k\neq 0\ \forall\ \indexdots{k}{1}{m}\}$
    \item $H_{kn}^m(l)=\{(i_1,\cdots,i_m)\in I_n^m\ \mid \ i_k=l\}$
  \end{itemize}
\end{notation}

\begin{notation}
  Si $\sigma=(1,\ldots,n)$, con la expresión $(x^\sigma)_\sigma$, estaremos escribiendo la $n$-tupla $(x^1,\ldots,
  x^n)$.
  Sin embargo con la expresión $x^\sigma$, estaremos escribiendo $x^{1,\ldots,n}$.
\end{notation}

\subsection{Criterio de índices}\label{subsec:criterio-de-indices}

Emplearemos continuamente el \textbf{convenio de sumas de Einstein}\index{convenio de sumas de Einstein} según el cual
índices repetidos arriba y abajo en una expresión están sumado en todos sus posibles valores,
así la expresión $y=\sum_{\alpha=1}^3 c_\alpha x^\alpha=c_1 x^1 + c_2 x^2 + c_3 x^3$
se simplifica por la convención a $y = c_\alpha x^\alpha$.


    \chapter{Nociones Básicas: Espacios vectoriales}\label{ch:basico-espacio-vectorial}
    \section{Espacio vectorial}\label{subsec:espacio-vectorial}
No necesitamos amplios conocimientos en espacios vectoriales y de forma implícita estaremos trabajando con espacios
vectoriales sobre $\R$, de dimensión finita y con un producto escalar definido, para una definición genérica de
espacio vectorial se puede consultar la wikipedia~\cite{wiki:espacio-vectorial}.

A lo largo del texto, cuando digamos que \glossarydef{espacio-vectorial}{$V$}{Espacio vectorial} es un espacio
vectorial, estamos diciendo que $V$ es un espacio vectorial sobre $\R$, de dimensión finita $n$ y con un producto
escalar definido.
Cuando sea importante indicar la dimensión del espacio vectorial lo denotaremos por
\glossarydef{espacio-vectorial-dimensional}{$V^n$}{Espacio vectorial de dimensión n}.

Una base del espacio vectorial $V$ lo denotamos por $\{v_\alpha\}$ sin indicar el valor que toma el índice $\alpha$,
pues daremos por entendido que toma todos los valores desde $1$ hasta la dimensión de $V$.

Todas las aplicaciones entre espacios vectoriales o entre el espacio vectorial y $\R$ son lineales.

\section{Espacio dual}\label{sec:espacio-dual}\index{Espacio dual}
La noción de espacio dual \glossarydef{dual}{$V^*$}{Espacio dual de $V$}\cite{wiki:espacio-dual}, el conjunto de las aplicaciones lineales de $V$ sobre $\R$, es
sencillo y no merece la pena dedicarle mucho detalle, pero por la importancia que tiene en el desarrollo teórico de
la GTR vamos a recordar unas igualdades sencillas.

\begin{proposition}
  \label{res:coordenadas_duales}
  Sea $V$ un espacio vectorial con $\{e^\alpha\}$ una base de $V$ y $\{e_\alpha\}$ su base dual.
  $\forall v\in V\coma \forall f\in V^*$ se cumple que:
  \begin{itemize}
    \item $v=e_\alpha(v)e^\alpha$.
    \item $f=f(e^\alpha)e_\alpha$.
    \item $f(v)=e_\alpha(v)f(e^\alpha)$.
  \end{itemize}
\end{proposition}

\subsection{Dual del cambio de base}\label{subsec:dual-del-cambio-de-base}
Sea $V$ un espacio vectorial con $\{e^\alpha\}$ y $\{e'^\alpha\}$ bases de $V$ y $A$ la matriz cambio de base.
Sea $\{f_\alpha\}$ y $\{f'_\alpha\}$ sus bases duales y $B$ la matriz cambio de base, entonces:
\[
  \delta_\alpha^\beta=f_\alpha(e^\beta)=B_\alpha^\mu f'_\mu(e^\beta)=B_\alpha^\mu A^\beta_\nu f'_\mu(e'^\nu)=B_\alpha^\mu A^\beta_\nu\delta^\nu_\mu=B_\alpha^\mu A^\beta_\mu=(BA)^\beta_\alpha.
\]
De forma análoga vemos que $\delta^\beta_\alpha=(AB)^\beta_\alpha$, y por tanto que $BA=AB=I_n$, se obtiene así el
siguiente resultado.

\begin{proposition}
  \label{res:dual_cambio_base}
  Sea $V$ un espacio vectorial con $\{e^\alpha\}$ y $\{e'^\alpha\}$ bases de $V$ y $A$ la matriz cambio de base.
  La matriz cambio de base de las bases duales $\{f_\alpha\}$ en $\{f'_\alpha\}$ es $A^{-1}$.

\end{proposition}


    \chapter{First chapter}

    \begin{summary}
        This first chapter illustrates how to use various elements of this
        text book template, such as definitions, theorems and exercises. You
        may want to start each chapter with a meta summary like this one, to
        explain to the reader what the chapter is all about, why it is
        important and how it fits into the bigger picture of the
        book. Another useful tip is to put the contents of each chapter into
        a separate \LaTeX{} file and then use the command
        \texttt{\textbackslash{}input\{\}} to include the chapter in the
        main document.
    \end{summary}


    \section{First section}

    Let's start out with the following theorem.

    \begin{theorem}[Logic algebra]
        \label{th:logicalgebra}
        \index{logic algebra}
        Let $P$, $Q$ and $R$ be logical propositions (true or false).
        Then the following propositions are true:
        \small
        \begin{align*}
            P \land Q &\Leftrightarrow Q \land P &
            P \lor  Q &\Leftrightarrow Q \lor P  &&
            \text{(commutative laws)} \\
            (P \land Q) \land R &\Leftrightarrow P \land (Q \land R) &
            (P \lor Q)  \lor  R &\Leftrightarrow P \lor  (Q \lor  R) &&
            \text{(associative laws)} \\
            P \land (Q \lor  R) &\Leftrightarrow (P \land Q) \lor  (P \land R) &
            P \lor  (Q \land R) &\Leftrightarrow (P \lor  Q) \land (P \lor  R) &&
            \text{(distributive laws)} \\
            \lnot (P \land Q) &\Leftrightarrow \lnot P \lor  \lnot Q &
            \lnot (P \lor  Q) &\Leftrightarrow \lnot P \land \lnot Q &&
            \text{(De Morgan's laws)}
        \end{align*}
    \end{theorem}
    \begin{proof}
        \newcommand{\T}{\mathsf{T}}
        \newcommand{\TT}{\mathbf{T}}
        \renewcommand{\F}{\mathsf{F}}
        We prove the first of De Morgan's laws and leave the proofs of
        the remaining propositions as exercises. To prove the statement,
        we create a truth table and fill in all possible values (true or
        false) for the propositions $P$ and $Q$. Each of these propositions
        can be either true or false and we thus obtain the following truth
        table with four cases:
        \begin{center}
            \begin{tabular}{cccccccccc}
                $\lnot$ & ($P$ & $\land$ & $Q$) & $\Leftrightarrow$ & $\lnot$ & $P$  & $\lor$ & $\lnot$ & $Q$  \\
                \midrule
                & $\T$ &         & $\T$ &                   &         & $\T$ &        &         & $\T$ \\
                & $\T$ &         & $\F$ &                   &         & $\T$ &        &         & $\F$ \\
                & $\F$ &         & $\T$ &                   &         & $\F$ &        &         & $\T$ \\
                & $\F$ &         & $\F$ &                   &         & $\F$ &        &         & $\F$
            \end{tabular}
        \end{center}
        By definition of the logical operators, we compete the table to obtain
        \begin{center}
            \begin{tabular}{cccccccccc}
                $\lnot$ & ($P$ & $\land$ & $Q$) & $\Leftrightarrow$ & $\lnot$ & $P$  & $\lor$ & $\lnot$ & $Q$  \\
                \midrule
                $\F$    & $\T$ & $\T$    & $\T$ & $\TT$             & $\F$    & $\T$ & $\F$   & $\F$    & $\T$ \\
                $\T$    & $\T$ & $\F$    & $\F$ & $\TT$             & $\F$    & $\T$ & $\T$   & $\T$    & $\F$ \\
                $\T$    & $\F$ & $\F$    & $\T$ & $\TT$             & $\T$    & $\F$ & $\T$   & $\F$    & $\T$ \\
                $\T$    & $\F$ & $\F$    & $\F$ & $\TT$             & $\T$    & $\F$ & $\T$   & $\T$    & $\F$
            \end{tabular}
        \end{center}
        It follows that the statement we want to prove (the equivalence $\Leftrightarrow$)
        is always true (a \emph{tautology}), which proves the statement.
    \end{proof}


    \section{Second section}

    We begin our next section with the following central definition.

    \begin{definition}[Rational Cauchy sequence]
        \label{th:rationalcauchysequence}
        \index{rational Cauchy sequence}
        A rational Cauchy sequence is a rational sequence
        $(x_n)_{n=0}^{\infty}$ such that
        \begin{equation}
            \forall \epsilon \in \mathbb{Q}_+ \;
            \exists N \in \mathbb{N} : m, n \geq N \Rightarrow |x_m - x_n| < \epsilon.
        \end{equation}
        In other words, for each (small) rational number $\epsilon > 0$
        there is a (big) number $N$ such that the distance $|x_m - x_n|$
        between $x_m$ and $x_n$ is less than $\epsilon$ if both $m$ and $n$
        are larger than or equal to $N$.
    \end{definition}

    \begin{remark}
        A remark may be in order here. This definition is concerned with
        \emph{rational} Cauchy sequences. We will later encounter a similar
        definition of \emph{real} Cauchy sequences.
    \end{remark}

    \begin{example}[Solving the equation $x^2 = 2$]
        Consider the equation $x^2 = 2$. It is easy to prove that this
        equation does not have any rational solutions. However, consider
        the following iteration formula:
        \begin{equation}
            x_n = \frac{x_{n-1} + 2 / x_{n - 1}}{2},
        \end{equation}
        where $n = 1,2,3,\ldots$ and $x_0 = 1$. The resulting sequence of
        rational numbers quickly approaches a number in the vicinity of
        $x = 1.4142135623731$:
        \begin{displaymath}
            \begin{array}{rclcl}
                x_0 & = & 1 \\
                x_{1}  & = & (x_{0} + 2 / x_{0}) / 2 & =       & 1.5             \\
                x_{2}  & = & (x_{1} + 2 / x_{1}) / 2 & \approx & 1.4166666666667 \\
                x_{3}  & = & (x_{2} + 2 / x_{2}) / 2 & \approx & 1.4142156862745 \\
                x_{4}  & = & (x_{3} + 2 / x_{3}) / 2 & \approx & 1.4142135623747 \\
                x_{5}  & = & (x_{4} + 2 / x_{4}) / 2 & \approx & 1.4142135623731 \\
                x_{6}  & = & (x_{5} + 2 / x_{5}) / 2 & \approx & 1.4142135623731 \\
                x_{7}  & = & (x_{6} + 2 / x_{6}) / 2 & \approx & 1.4142135623731 \\
                x_{8}  & = & (x_{7} + 2 / x_{7}) / 2 & \approx & 1.4142135623731 \\
                x_{9}  & = & (x_{8} + 2 / x_{8}) / 2 & \approx & 1.4142135623731 \\
                x_{10} & = & (x_{9} + 2 / x_{9}) / 2 & \approx & 1.4142135623731
            \end{array}
        \end{displaymath}
        We will later see that this iteration, or any other equivalent
        iteration, defines the real number $\sqrt{2}$.
    \end{example}


    \section{Third section}

    Now let's move on to the definition of the real number system. This
    may be defined in a multitude of ways, one of which is to think about
    a real number as a rational Cauchy sequence, or rather the equivalence
    class of Cauchy sequences ``converging to'' that number.

    \begin{definition}[The real numbers $\mathbb{R}$]
        \label{def:realnumbers}
        \index{real numbers}
        The real numbers $\mathbb{R}$ is the set of all equivalence classes
        of rational Cauchy sequences.
    \end{definition}

    Now that this is settled, lets prove the completeness of the real
    number system.

    \begin{theorem}[The completeness of the real numbers]
        \label{th:realnumberscomplete}
        \index{completeness of the real numbers}
        Let $(x_n)_{n=0}^{\infty}$ be a sequence of real numbers.
        Then $(x_n)_{n=0}^{\infty}$ is convergent if and only if
        it is also a real Cauchy sequence.
    \end{theorem}
    \begin{proof}
        Write $x_m = [(x_{mn})_{n=0}^{\infty}]$ where
        $x_{mn}$ is the $n$th number in a rational Cauchy sequence
        representing the real number $x_m$. And so on\ldots.
    \end{proof}

    For further reading, there are several excellent works that one could
    cite, such as \cite{Tao2006,Turing1936}.

    \section*{Exercises}

    \begin{exercise}
        Let $A = \{1, 2, 3\}$ and $B = \{2, 3, 4\}$.
        Determine the following sets. \\
        (a) $A \cup B$ \quad
        (b) $A \cap B$ \quad
        (c) $A \setminus B$ \quad
        (d) $A \times B$
    \end{exercise}

    \begin{exercise}
        Let $A = \{1, 3, 5, 7, 9\}$ and $B = \{2, 4, 6, 8, 10\}$.
        Determine the following sets. \\
        (a) $A \cup B$ \quad
        (b) $A \cap B$ \quad
        (c) $A \setminus B$ \quad
        (d) $A \times B$
    \end{exercise}

    \begin{exercise}
        Let $A = \{1, 2, 3\}$, $B = \{2, 3, 4\}$ and $C = \{3, 4, 5\}$.
        Determine the following sets. \\
        (a) $A \cup B \cup C$ \quad
        (b) $A \cap B \cap C$ \quad
        (c) $(B \setminus A) \cap C$ \quad
        (d) $(A \times B) \times C$
    \end{exercise}

    \section*{Problem}

    \begin{problem}
        Interpret the following set definition (Russell's paradox) and discuss
        whether $X \in X$ or $X \notin X$:
        \begin{equation}
            X = \{x \mid x \notin x\}.
        \end{equation}
    \end{problem}

    \section*{Computer exercises}

    \begin{programming}
        Write a program that generates the sequence $(x_n)_{n=0}^{100}$
        for $x_n = n$.
    \end{programming}

    \begin{programming}
        Write a program that generates the odd numbers between $1$ and $100$.
    \end{programming}

    \begin{programming}
        Write a program that computes the sum $\sum_{n=0}^{100} x_n$
        for $x_n = n$.
    \end{programming}

%---------------------------------------------------------------------------


    \chapter{Second chapter}

    \begin{summary}
        \blindtext
    \end{summary}


    \section{First section}
    \Blindtext


    \section{Second section}
    \Blindtext


    \section{Third section}
    \Blindtext

%---------------------------------------------------------------------------


    \chapter{Third chapter}

    \begin{summary}
        \blindtext
    \end{summary}


    \section{First section}
    \Blindtext


    \section{Second section}
    \Blindtext


    \section{Third section}
    \Blindtext

%---------------------------------------------------------------------------
% Bibliography
%---------------------------------------------------------------------------

    \addcontentsline{toc}{chapter}{\textcolor{tssteelblue}{Literature}}
    \printbibliography{}

%---------------------------------------------------------------------------
% Index
%---------------------------------------------------------------------------

    \printindex

\end{document}
