\documentclass[11pt,fleqn]{book}
\usepackage[spanish]{babel} % Spanish language/hyphenation
\usepackage[utf8]{inputenc} % Required for including letters with accents
\usepackage{amsmath,amsfonts,amssymb,amsthm,mathrsfs}
\usepackage{avant} % Use the Avantgarde font for headings
\usepackage{mathptmx} % Use the Adobe Times Roman as the default text font together with math symbols from the Sym­bol, Chancery and Com­puter Modern fonts
\usepackage{microtype} % Slightly tweak font spacing for aesthetics
\usepackage[T1]{fontenc} % Use 8-bit encoding that has 256 glyphs
\usepackage{lmodern}
\usepackage{calc} % For simpler calculation - used for spacing the index letter headings correctly
\usepackage{tikz}
\usepackage{enumitem} % Customize lists
\usepackage{booktabs} % Required for nicer horizontal rules in tables
\usepackage{xcolor} % Required for specifying colors by name
\usepackage{geometry} % Required for adjusting page dimensions and margins
\usepackage{titletoc} % Required for manipulating the table of contents
\usepackage{makeidx} % Required to make an index
\RequirePackage[framemethod=default]{mdframed} % Required for creating the theorem, definition, exercise and corollary boxes
\usepackage{glossaries}
\usepackage{url}

%Para mapas de aplicaciones
\usetikzlibrary{cd}

\setlist{nolistsep} % Reduce spacing between bullet points and numbered lists

\definecolor{ocre}{RGB}{243,102,25} % Define the orange color used for highlighting throughout the book

%----------------------------------------------------------------------------------------
%	MARGINS
%----------------------------------------------------------------------------------------

\geometry{
	paper=a4paper, % Paper size, change to letterpaper for US letter size
	top=1.5cm, % Top margin
	bottom=1.5cm, % Bottom margin
	left=1cm, % Left margin
	right=1cm, % Right margin
	headheight=14pt, % Header height
	footskip=20pt, % Space from the bottom margin to the baseline of the footer
	headsep=10pt, % Space from the top margin to the baseline of the header
	%showframe, % Uncomment to show how the type block is set on the page
}

%----------------------------------------------------------------------------------------
%	FONTS
%----------------------------------------------------------------------------------------

\makeindex % Tells LaTeX to create the files required for indexing

%----------------------------------------------------------------------------------------
%	MAIN TABLE OF CONTENTS
%----------------------------------------------------------------------------------------

\contentsmargin{0cm} % Removes the default margin

%----------------------------------------------------------------------------------------
%	THEOREM STYLES
%----------------------------------------------------------------------------------------
\renewcommand{\qedsymbol}{$\blacksquare$}

% Defines the theorem text style for each type of theorem to one of the three styles above
\newcounter{dummy}
\numberwithin{dummy}{section}
\newtheorem{theoremeT}[dummy]{Theorem}
\newtheorem{exerciseT}[dummy]{Ejercicio}
\newtheorem{exampleT}[dummy]{Example}
\newtheorem{definitionT}[dummy]{Definición}
\newtheorem{proposition}[dummy]{Resultado}
\newtheorem{notation}[dummy]{Notación}
\newtheorem*{remark}{OJO}
\numberwithin{equation}{dummy}

% Theorem box
\newmdenv[skipabove=7pt,
skipbelow=7pt,
linewidth=1pt,
backgroundcolor=black!5,
linecolor=ocre]{tBox}

% Exercise box
\newmdenv[skipabove=7pt,
skipbelow=7pt,
rightline=false,
leftline=true,
topline=false,
bottomline=false,
backgroundcolor=ocre!10,
linecolor=ocre,
innerleftmargin=5pt,
innerrightmargin=5pt,
innertopmargin=5pt,
innerbottommargin=5pt,
leftmargin=0cm,
rightmargin=0cm,
linewidth=4pt]{eBox}

% Definition box
\newmdenv[skipabove=7pt,
skipbelow=7pt,
rightline=false,
leftline=true,
topline=false,
bottomline=false,
linecolor=ocre,
innerleftmargin=5pt,
innerrightmargin=5pt,
innertopmargin=0pt,
leftmargin=0cm,
rightmargin=0cm,
linewidth=4pt,
innerbottommargin=0pt]{dBox}

% Creates an environment for each type of theorem and assigns it a theorem text style from the "Theorem Styles" section above and a colored box from above
\newenvironment{theorem}{\begin{tBox}\begin{theoremeT}}{\end{theoremeT}\end{tBox}}
\newenvironment{definition}{\begin{eBox}\begin{definitionT}}{\hfill{\color{ocre}\tiny\ensuremath{\blacksquare}}\end{definitionT}\end{eBox}}
\newenvironment{exercise}{\begin{dBox}\begin{exerciseT}}{\end{exerciseT}\end{dBox}}

\providecommand{\N}{\mathbb{N}}
\providecommand{\Z}{\mathbb{Z}}
\providecommand{\R}{\mathbb{R}}
\providecommand{\C}{\mathbb{C}}
\providecommand{\pescalar}[2]{\langle #1,#2 \rangle}
\providecommand{\braket}[2]{\left\langle#1\mid#2\right\rangle}
\providecommand{\bra}[1]{\left\langle#1\right\rvert}
\providecommand{\ket}[1]{\left\lvert#1\right\rangle}
\providecommand{\so}{\Rightarrow}
\providecommand*{\circled}[1]{\tikz[baseline=(char.base)]{\node[shape=circle,draw,inner sep=2pt] (char) {#1};}}
\providecommand{\by}[1]{\overset{\fbox{\tiny #1}}{=}}
\providecommand{\maps}[3]{#1:#2\longrightarrow #3}
\providecommand{\cartalocal}{(M^n,p,U,\varphi)\textmd{ una carta local}}
\providecommand{\doscartalocal}{(M,p,U,\varphi)\textmd{ y } (N,q,V,\psi)\textmd{ dos cartas locales}}
\providecommand{\indexdots}[3]{#1=#2,\ldots,#3}
\providecommand{\define}[2]{\textbf{#1}\label{def:#2}}
\providecommand{\avg}[1]{\left\langle#1\right\rangle}
\providecommand{\abs}[1]{\lvert#1\rvert}
\providecommand{\nor}[1]{\lVert#1\rVert}
\providecommand{\operatoravg}[3]{\left\langle#1|#2|#3\right\rangle}
\providecommand{\cinfinity}[1]{\mathscr{C}^\infty(#1)}
\providecommand{\mapsdef}[5]{ #1:\ #2 & \longrightarrow #3 \\ #4 & \longmapsto #5}
\providecommand{\glossarydef}[3]{\newglossaryentry{#1}{name={#2},description={#3}}\gls{#1}}

\makeglossaries

\begin{document}


  \printglossaries


  \chapter{La chapa}\label{ch:la-chapa}
  Este libro quiere presentar matemáticamente los conceptos necesarios para el desarrollo de la Teoría General de la
Relatividad (GTR) de Einstein.

Recoger y unificar la terminología usada de forma diferente por diversos autores, ha sido el trabajo más importante
de este libro con respecto a otros libros sobre este tema.

Presentamos conceptos en geometría diferencial, variedades diferenciales, cálculo tensorial, etc.

Este documento puede ser usado libremente en las condiciones que establece la licencia GNU Free Documentation License
(www.gnu.org/copyleft/fdl.html).

Documento escrito en \LaTeX, con IntelliJ Idea como IDE y el plugin TeXiFy IDEA
de Hannah-Sten en un ordenador GNU/Linux (Manjaro).

Iniciado durante la pandemia del 2020.

\section{Notaci\'on}\label{ch:notacion}
\begin{itemize}
  \item Diremos que $V$ es un e-vectorial para indicar que es un espacio vectorial
  de dimensión finita y con un producto escalar $\pescalar{}{}$ sobre un cuerpo $K$.
  \item Si $V$ es un espacio vectorial, llamamos $V^*$ a su espacio dual.
  \item Diremos que una función de $\R^n$ en $\R^m$ es $(n,m)$-suave si se puede diferencial indefinidamente.
  \item Diremos que $M$ es una variedad diferenciable de dimensión $n$ con una métrica.
  \item Se llama a $\cartalocal$ si $M$ es un variedad diferenciable, $p\in U$ con $U$ abierto en $M$ y $\maps{\varphi}{U}{\R^n}$ es un homeomorfismo.
  \item Usaremos la función delta de kronecker con los índices en sus variantes
  $\delta_{\alpha\beta}=\delta_\alpha^\beta=\delta^{\alpha\beta}$.
  \item Dada una matriz $A$ describiremos a sus entradas con los índices en sus variantes
  $A=(A_{\alpha\beta})_{\alpha\beta}=(A_\alpha^\beta)_\alpha^\beta=(A^{\alpha\beta})^{\alpha\beta}$.
  \item Si $V$ es un espacio vectorial, denotaremos
  $V^{\otimes r}=V\otimes\overbrace{\cdots}^\text{r\ veces}\otimes V$.
\end{itemize}

\section{\'Indices}\label{sec:indices}
Principalmente para simplificar la notación al trabajar con índices, vamos
a introducir algunos conceptos y establecer una notación que haga más cómodo
trabajar con variables multi-indexadas.

\begin{notation}
  Denotaremos por:
  \begin{itemize}
    \item $I_n$ al conjunto de los naturales menores o iguales a $n$.
    \item $I_n^m$ al producto cartesiano $m$-veces de $I_n$.
    \item $P_n^m=\{(i_1,\cdots,i_m)\in I_n^m\ \mid \ i_k\neq 0\ \forall\ \indexdots{k}{1}{m}\}$
    \item $H_{kn}^m(l)=\{(i_1,\cdots,i_m)\in I_n^m\ \mid \ i_k=l\}$
  \end{itemize}
\end{notation}

\begin{notation}
  Si $\sigma=(1,\ldots,n)$, con la expresión $(x^\sigma)_\sigma$, estaremos escribiendo la $n$-tupla $(x^1,\ldots,
  x^n)$.
  Sin embargo con la expresión $x^\sigma$, estaremos escribiendo $x^{1,\ldots,n}$.
\end{notation}

\subsection{Criterio de índices}\label{subsec:criterio-de-indices}

Emplearemos continuamente el \textbf{convenio de sumas de Einstein}\index{convenio de sumas de Einstein} según el cual
índices repetidos arriba y abajo en una expresión están sumado en todos sus posibles valores,
así la expresión $y=\sum_{\alpha=1}^3 c_\alpha x^\alpha=c_1 x^1 + c_2 x^2 + c_3 x^3$
se simplifica por la convención a $y = c_\alpha x^\alpha$.



  \chapter{Nociones Básicas: Espacios vectoriales}\label{ch:basico-espacio-vectorial}
  \section{Espacio vectorial}\label{subsec:espacio-vectorial}
No necesitamos amplios conocimientos en espacios vectoriales y de forma implícita estaremos trabajando con espacios
vectoriales sobre $\R$, de dimensión finita y con un producto escalar definido, para una definición genérica de
espacio vectorial se puede consultar la wikipedia~\cite{wiki:espacio-vectorial}.

A lo largo del texto, cuando digamos que \glossarydef{espacio-vectorial}{$V$}{Espacio vectorial} es un espacio
vectorial, estamos diciendo que $V$ es un espacio vectorial sobre $\R$, de dimensión finita $n$ y con un producto
escalar definido.
Cuando sea importante indicar la dimensión del espacio vectorial lo denotaremos por
\glossarydef{espacio-vectorial-dimensional}{$V^n$}{Espacio vectorial de dimensión n}.

Una base del espacio vectorial $V$ lo denotamos por $\{v_\alpha\}$ sin indicar el valor que toma el índice $\alpha$,
pues daremos por entendido que toma todos los valores desde $1$ hasta la dimensión de $V$.

Todas las aplicaciones entre espacios vectoriales o entre el espacio vectorial y $\R$ son lineales.

\section{Espacio dual}\label{sec:espacio-dual}\index{Espacio dual}
La noción de espacio dual \glossarydef{dual}{$V^*$}{Espacio dual de $V$}\cite{wiki:espacio-dual}, el conjunto de las aplicaciones lineales de $V$ sobre $\R$, es
sencillo y no merece la pena dedicarle mucho detalle, pero por la importancia que tiene en el desarrollo teórico de
la GTR vamos a recordar unas igualdades sencillas.

\begin{proposition}
  \label{res:coordenadas_duales}
  Sea $V$ un espacio vectorial con $\{e^\alpha\}$ una base de $V$ y $\{e_\alpha\}$ su base dual.
  $\forall v\in V\coma \forall f\in V^*$ se cumple que:
  \begin{itemize}
    \item $v=e_\alpha(v)e^\alpha$.
    \item $f=f(e^\alpha)e_\alpha$.
    \item $f(v)=e_\alpha(v)f(e^\alpha)$.
  \end{itemize}
\end{proposition}

\subsection{Dual del cambio de base}\label{subsec:dual-del-cambio-de-base}
Sea $V$ un espacio vectorial con $\{e^\alpha\}$ y $\{e'^\alpha\}$ bases de $V$ y $A$ la matriz cambio de base.
Sea $\{f_\alpha\}$ y $\{f'_\alpha\}$ sus bases duales y $B$ la matriz cambio de base, entonces:
\[
  \delta_\alpha^\beta=f_\alpha(e^\beta)=B_\alpha^\mu f'_\mu(e^\beta)=B_\alpha^\mu A^\beta_\nu f'_\mu(e'^\nu)=B_\alpha^\mu A^\beta_\nu\delta^\nu_\mu=B_\alpha^\mu A^\beta_\mu=(BA)^\beta_\alpha.
\]
De forma análoga vemos que $\delta^\beta_\alpha=(AB)^\beta_\alpha$, y por tanto que $BA=AB=I_n$, se obtiene así el
siguiente resultado.

\begin{proposition}
  \label{res:dual_cambio_base}
  Sea $V$ un espacio vectorial con $\{e^\alpha\}$ y $\{e'^\alpha\}$ bases de $V$ y $A$ la matriz cambio de base.
  La matriz cambio de base de las bases duales $\{f_\alpha\}$ en $\{f'_\alpha\}$ es $A^{-1}$.

\end{proposition}


  \chapter{Nociones Básicas: Variedades diferenciables}\label{ch:basico-variedad-diferencial}
  \section{Variedad diferenciable}\label{sec:variedad-diferenciable}
Vamos a repasar todos los conceptos que necesitamos y esto nos servirá para fijar la notación que vamos a usar.

\begin{definition}
  Sea $\mathcal{M}$ un conjunto.
  Una \define{carta n-dimensional}{carta-n-dimensional} (o simplemente carta) sobre $\mathcal{M}$ es un par $
  (U\coma\varphi)$ donde $U$ es un
  abierto de
  $\mathcal{M}$ y $\maps{\varphi}{U}{\R^n}$ es un homeomorfismo.
  Al conjunto $U$ se le denomina \define{entorno coordenado}{entorno-coordenado}.
\end{definition}

Vamos a definir sobre $\mathcal{M}$ algunos conceptos trasladados del espacio Euclídeo $\R^n$, restringiendo su
alcance al entorno coordenado y apoyándonos siempre sobre como se proyecta en $\R^n$ a través de las cartas.
Las coordenadas del espacio Euclídeo las denotaremos por $(u^\alpha)$ y por $\{e_\alpha\}$ la base canónica de $\R^n$,
$\pi^\alpha$ es la proyección canónica de la $\alpha$-ésima coordenada de $\R^n$ en $\R$ y $i^\alpha$ es la inclusión
canónica de la $\alpha$-ésima coordenada de $\R$ en $\R^n$.

\begin{definition}
Llamamos \define{$\alpha$-ésima función coordenada}{alpha-esima-funcion-coordenada} a
$\varphi^\alpha=\pi^\alpha\circ\varphi\in\cinfinity{p}$.
\end{definition}

Sobre un punto $p\in\mathcal{M}$ podemos definir muchas cartas distintas, para que las definiciones siguientes no
dependan de la elección de la carta, tenemos que imponer un criterio igualdad topológica.

\begin{definition}
Sea $\mathcal{M}$ un conjunto y $(U\coma\varphi)$ y $(V\coma\phi)$ dos cartas sobre $p\in\mathcal{M}$.
Diremos que son \define{cartas compatibles}{cartas-compatibles}, si $U\cap V=\emptyset$ o los conjuntos $\varphi
(U\cap V)$ y $\phi(U\cap V)$ son abiertos en $\R^n$ y las aplicaciones $\varphi\circ\phi^{-1}$ y
$\phi\circ\varphi^{-1}$ son difeomorfismo de clase $\mathscr{C}^\infty$.
  A las aplicaciones $\varphi\circ\phi^{-1}$ y $\phi\circ\varphi^{-1}$ se les llaman \define{aplicaciones cambio de
coordenadas}{aplicacion-cambio-coordenadas}.
\end{definition}

Si un conjunto está contenido en una única carta, dicho conjunto es difeomorfo al espacio euclídeo y su estudio no
ofrecería ninguna novedad, las estructuras interesantes son por tanto, aquellas que no pueden ser cubiertas por una
única carta, y por tanto necesitan al menos dos cartas para que todos sus puntos estén en algún entorno coordenado.

\begin{definition}
Sea $\mathcal{M}$ un conjunto, llamamos \define{atlas diferenciable n-dimensional}{atlas} (o simplemente atlas) a una
familia contable de cartas compatibles donde la unión de los entornos coordenados cubre a $\mathcal{M}$.
\end{definition}

En algunos textos, se diferencia entre atlas y atlas maximal, pero no es necesario ya que la estructura que dota al
conjunto un atlas es la misma que un atlas maximal, así que nos quedamos con la siguiente definición de variedad
diferenciable.

\begin{definition}
  Una \define{variedad diferenciable n-dimensional}{variedad-diferenciable} es un par $(\mathcal{M}\coma\mathcal{A})$
  donde \glossarydef{variedad-diferencial}{$\mathcal{M}$}{Variedad diferencial} es un conjunto y $\mathcal{A}$ es un atlas diferenciable n-dimensional.
\end{definition}
Para simplificar el desarrollo del texto, diremos que tenemos \glossarydef{carta-local}{$\cartalocal$}{Carta local},
para indicar que $\mathcal{M}$ es una variedad diferenciable de dimensión $n$, que $p$ es un punto de
$\mathcal{M}$ y que $(U\coma\varphi)$ es una carta con $p\in
U$.

\begin{definition}
  Sea $\cartalocal$, diremos que $\gamma$ es una \define{curva centrada en
  $p$}{Curva centrada}\label{curva-centrada} si $\exists\ \epsilon\in\R \ \mid \maps{\gamma}{(-\epsilon\coma\epsilon)
  }{\varphi(U)}$ y
  $\gamma(0)=p$.
\end{definition}

\section{Aplicaciones diferenciables}
Vamos a extender todos los conceptos del cálculo diferencial en $\R^n$ a una variedad diferenciable, dicha extensión
de conceptos inicialmente lo haremos de forma local, es decir, fijado un punto, pues tendremos que apoyarnos siempre
en la existencia de una carta y por tanto de las coordenadas locales en dicho punto.

\begin{definition}
  Sean $\doscartaslocales$, una aplicación
  $\maps{\phi}{M}{N}$ con $\phi(p)=q$ se dice
  \define{diferenciable en p}{aplicación diferenciable} si la aplicación
  $\psi\circ \phi\circ\varphi^{-1}$ es diferenciable en $\varphi(p)$.
  La aplicación se llama diferenciable si lo es en todos los puntos del dominio.
\end{definition}

\begin{notation}
  \
  \begin{itemize}
    \item Se llama \glossarydef{c-infinity-m-n}{$\cinfinity{M, N}$}{conjunto de las aplicaciones diferenciables de
    $M$ en $N$} al conjunto de todas las aplicaciones de $M$ en $N$ diferenciables.
    \item Se llama \glossarydef{c-infinity-p-n}{$\cinfinity{p, N}$}{conjunto de las aplicaciones diferenciables en
    $p$ de $M$ en $N$} al conjunto de todas las aplicaciones de $M$ en $N$ diferenciables en $p$.
    \item Si $N=\R$, se llama \glossarydef{c-infinity-m-R}{$\cinfinity{M}$}{$\cinfinity{M, \R}$} $=\cinfinity{M,\R}$.
    \item Si $N=\R$, se llama \glossarydef{c-infinity-p-R}{$\cinfinity{p}$}{$\cinfinity{p, \R}$} $=\cinfinity{p,\R}$.
  \end{itemize}
\end{notation}

\begin{definition}
  Sea $\cartalocal$ y $\gamma$ una curva centrada en $p$, llamamos \define{derivada de $\gamma$ en $p$}{Derivada de una
  curva} a
  \begin{equation*}
    \gamma^{'}(p)=\left.\frac{d(\varphi\circ\gamma)
    }{dt}\right|_{t=0}=((\varphi^\alpha\circ\gamma)^{'}(0))_\alpha
  \end{equation*}
\end{definition}

El concepto de derivada de una curva en un punto es muy importante, pues la vamos a usar para definir el espacio
vectorial tangente en un punto, concepto crucial para el desarrollo de este libro.

\begin{definition}
  Sean $\doscartaslocales$ y $\phi\in\cinfinity{p, N}$ con $\phi(p)=q$, definimos la
  \define{derivada parcial}{Derivada parcial de una función} de $\phi$ con respecto la coordenada local
  $\varphi^\alpha$ en $p$ por
  \begin{equation*}
    \label{eq:parcial-funcion-punto}
    \frac{\partial\phi}{\partial \varphi^\alpha}(p)=\left.\frac{\partial (\psi\circ\phi\circ\varphi^{-1})}{\partial
    u^\alpha}(\varphi(p))
  \end{equation*}
\end{definition}

\begin{definition}
  Sean $\doscartaslocales$, se define el
  \define{operador derivada parcial}{Operador derivada parcial de una función} con
  respecto a la coordenada local $\varphi^\alpha$ en $p$ por
  \begin{equation*}
    \begin{alignedat}{1}
      \mapsdef{\partial^\alpha_p}{\cinfinity{p\coma N}}{\R^m}{\phi}{\frac{\partial
      \phi}{\partial\varphi^\alpha}(p)}
    \end{alignedat}
  \end{equation*}
\end{definition}

\begin{notation}
Cuando no sea preciso indicar el punto $p$, denotamos el operador derivada parcial con respecto a la coordenada
$\varphi^\alpha$ en $p$ por $\partial^\alpha$.
\end{notation}



  \chapter{Tensores}\label{ch:tensores}
  \section{Tensores como vectores}\label{sec:tensores-como-vectores}
Sean $V_1,\ldots,V_n$ espacios vectoriales de dimensión $\dim(V_i)=n_i$ y
$\{x_i^\alpha\}_{\indexdots{\alpha}{1}{n_i}}$ base de $V_i$.
Si consideramos el producto tensorial, tenemos un nuevo espacio vectorial $V=V_1\otimes\cdots\otimes V_n$
de dimensión $\sum_{\indexdots{\alpha}{1}{n}} n_\alpha$ y con base
$\{x_1^{\alpha_1}\otimes\cdots\otimes x_n^{\alpha_n}\}_{\indexdots{\alpha_i}{1}{n_i}}$.

\begin{definition}
  Llamamos \define{tensor}{tensor} a los vectores del producto tensorial de espacios vectoriales.
\end{definition}

\begin{remark}
  Hay que comprobar que $\forall v\in V, \exists v_1\in V_1, \cdots v_n\in V_n\ |\ v=v_1\otimes\cdots\otimes v_n$.
\end{remark}

Por tanto un tensor se expresa en la base $B$ como $v=v_{\alpha_1,\cdots,\alpha_n} x_1^{\alpha_1}\otimes\cdots\otimes x_n^{\alpha_n}$.

\section{Cambio de base en tensores}\label{sec:cambio-de-base-en-tensores}
Sean $V=V_1\otimes\cdots\otimes V_n$ espacio tensorial y $\{x_i^\alpha\}$, $\{y_i^\alpha\}$ bases de $V_i$, llamemos $A_i$
la matriz cambio de base de $\{x_i^\alpha\}$, $\{y_i^\alpha\}$.
Sea $v=v_1\otimes\cdots\otimes v_n \in V$ que con respecto a la base
$B=\{x_1^{\alpha_1}\otimes\cdots\otimes x_n^{\alpha_n},\ \alpha_i=1,\cdots,n_\alpha \}$
se expresa como $v=v_{\alpha_1,\cdots,\alpha_n} x_1^{\alpha_1}\otimes\cdots\otimes x_n^{\alpha_n}$ y
que con respecto a la base $C=\{y_1^{\alpha_1}\otimes\cdots\otimes y_n^{\alpha_n},\ \alpha_i=1,\cdots,n_\alpha \}$
se expresa como $v=w_{\alpha_1,\cdots,\alpha_n} y_1^{\alpha_1}\otimes\cdots\otimes y_n^{\alpha_n}$.

La relación entre las coordenadas $v_{\alpha_1,\cdots,\alpha_n}$ y $w_{\alpha_1,\cdots,\alpha_n}$
vienen expresadas por la propiedad multilineal del producto tensorial, puesto
que $x_i^{\alpha_i}=(A_i)^{\alpha_i}_\beta y_i^{\beta}$ se tiene que

\begin{multline*}
  v=v_{\alpha_1,\cdots,\alpha_n} x_1^{\alpha_1}\otimes\cdots\otimes x_n^{\alpha_n}=
  v_{\alpha_1,\cdots,\alpha_n} ((A_1)^{\alpha_1}_{\beta_1}y_1^{\beta_1})\otimes\cdots\otimes ((A_n)^{\alpha_n}_{\beta_n}y_n^{\beta_n})=\\
  =v_{\alpha_1,\cdots,\alpha_n}(A_1)^{\alpha_1}_{\beta_1}\cdots (A_n)^{\alpha_n}_{\beta_n} y_1^{\beta_1}\otimes\cdots\otimes y_n^{\beta_n}
\end{multline*}

Por tanto se tiene la igualdad
\begin{equation}
  \label{eq:tensores_cambio_base}
  w_{\beta_1,\cdots,\beta_n}=v_{\alpha_1,\cdots,\alpha_n}(A_1)^{\alpha_1}_{\beta_1}\cdots (A_n)^{\alpha_n}_{\beta_n}
\end{equation}

\subsection{Tensores sobre un único espacio vectorial y su dual}\label{subsec:tensores-sobre-un-único-espacio-vectorial-y-su-dual}
Como caso especial, vamos a considerar la situación en la que tomamos el e-vectorial
$V^{\otimes r}\otimes (V^*)^{\otimes s}$, llamamos a este espacio vectorial el \textbf{$(r,s)$-espacio tensorial sobre $V$}
y se llama a los vectores de este espacio vectorial un \textbf{$(r,s)$-tensor}.
\begin{notation}
  \
  \begin{itemize}
    \item Se escribe $\mathcal{T}^r_s(V)$ al $(r,s)$-espacio tensorial sobre $V$.
    \item $\mathcal{T}^0_0(V)$ es el cuerpo de los escalares del espacio vectorial $V$.
    \item $\mathcal{T}^0_1(V) = V$.
    \item $\mathcal{T}^1_0(V)=V^*$.
  \end{itemize}
\end{notation}

Si tenemos $\{x^\alpha\}$ base de $V$ y $\{f_\beta\}$ base de $V^*$, un $(r,s)$-tensor se expresa en la base
\begin{equation}
  \label{eq:r-s-tensor-componentes}
  v=v_{\alpha_1,\cdots,\alpha_r}^{\beta_1,\cdots, \beta_s} x^{\alpha_1}\otimes\cdots\otimes x^{\alpha_r}\otimes f_{\beta_1}\otimes\cdots\otimes f_{\beta_s}
\end{equation}
Y si $\{y^\alpha\}$ base de $V$ y $\{g_\beta\}$ base de $V^*$, con $A$ y $B$ las matrices de cambio de base, la expresión \ref{eq:tensores_cambio_base}
queda
\begin{equation}
  \label{eq:r-s-tensores_cambio_base}
  w_{\mu_1,\cdots,\mu_r}^{\nu_1,\cdots, \nu_s}=v_{\alpha_1,\cdots,\alpha_r}^{\beta_1,\cdots, \beta_s}A^{\alpha_1}_{\nu_1}\cdots A^{\alpha_r}_{\nu_r}B_{\beta_1}^{\mu_1}\cdots A_{\beta_s}^{\mu_s}
\end{equation}
Si además, tomamos en $V^*$ las respectivas base duales de $V$, por \ref{res:dual_cambio_base} la expresión \ref{eq:r-s-tensores_cambio_base}
queda
\begin{equation}
  \label{eq:r-s-tensores_cambio_base_dual}
  w_{\mu_1,\cdots,\mu_r}^{\nu_1,\cdots, \nu_s}=v_{\alpha_1,\cdots,\alpha_r}^{\beta_1,\cdots, \beta_s}A^{\alpha_1}_{\nu_1}\cdots A^{\alpha_r}_{\nu_r}(A^{-1})_{\beta_1}^{\mu_1}\cdots (A^{-1})_{\beta_s}^{\mu_s}
\end{equation}


  \chapter{Análisis}\label{ch:analisis}


  \section{Gradiente}\label{sec:gradiente}
  \section{Gradiente}\label{sec:gradiente}
\begin{definition}
  Dado un campo escalar $f$, se llama \textbf{gradiente}\label{def:gradiente} de $f$ y se escribe $\nabla
  f$, a la función $(n,n)$-suave que a cada punto $p$ le asigna el vector cuyas
  coordenadas cartesianas son las derivadas parciales de $f$ en $p$, $\nabla f(p)=\left({\frac
  {\partial f}{\partial x^\alpha}}(p)\right)_\alpha$.
\end{definition}

Considerando $\nabla$ como un operador, es lineal y cumple la regla del producto.

\begin{theorem}[Regla de la cadena]\label{th:regla-cadena}
  Si $f$ es un campo escalar en $\R^n$ y $\gamma$ una curva en $\R^n$, la derivada de la composición
es $(f\circ\gamma)^{'}(t)=\nabla f(\gamma(t))\gamma^{'}(t)$.
\end{theorem}

\section{Matriz Jacobiana}\label{sec:matriz-jacobiana}
\begin{definition}
  \glossarydef{matriz-jacobiana}{$\mathbf{J}_f$}{Matriz Jacobiana de una función f}
  Sea $f$ una función $(n,m)$-suave, se llama \textbf{matriz
  Jacobiana}\label{def:matriz-jacobiana} de $f$ y se escribe $\mathbf{J}_f$ a la matriz
  definida por
  $\mathbf{J}_f=\left({\frac {\partial f^\beta}{\partial x^\alpha}}\right)_\alpha^\beta$.
  Llamamos \textbf{Jacobiano}\label{def:jacobiano} de $f$ al determinante de la matriz Jacobiana.
\end{definition}
Cuando $f$ es un campo escalar en $\R^n$, la matriz jacobiana es el gradiente.
Además, cuando esté definida $f^{-1}$, se cumple que $\mathbf{J}_{f^{-1}}=\mathbf{J}_f^{-1}$.



  \chapter{Geometría diferencial}\label{ch:geometria-diferencial}
  \section{Aplicaciones diferenciables}\label{sec:aplicaciones-diferenciables}

\begin{definition}
  Sea $\cartalocal$, una función $\maps{f}{M}{\R}$ se dice
  \define{diferenciable en p}{función diferenciable} si la función $f\circ\varphi^{-1}$ es
  diferenciable en $\varphi(p)$.
  La función se llama diferenciable si lo es en todos los puntos del dominio.
\end{definition}

\begin{notation}
  Llamamos $\cinfinity{M}$ al conjunto de todas las funciones diferenciables de $M$ y
  $\cinfinity{p}$ al conjunto de todas las funciones diferenciables en $p$.
\end{notation}

\begin{definition}
  Sean $(M, p, U, \varphi)$ y $(N, q, V, \phi)$ dos cartas locales, una aplicación
  $\maps{f}{M}{N}$ con $f(p)=q$ se dice
  \define{diferenciable en p}{aplicación diferenciable} si la aplicación
  $\phi\circ f\circ\varphi^{-1}$ es diferenciable en $\varphi(p)$.
  La aplicación se llama diferenciable si lo es en todos los puntos del dominio.
\end{definition}

\begin{notation}
  Llamamos $\cinfinity{M, N}$ al conjunto de todas las aplicaciones diferenciables de $M$ en $N$ y
  $\cinfinity{p, q}$ al conjunto de todas las aplicaciones diferenciables en $p$.
\end{notation}

\begin{definition}
  Sea $\cartalocal$, llamamos \define{coordenadas locales}{coordenadas-locales} de la carta, a las
  aplicaciones $\maps{\varphi^\alpha}{U}{\R}$ definido por $\varphi^\alpha(q)=\pi^\alpha(\varphi
  (q))$
  donde $\pi^\alpha$ es la proyección canónica de $\R^n$ en $\R$.
\end{definition}


\section{Espacio tangente}\label{sec:espacio-tangente}
\begin{definition}
  Sea $\cartalocal$, diremos que $\gamma$ es una \define{curva centrada en
  $p$}{Curva centrada} si $\exists\ \epsilon\in\R \ \mid \maps{\gamma}{(-\epsilon,\epsilon)}{U}\land
  \gamma(0)=p$.
  Se llama \define{derivada de $\gamma$ en $p$} al vector
  \begin{equation}
    \label{eq:derivada-curva}
    \left.\frac{d(\varphi\circ\gamma)
    }{dt}\right|_{t=0}=((\varphi^\alpha\circ\gamma)^{'}(0))_\alpha.
  \end{equation}
\end{definition}
\begin{notation}
  En un abuso de notación escribimos $\gamma^{'}(0)$ al vector tangente a $\gamma$ en $p$.
\end{notation}

Considerando una función real definida en $M$ vamos a definir el espacio tangente en un
punto $p$ usando la noción de derivada direccional del vector tangente (velocidad) de una curva
$\gamma$ en el punto $p$.

\begin{definition}
  Sea $\cartalocal$ y $\phi$ una función diferenciable en un entorno de $p$, llamamos
  \define{derivada direccional}{Derivada direccional} de $f$ en la dirección $v\in\R^n$ a
  \begin{equation}
    \label{eq:derivada-direccional}
    \left.\frac{d(f\circ\gamma)}{dt}\right|_{t=0}
  \end{equation}
  donde $\gamma$ es una curva centrada en $p$ con $\gamma^{'}(0)=v$.
\end{definition}

Al desarrollar la expresión \ref{eq:derivada-direccional} como
$f\circ\gamma=f\circ\varphi^{-1}\circ\varphi\circ\gamma$, tenemos por un lado
$\maps{f\circ\varphi^{-1}}{\R^n}{\R}$
y por otro
$\maps{\varphi\circ\gamma}{\R}{\R^n}$, $(\varphi\circ\gamma)(t)=
(x^\alpha=\varphi^\alpha\circ\gamma)_{\alpha}$,
por la regla de la cadena \ref{th:regla-cadena} se tiene
\begin{multline*}
  \left.\frac{d(f\circ\gamma)}{dt}\right|_{t=0}=
  \left.\frac{d(f\circ\varphi^{-1}\circ\varphi\circ\gamma)}{d\varphi^\alpha}\right|_{t=0}=
  \left.\frac{\partial(f\circ\varphi^{-1})}{\partial x^\alpha}\right|_{\varphi(p)}\left.\frac{d
  x^\alpha  }{dt}\right|_{t=0}=\\
  =\left.\frac{\partial(f\circ\varphi^{-1})}{\partial x^\alpha}\right|_{\varphi(p)}\left.
  (\varphi^\alpha\circ\gamma)^{'}(0)=\left.\frac{\partial(f\circ\varphi^{-1})}{\partial x^\alpha}\right|_{\varphi(p)}\left.
  v_\alpha.
\end{multline*}

Como se ha comprobado, la definición de \ref{def:Derivada direccional} no depende de $\gamma$
sino sólo de $v$ y por tanto está bien definida y se puede hacer la siguiente construcción.

Sea $\cartalocal$ y $\gamma_1$ y $\gamma_2$ dos curvas centradas en $p$, se define la relación
$\gamma_1\sim\gamma_2$ cuando $\gamma_1^{'}(0)=\gamma_2^{'}(0)$.

\begin{exercise}
  \label{ex:relacion-equivalencia}
  La relación anterior es una relación de equivalencia.
\end{exercise}

Escribimos $\mathring{\gamma}$ a la clase de equivalencia de $\gamma$ en el conjunto cociente
$\cinfinity{p}/\sim$.

\begin{definition}
  Sea $\cartalocal$, $\gamma$ una función diferenciable y $\sim$ la relación de equivalencia
  descrita en~\ref{ex:relacion-equivalencia}.
  Se llama \define{vector tangente a la curva $\gamma$ en p}{Vector tangente a la curva} a la
  función
  \begin{equation}
    \label{eq:vector-tangente-curva}
    \begin{alignat*}{2}
      \mapsdef{\mathring{\gamma}}{\cinfinity{p}}{\R}{f}{\left.\frac{d(f\circ\gamma)}{dt}\right|_{t=0}}.
    \end{alignat*}
  \end{equation}
  Se llama \define{espacio tangente de $p$ en $M$}{Espacio tangente de una variedad}, $T_pM$ al
  conjunto de todas los vectores tangentes a todas las curvas centradas en $p$.
\end{definition}

\begin{exercise}
  Comprobar que con las operaciones $\mathring{\gamma_p}+\mathring{\gamma_p^\prime}=
  (\mathring{\gamma+\gamma^\prime})_p$ y $r\mathring{\gamma_p}=(\mathring{r\gamma})_p$, el espacio
  tangente tiene estructura de espacio vectorial.
\end{exercise}

Dados dos variedades diferenciables $M$ y $N$ y una aplicación diferenciable $\maps{f}{M}{N}$,
podemos definir una aplicación entre espacios tangentes a ambas variedades de forma natural, sin
más que considerar la composición de aplicaciones de la siguiente manera:
\begin{equation}
  \label{eq:diferencial-aplicacion}
  \begin{alignat*}{2}
    \mapsdef{d_pf}{Tp(M)}{T_{\varphi(p)}(N)}{\mathring{\gamma_p}}{\mathring{(f\circ\gamma)
    }_{\varphi
    (p)}}
  \end{alignat*}
\end{equation}

\begin{exercise}
  Comprobar que $d_pf$ está bien definido y que si $f$ es un difeomorfismo, entonces $d_pf$ es
  un isomorfismo de espacios vectoriales y su inversa es $(d_pf)^{-1}=d_{f(p)}f^{-1}$.
\end{exercise}

Como caso especial, si $N=\R$, entonces $\maps{d_pf}{T_pM}{\R}$, es decir $d_p f$ es un
elemento del dual del espacio tangente, $d_p f\in T_pM^*$


\section{Bases del espacio tangente}\label{sec:bases-del-espacio-tangente}
De forma natural tenemos una base del espacio tangente, usando las curvas definidas por las
coordenadas canónicas de $\R^n$, es decir, que si consideramos $\cartalocal$ y $\maps{
x^\alpha}{(-\epsilon, \epsilon)
\subset\R}{U\subset M}$ definido por $x^\alpha(t)=\varphi^{-1}(\varphi(p)+te^\alpha)$, las curvas en
$M$
definidas por las coordenadas canónicas $e^\alpha$,
vamos a comprobar que $\{\mathring{x_p^\alpha}\}$ es una base de $T_pM$.

\begin{proposition}
  Sea $\cartalocal$ y $\maps{
  x^\alpha}{(-\epsilon, \epsilon)
  \subset\R}{U\subset M}$ definido por $x^\alpha(t)=\varphi^{-1}(\varphi(p)+te^\alpha)$, las
  curvas en
  $M$
  definidas por las coordenadas canónicas $e^\alpha$, entonces
  $\mathring{x_p^\alpha}=\mathring{x_p^\beta}\so\alpha=\beta$.
\end{proposition}
\begin{proof}
  Que $\mathring{x_p^\alpha}=\mathring{x_p^\beta}$ significa que
  $x^\alpha\sim x^\beta$, es decir, que las derivadas de
  $\varphi\circ x^\alpha$ y $\varphi\circ x^\beta$ en 0 son iguales.
  Las coordenadas de estas funciones en $\R^n$ se expresan como:
  \begin{equation*}
    (\pi^\gamma\circ\varphi\circ x^\alpha)(t)=\pi^\gamma(\varphi(\varphi^{-1}(\varphi(p)
    +te^\alpha)))=\pi^\gamma(\varphi(p)
    +te^\alpha)=\varphi^\gamma(p)+\delta^{\gamma\alpha} t
  \end{equation*}
  Por lo tanto, las coordenadas de la derivada de $\varphi\circ x^\alpha$ es
  $(\pi^\gamma\circ\varphi\circ x^\alpha)^\prime(t)=\delta^{\gamma\alpha}$ y de la igualdad dada
  por la relación de equivalencia, tenemos que
  $\delta^{\gamma\alpha}=\delta^{\gamma\alpha}\so\alpha=\beta$.
\end{proof}

\begin{proposition}
  Sea $\cartalocal$ y $\maps{
  x^\alpha}{(-\epsilon, \epsilon)
  \subset\R}{U\subset M}$ definido por $x^\alpha(t)=\varphi^{-1}(\varphi(p)+te^\alpha)$, las curvas en
  $M$
  definidas por las coordenadas canónicas $e^\alpha$, entonces, toda familia de
  $\{\lambda_\alpha\}\subset\R$, tal que
  $\lambda_\alpha\mathring{x_p^\alpha}=0\so\lambda_\alpha=0\ \forall\ \alpha$.
\end{proposition}
\begin{proof}
  Como hemos visto anteriormente, la coordenada $\gamma$ de la derivada en $0$ de $x^\alpha$ es
  $\delta^{\gamma\alpha}$ y por tanto la coordenada $\gamma$ de la derivada en $0$ de
  $\lambda_\alpha x^\alpha$ es $\lambda_\alpha\delta^{\gamma\alpha}=\lambda_\gamma$ y como este
  debe ser igual a $0$ en cada coordenada se tiene que $\lambda_\gamma=0\ \forall\ \gamma$.
\end{proof}

\begin{proposition}
  Sea $\cartalocal$ y $\maps{
  x^\alpha}{(-\epsilon, \epsilon)
  \subset\R}{U\subset M}$ definido por $x^\alpha(t)=\varphi^{-1}(\varphi(p)+te^\alpha)$, las curvas en
  $M$
  definidas por las coordenadas canónicas $e^\alpha$, entonces, para toda curva $\gamma$ en $M$ con
  $\gamma(0)=p$, existe una familia
  $\{\lambda_\alpha\}\subset\R$, tal que
  $\mathring{f}=\lambda_\alpha\mathring{x_p^\alpha}=0\so\lambda_\alpha$.
\end{proposition}
\begin{proof}
  La derivada de $f$ en $0$ es un elemento de $\R^n$ con coordenadas $(f_\alpha)$, como hemos visto
  anteriormente, las coordenadas de $\lambda_\alpha x^\alpha$ es $(\lambda_\alpha)$, por lo que
  solo es necesario definir $\lambda_\alpha=f_\alpha$.
\end{proof}

Con estos tres resultados, hemos comprobado que $T_pM$ es un espacio vectorial de dimensión $n$
y además hemos definido una base que usa las coordenadas canónicas de $\R^n$.
Por eso dejamos de usar la notación de la clase de equivalencia para referirnos a un elemento del
espacio tangente y lo denotaremos como vectores.

Sin embargo, esta forma de caracterizar el espacio vectorial no es útil, y resulta más práctico
obtener una descripción del espacio tangente y de una base canónica en función de las coordenadas
locales.
Esto es lo que vamos a ver más adelante.


\section{Fibrado tangente}\label{sec:fibrado-tangente}
\begin{definition}
  Sea $M$ una variedad, llamamos \define{fibrado tangente}{Fibrado tangente} al
  conjunto $TM=\{(p,v)\ \mid\ p\in M\ \land\ v\in T_pM\}$
\end{definition}

\begin{proposition}
  El fibrado tangente de una variedad $n$-dimensional es una variedad $2n$-dimensional.
\end{proposition}

Si consideramos la proyección canónica de $\maps{\pi}{T(M)}{M}$, tenemos que para cada punto $p$ de
$M$ el espacio tangente es $T_pM=\pi^{-1}(p)$, vamos a generalizar este concepto.

\begin{definition}
  Un \define{paquete de vectores}{Paquete de vectores} de rango $m$ sobre una variedad $M$
  n-dimensional es una variedad $E$ junto a una aplicación diferenciable de $\maps{\pi}{E}{M}$ tal
  que para cada punto $p$ de $M$, $E_p=\pi^{-1}(p)$ es un espacio vectorial, además existe $U$ un
  entorno de $p$ en $M$ y un difeomorfismo $\maps{\phi}{\pi^{-1}(U)}{U\times\R^m}$ con la condición
  de que:
  \begin{itemize}
    \item $pr_1\circ\phi=\pi$.
    \item $\forall\ p\in M, pr_2\circ\phi\mid_E_p$ es un isomorfismo lineal.
  \end{itemize}
\end{definition}


\section{Campo de vectores}\label{sec:campo-de-vectores}
\begin{definition}
  Llamamos \define{campo de vectores}\label{def:campo-vectores} $X$ en $M$ al subconjunto de $T(M)$ formado por
  $X=\cup_{p\in M}(p,
  v)$.
\end{definition}

\begin{definition}
  Sea $X$ un campo de vectores sobre $M$ y $f\in\cinfinity{M}$, llamamos \define{derivada de Lie de
  f en X}{Derivada de Lie sobre una función} a:
  \begin{equation}
    \label{eq:derivada-lie-funcion-campo}
    \begin{alignat*}{2}
      \mapsdef{\mathscr{L}_X f}{M}{\R}{p}{d_p f(X(p))}
    \end{alignat*}
  \end{equation}
\end{definition}


\section{Espacio tangente como derivaciones}\label{sec:espacio-tangente-como-derivaciones}
\begin{definition}
  Sea $\cartalocal$ con coordenadas locales $(x^\alpha)$, sea $f\in\cinfinity{p}$, definimos la
  \define{derivada parcial}{Derivada parcial de una función} de $f$ con respecto a $x^\alpha$ en
  el punto $p$ por
  \begin{equation}
    \label{eq:parcial-funcion-punto}
    \frac{\partial f}{\partial x^\alpha}(p)=\frac{\partial (f\circ\varphi^{-1})}{\partial
    e^\alpha}(\varphi(p)).
  \end{equation}
  Donde $(e^\alpha)$ son las coordenadas canónicas de $\R^n$.
\end{definition}

\begin{definition}
  Sea $\cartalocal$ con coordenadas locales $(x^\alpha)$, definimos el
  \define{operador derivada parcial}{Operador derivada parcial de una función} con
  respecto a $x^\alpha$ en
  el punto $p$ al operador
  \begin{equation}
    \label{eq:operador-parcial-funcion-punto}
    \begin{alignat*}{2}
      \mapsdef{\left.\frac{\partial}{\partial x^\alpha}\right|_p}{\cinfinity{p}}{\R}{f}{\frac{\partial f}{\partial x^\alpha}(p)}
    \end{alignat*}
  \end{equation}
\end{definition}

\begin{notation}
  Denotamos al operador derivada parcial con respecto a $x^\alpha$ en $p$:
  \begin{itemize}
    \item Cuando no sea preciso indicar el punto $p$, por $\partial_{x^\alpha}$.
    \item Cuando tampoco sea preciso indicar las coordenadas, por $\partial^\alpha$ o $\partial_\alpha$.
  \end{itemize}
\end{notation}

\section{Diferencial de una función}\label{sec:diferencial-de-una-funcion}
\begin{definition}
  Sea $\cartalocal$ y $\mathring{\gamma_p}$ un elemento del espacio tangente en $p$, dada una
  función
  $f\in\cinfinity{M}$, definimos la \define{función diferencial de $f$ sobre
  $\mathring{\gamma_p}$}{diferencial de función en un punto} por
  \begin{equation}
    \label{eq:diferencial-funcion-punto}
    d_{\mathring{\gamma_p}}(f)=\frac{d}{dt}(f\circ\gamma)(t)\mid_{t=0}.
  \end{equation}
\end{definition}



  \chapter{Conexiones}\label{ch:conexiones}
  Una sección de un $s$-fibrado vectorial generaliza la noción de función sobre una variedad, en el sentido de
que cualquier
función $\maps{f}{M}{\R^n}$ se puede ver como una sección del $n$-fibrado vectorial trivial $M\times\R^n$.

¿Cómo podemos generalizar el concepto de diferenciación en secciones?, para que además se una extensión natural del
concepto de diferenciación sobre funciones.
No hay una respuesta única.

Para el caso conocido, se tiene una función $\maps{f}{\R^n}{\R^m}$, $x\in\R^n$ y $v\in T_x \R^n=\R^n$, la diferencial
de $f$ en la dirección de $v$ se calcula por $df_v(x)=\lim_{t\to 0}{\frac {f(x+tv)-f(x)}{t}}$.

Cuando se toma una sección $\mathcal{S}\in\Gamma(M, E)$, $p\in M$ y $\mathring{\gamma}\in T_p M$, no tiene sentido la
expresión $p+t\mathring{\gamma}$, aunque sí se puede hablar del cambio de \mathcal{S} a lo largo de la curva $\gamma$
y por tanto considerar $d\mathcal{S}_{\mathring{\gamma}}(p)=\lim_{t\to 0}\frac{\mathcal{S}(\gamma (t))-\mathcal{S}
(\gamma (0))}{t}$.

Sin embargo, la definición anterior aún no tiene sentido, porque $\mathcal{S}(\mathring{\gamma}(t))\in
E_{\mathring{\gamma}(t)}$ y $\mathcal{S}(\mathring{\gamma}(0))\in E_p$ que son dos espacios vectoriales distintos y
por tanto no puedes ser restados.

Este problema admite 3 formas distintas de solución:
\begin{itemize}
  \item \textbf{Transporte paralelo}.
  Aunque distintos, $E_{\gamma(t)}\cong E_p$, por tanto $\exists\ \maps{\phi_t}{E_{\gamma(t)
  }}{E_p}$ y de esta manera la expresión $\phi_t(\mathcal{S}(\gamma (t)))-\mathcal{S}(\gamma (0))$ tiene sentido.
  \[
    d\mathcal{S}_{\mathring{\gamma}}(p)=\lim_{t\to 0}\frac{\phi_t(\mathcal{S}(\gamma (t)))-\mathcal{S}
    (\gamma (0))}{t}
  \]

  \item \textbf{Ehresmann connection}.
  Use the notion of differential of a map of smooth manifolds.
  A section $s\in \Gamma (E)$ is by definition a smooth map $s:M\to E$ such
  that $\pi \circ s\operatorname{Id}$.
  This has a differential $ds:TM\to TE$, with the property that $ds(X)\in \Gamma (TE)$
 for a vector field $X\in \Gamma (TM)$.
  However, one would like instead for $ds(X)$ to be a section of $E$ itself.

  In fact, since the tangent space to a vector space is isomorphic to the space itself, the vertical bundle
  $V\subset TE$ of all tangent spaces to the fibres of $E$ is naturally isomorphic to a copy of $E$
  itself, or more precisely the pullback of $E$ along the projection $TE\to E$.
  If one chooses a projection $\maps{\nu}{TE}{V}$ from $TE$ to this vertical subbundle
  $V$ which is compatible with the linear structure of the fibres, composing with this projection would land
  $ds(X)$ back in $E$.


  \item \textbf{Derivada covariante}.
  Definir de forma axiomática que propiedades debe cumplir una derivación para secciones.
\end{itemize}

\begin{definition}
  Sea $(E, \pi)$ un $s$-fibrado vectorial sobre $M$, se llama conexión en $E$ a cualquier operador $\R$-lineal
  $\nabla$ sobre $\Gamma(M, E)$ con valores en $\Gamma(M, T^* M\otimes E)$ que cumpla la condición $\nabla(f\mathcal{S})=df\otimes \mathcal{S}+f\nabla(\mathcal{S})$\label{def:conexion-condicion} donde $f\in\cinfinity{M}$ y
  $\mathcal{S}\in\Gamma(M, E)$.
\end{definition}

\begin{example}
Si consideramos el fibrado lineal canónico~(\ref{def:fibrado-lineal-canonico}), se tiene que $\Gamma(M, E)
\cong\cinfinity{M}$ y la condición~\ref{def:conexion-condicion} se convierte en $\nabla(fg)=df\otimes g+f\nabla(g)$
para todo $f,g\in\cinfinity{M}$, tenemos así que la derivada exterior es una conexión sobre $E$.
\end{example}


  \bibliography{wiki}
  \bibliographystyle{plain}
\end{document}
