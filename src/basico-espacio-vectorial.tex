\begin{summary}
No necesitamos amplios conocimientos en espacios vectoriales y de forma implícita estaremos trabajando con espacios
vectoriales sobre $\R$, de dimensión finita y con un producto escalar definido, para una definición genérica de
espacio vectorial se puede consultar la wikipedia~\cite{wiki:espacio-vectorial}.

A lo largo del texto, cuando digamos que \glossarydef{espacio-vectorial}{$V$}{Espacio vectorial} es un espacio
vectorial, estamos diciendo que $V$ es un espacio vectorial sobre $\R$, de dimensión finita $n$ y con un producto
escalar definido.
Cuando sea importante indicar la dimensión del espacio vectorial lo denotaremos por
\glossarydef{espacio-vectorial-dimensional}{$V^n$}{Espacio vectorial de dimensión n}.

Una base del espacio vectorial $V$ lo denotamos por $\{v_\alpha\}$ sin indicar el valor que toma el índice $\alpha$,
pues daremos por entendido que toma todos los valores desde $1$ hasta la dimensión de $V$.

Todas las aplicaciones entre espacios vectoriales o entre el espacio vectorial y $\R$ son lineales.
\end{summary}

\section{Espacio dual}\label{sec:espacio-dual}\index{Espacio dual}
La noción de espacio dual \glossarydef{dual}{$V^*$}{Espacio dual de $V$}\cite{wiki:espacio-dual}, el conjunto de las aplicaciones lineales de $V$ sobre $\R$, es
sencillo y no merece la pena dedicarle mucho detalle, pero por la importancia que tiene en el desarrollo teórico de
la GTR vamos a recordar unas igualdades sencillas.

\begin{result}
  \label{res:coordenadas_duales}
  Sea $V$ un espacio vectorial con $\{e^\alpha\}$ una base de $V$ y $\{e_\alpha\}$ su base dual.
  $\forall v\in V\coma \forall f\in V^*$ se cumple que:
  \begin{itemize}
    \item $v=e_\alpha(v)e^\alpha$.
    \item $f=f(e^\alpha)e_\alpha$.
    \item $f(v)=e_\alpha(v)f(e^\alpha)$.
  \end{itemize}
\end{result}

\section{Dual del cambio de base}\label{sec:dual-del-cambio-de-base}
Sea $V$ un espacio vectorial con $\{e^\alpha\}$ y $\{e'^\alpha\}$ bases de $V$ y $A$ la matriz cambio de base.
Sea $\{f_\alpha\}$ y $\{f'_\alpha\}$ sus bases duales y $B$ la matriz cambio de base, entonces:
\[
  \delta_\alpha^\beta=f_\alpha(e^\beta)=B_\alpha^\mu f'_\mu(e^\beta)=B_\alpha^\mu A^\beta_\nu f'_\mu(e'^\nu)=B_\alpha^\mu A^\beta_\nu\delta^\nu_\mu=B_\alpha^\mu A^\beta_\mu=(BA)^\beta_\alpha.
\]
De forma análoga vemos que $\delta^\beta_\alpha=(AB)^\beta_\alpha$, y por tanto que $BA=AB=I_n$, se obtiene así el
siguiente resultado.

\begin{result}
  \label{res:dual_cambio_base}
  Sea $V$ un espacio vectorial con $\{e^\alpha\}$ y $\{e'^\alpha\}$ bases de $V$ y $A$ la matriz cambio de base.
  La matriz cambio de base de las bases duales $\{f_\alpha\}$ en $\{f'_\alpha\}$ es $A^{-1}$.

\end{result}
