\begin{definition}
  Sea $\mathcal{M}$ una variedad diferenciable, se llama \define{s-fibrado vectorial}{s-fibrado-vectorial} en $\mathcal{M}$ al
  par $(E\coma \pi)$ donde
  \begin{enumerate}
    \item $\maps{\pi}{E}{\mathcal{M}}$ es sobreyectiva.
    \item $\forall p\in \mathcal{M}\coma\pi^{-1}(p)$ es un espacio vectorial.
    \item $\forall p\in \mathcal{M}\coma\exists\ U\subset \mathcal{M}$ abierto y $\maps{\psi}{\pi^{-1}(U)}{U\times\R^s}$
    tal que los siguientes diagramas son conmutativos:
  \end{enumerate}
  \[
    \begin{tikzcd}
      \pi^{-1}(U)\arrow{r}{\psi}\arrow[hook]{d} & U\times\R^s\arrow{d}{pr_1}\\
      E\arrow{r}{\pi} & U
    \end{tikzcd}
    \begin{tikzcd}
      \pi^{-1}(p)\arrow{r}{\psi}\arrow{dr}{\simeq} & U\times\R^s\arrow{d}{pr_2}\\
       & \R^s
    \end{tikzcd}
  \]
\end{definition}
\begin{notation}
  Se escribe \glossarydef{p-fibrado}{$E_p$}{Fibrado en p}$=\pi^{-1}(p)$.
\end{notation}

\begin{definition}
Se llama \define{fibrado lineal canónico}{fibrado-lineal-canonico} al 1-fibrado vectorial $
(E=\mathcal{M}\times\R\coma\pi)$ donde
$\pi$
es la proyección canónica de $E$ en $\mathcal{M}$.
\end{definition}

\begin{definition}
  Sea $\mathcal{M}$ una variedad, se llama \define{fibrado tangente}{fibrado-tangente} al conjunto $\bigcup_{p\in \mathcal{M}}\{p\}\times T_p\mathcal{M}$.
  Se llama \define{dual del fibrado tangente}{dual-fibrado-tangente} al conjunto $\bigcup_{p\in \mathcal{M}}\{p\}\times T^*_p\mathcal{M}$.
\end{definition}
\begin{notation}
  Se escribe \glossarydef{fibrado-tangente}{$T\mathcal{M}$}{Fibrado tangente de $\mathcal{M}$} al fibrado tangente de $\mathcal{M}$.
  Se escribe \glossarydef{fibrado-dual}{$T^* \mathcal{M}$}{Dual del fibrado de $\mathcal{M}$} al dual del fibrado
  tangente de $\mathcal{M}$.
\end{notation}

\begin{exercise}
  El fibrado tangente y el dual del fibrado tangente de una variedad $n$-dimensional son variedades
  $2n$-dimensionales.
\end{exercise}

\begin{proposition}
El fibrado tangente de una variedad $n$-dimensional es un $n$-fibrado tangente de $\mathcal{M}$.
\end{proposition}

\begin{exercise}
  Demostrar que $T^*\mathcal{M}$ es un $n$-fibrado tangente de $\mathcal{M}$.
\end{exercise}

\begin{definition}
  Sea $(E\coma\pi)$ un $s$-fibrado vectorial sobre la variedad $\mathcal{M}$, se llama \define{sección de $\mathcal{M}$ en
  E}{sección-fibrado-tangente} a la
  aplicación $\maps{\mathcal{S}}{\mathcal{M}}{E}\ \mid\ \pi\circ\mathcal{S}=\operatorname{Id}$.
\end{definition}
\begin{notation}
  Se escribe $\Gamma(\mathcal{M},E)$ al conjunto de todas las secciones de $\mathcal{M}$ en $E$.
\end{notation}
