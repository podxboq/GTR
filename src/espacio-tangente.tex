Recordemos que dada una función $f\in\cinfinity{\R^n\coma\R^m}$, le podemos asociar a cada punto $x\in\R^n$ la
derivada de la función en $x$ que denotamos por $Df_x$.
La estructura afín de $\R^n$ permite interpretar un vector $v\in\R^n$ como un vector tangente a $x$ asociando a cada
función real $g\in\cinfinity{\R^n}$ la derivada direccional $D_{v}g_x=Dg_x(v)$.
De este modo podemos interpretar cada vector tangente como una aplicación que asocia a cada función real de clase
$\cinfinity$ un valor real.

De forma análoga vamos a definir un vector tangente a un punto $p$ de una variedad diferenciable, es decir, que un
vector tangente será una aplicación que a cada función $\cinfinity{p}$ le aplica un número real y que además cumpla
con las propiedades que caracterizan a las derivaciones en $\R^n$, a saber, la linearidad y la regla de Leibniz.

\section{Espacio tangente}\label{sec:espacio-tangente}

\begin{definition}
  Sea $\cartalocal$ y $\maps{\mathrm{v}}{\cinfinity{p}}{\R}$, diremos que $\mathrm{v}$ es un \define{vector tangente en
  $p$}{Vector tangente en p}, si es $\R$-lineal y cumple la regla de Leibniz.
  Llamamos \glossarydef{espacio-tangente}{$T_p\mathcal{M}$}{Espacio tangente de p en $\mathcal{M}$}
  \define{espacio tangente de
  $p$ en $\mathcal{M}$}{def:espacio-tangente}, al conjunto de todos los vectores tangentes en $p$.
\end{definition}

Por definición, y por lo comentado en~\ref{propiedad-derivada-parcial}, los operadores $\partial^\alpha$ derivada
parcial con respecto a la coordenada $\varphi^\alpha$ son vectores tangentes en $p$, es más, dada cualquier curva
$\gamma$ centrada en $p$ sobre $\mathcal{M}$ tiene asociado un vector tangente de forma natural, pues a cada función
$f\in\cinfinity{p}$ tenemos según~\ref{eq:derivada-parcial-curva} $\gamma^{'}(f)$ haciendo que podamos definir la
aplicación $\maps{\gamma^{'}}{\cinfinity{p}}{\R}$ que es $\R$-lineal y cumple la regla de Leibniz y por tanto por
definición $\gamma^{'}$ es un vector tangente en $p$.

Por otra parte fijado $\alpha$, y tomando $\epsilon\in\R\tq (-\epsilon, \epsilon)\subset\varphi^{-1}(U)$, podemos
definir la siguiente curva centrada en $p$, $\gamma_\alpha(t)=\varphi^{-1}(\varphi(p)+te^\alpha)$, vamos a calcular cual
es el vector tangente a $\gamma_\alpha$ en $p$.
Dado $f\in\cinfinity{p}$

\begin{multline}
\gamma_\alpha^{'}(f)\by{\ref{eq:derivada-parcial-curva}}
  \frac{d(f\circ\gamma_\alpha)}{dt}(0)=
  \frac{d(f\circ\varphi^{-1}\circ\varphi\circ\gamma_\alpha)}{dt}(0)=\\
  =\frac{\partial(f\circ\varphi^{-1})}{\partial u^\beta}(\varphi(p))
  \frac{d u^\beta(\varphi\circ\gamma_\alpha)}{dt}(0)\by{\ref{def:Derivada parcial de una función}}
  \frac{\partial f}{\partial \varphi^\beta}(p)\delta^{\alpha\beta}
=\partial^\alpha(f)
\end{multline}

Acabamos de ver, que $\partial^\alpha$ es un vector tangente a una curva en $p$, exactamente de la $\alpha$-ésima
curva local.
Ahora vamos a ver como se expresa cualquier vector.

\begin{multline}\label{eq:imagen-vector-funcion}
  \gamma^{'}(f)\by{\ref{eq:derivada-parcial-curva}}
  \frac{d(f\circ\gamma)}{dt}(0)=
  \frac{d(f\circ\varphi^{-1}\circ\varphi\circ\gamma)}{dt}(0)=\\
  =\frac{\partial(f\circ\varphi^{-1})}{\partial u^\beta}(\varphi(p))
  \frac{d u^\beta(\varphi\circ\gamma)}{dt}(0)\by{\ref{def:Derivada parcial de una función}}
  \frac{d(\varphi^\beta\circ\gamma)}{dt}(0)\frac{\partial f}{\partial \varphi^\beta}(p)
  =(\frac{d(\varphi^\beta\circ\gamma)}{dt}(0)\partial^\beta)(f)
\end{multline}
