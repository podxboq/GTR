Una sección de un $s$-fibrado vectorial generaliza la noción de función sobre una variedad, en el sentido de
que cualquier
función $\maps{f}{M}{\R^n}$ se puede ver como una sección del $n$-fibrado vectorial trivial $M\times\R^n$.

¿Cómo podemos generalizar el concepto de diferenciación en secciones?, para que además se una extensión natural del
concepto de diferenciación sobre funciones.
No hay una respuesta única.

Para el caso conocido, se tiene una función $\maps{f}{\R^n}{\R^m}$, $x\in\R^n$ y $v\in T_x \R^n=\R^n$, la diferencial
de $f$ en la dirección de $v$ se calcula por $df_v(x)=\lim_{t\to 0}{\frac {f(x+tv)-f(x)}{t}}$.

Cuando se toma una sección $\mathcal{S}\in\Gamma(M, E)$, $p\in M$ y $\mathring{\gamma}\in T_p M$, no tiene sentido la
expresión $p+t\mathring{\gamma}$, aunque sí se puede hablar del cambio de \mathcal{S} a lo largo de la curva $\gamma$
y por tanto considerar $d\mathcal{S}_{\mathring{\gamma}}(p)=\lim_{t\to 0}\frac{\mathcal{S}(\gamma (t))-\mathcal{S}
(\gamma (0))}{t}$.

Sin embargo, la definición anterior aún no tiene sentido, porque $\mathcal{S}(\mathring{\gamma}(t))\in
E_{\mathring{\gamma}(t)}$ y $\mathcal{S}(\mathring{\gamma}(0))\in E_p$ que son dos espacios vectoriales distintos y
por tanto no puedes ser restados.

Este problema admite 3 formas distintas de solución:
\begin{itemize}
  \item \textbf{Transporte paralelo}.
  Aunque distintos, $E_{\gamma(t)}\cong E_p$, por tanto $\exists\ \maps{\phi_t}{E_{\gamma(t)
  }}{E_p}$ y de esta manera la expresión $\phi_t(\mathcal{S}(\gamma (t)))-\mathcal{S}(\gamma (0))$ tiene sentido.
  \[
    d\mathcal{S}_{\mathring{\gamma}}(p)=\lim_{t\to 0}\frac{\phi_t(\mathcal{S}(\gamma (t)))-\mathcal{S}
    (\gamma (0))}{t}
  \]

  \item \textbf{Ehresmann connection}.
  Use the notion of differential of a map of smooth manifolds.
  A section $s\in \Gamma (E)$ is by definition a smooth map $s:M\to E$ such
  that $\pi \circ s\operatorname{Id}$.
  This has a differential $ds:TM\to TE$, with the property that $ds(X)\in \Gamma (TE)$
 for a vector field $X\in \Gamma (TM)$.
  However, one would like instead for $ds(X)$ to be a section of $E$ itself.

  In fact, since the tangent space to a vector space is isomorphic to the space itself, the vertical bundle
  $V\subset TE$ of all tangent spaces to the fibres of $E$ is naturally isomorphic to a copy of $E$
  itself, or more precisely the pullback of $E$ along the projection $TE\to E$.
  If one chooses a projection $\maps{\nu}{TE}{V}$ from $TE$ to this vertical subbundle
  $V$ which is compatible with the linear structure of the fibres, composing with this projection would land
  $ds(X)$ back in $E$.


  \item \textbf{Derivada covariante}.
  Definir de forma axiomática que propiedades debe cumplir una derivación para secciones.
\end{itemize}

\begin{definition}
  Sea $(E, \pi)$ un $s$-fibrado vectorial sobre $M$, se llama conexión en $E$ a cualquier operador $\R$-lineal
  $\nabla$ sobre $\Gamma(M, E)$ con valores en $\Gamma(M, T^* M\otimes E)$ que cumpla la condición $\nabla(f\mathcal{S})=df\otimes \mathcal{S}+f\nabla(\mathcal{S})$\label{def:conexion-condicion} donde $f\in\cinfinity{M}$ y
  $\mathcal{S}\in\Gamma(M, E)$.
\end{definition}

\begin{example}
Si consideramos el fibrado lineal canónico~(\ref{def:fibrado-lineal-canonico}), se tiene que $\Gamma(M, E)
\cong\cinfinity{M}$ y la condición~\ref{def:conexion-condicion} se convierte en $\nabla(fg)=df\otimes g+f\nabla(g)$
para todo $f,g\in\cinfinity{M}$, tenemos así que la derivada exterior es una conexión sobre $E$.
\end{example}
