\chapter{Notaci\'on}\label{ch:notacion}
\begin{itemize}
  \item Diremos que $V$ es un e-vectorial para indicar que es un espacio vectorial
  de dimensión finita y con un producto escalar $\pescalar{}{}$ sobre un cuerpo $K$.
  \item Si $V$ es un espacio vectorial, llamamos $V^*$ a su espacio dual.
  \item Diremos que una función de $\R^n$ en $\R^m$ es $(n,m)$-suave si se puede diferencial indefinidamente.
  \item Diremos que $M$ es una variedad diferenciable de dimensión $n$ con una métrica.
  \item La cuaterna $(M, p, U, \phi)$ es carta local si $M$ es un variedad diferenciable, $p\in U\subset M$ con $U$ abierto en $M$ y $\maps{\phi}{U}{\R^n}$.
  \item Usaremos la función delta de kronecker con los índices en sus variantes
  $\delta_{\alpha\beta}=\delta_\alpha^\beta=\delta^{\alpha\beta}$.
  \item Dada una matriz $A$ describiremos a sus entradas con los índices en sus variantes
  $A=(A_{\alpha\beta})_{\alpha\beta}=(A_\alpha^\beta)_\alpha^\beta=(A^{\alpha\beta})^{\alpha\beta}$.
  \item Si $V$ es un espacio vectorial, denotaremos
  $V^{\otimes r}=V\otimes\overbrace{\cdots}^\text{r\ veces}\otimes V$.
\end{itemize}


\section{\'Indices}\label{sec:indices}
Principalmente para simplificar la notación al trabajar con índices, vamos
a introducir algunos conceptos y establecer una notación que haga más cómodo
trabajar con variables multi-indexadas.

\begin{notation}
  Denotaremos por:
  \begin{itemize}
    \item $I_n$ al conjunto de los naturales menores o iguales a $n$.
    \item $I_n^m$ al producto cartesiano $m$-veces de $I_n$.
    \item $P_n^m=\{(i_1,\cdots,i_m)\in I_n^m\ \mid \ i_k\neq 0\ \forall\ \indexdots{k}{1}{m}\}$
    \item $H_{kn}^m(l)=\{(i_1,\cdots,i_m)\in I_n^m\ \mid \ i_k=l\}$
  \end{itemize}
\end{notation}

\begin{notation}
  Si $\sigma=(1,\ldots,n)$, con la expresión $x^{(\sigma)}$, estaremos escribiendo la $n$-tupla $(x^1,\ldots,x^n)$.
  Sin embargo con la expresión $x^\sigma$, estaremos escribiendo $x^{1,\ldots,n}$.
\end{notation}

\subsection{Criterio de índices}\label{subsec:criterio-de-indices}

Emplearemos continuamente el \textbf{convenio de sumas de Einstein}\index{convenio de sumas de Einstein} según el cual
índices repetidos arriba y abajo en una expresión se supone sumado en todos sus posibles valores,
así la expresión $y=\sum_{\alpha=1}^3 c_\alpha x^\alpha=c_1 x^1 + c_2 x^2 + c_3 x^3$
se simplifica por la convención a $y = c_\alpha x^\alpha$.


\chapter{Álgebra lineal}\label{ch:algebra-lineal}


\section{Base dual}\label{sec:base-dual}\index{Base dual}

Sea $V$ un e-vectorial con $\{e^\alpha\}$ una base de $V$.
Definamos las aplicaciones lineales $e_\alpha: V \longrightarrow K$ actuando sobre la base
como $e_\alpha(e^\beta)=\delta_\alpha^\beta$, con esta definición, dado $v=v_\alpha e^\alpha\in V$ tenemos
que $e_\alpha(v)=v_\alpha$.

\begin{exercise}
  Las aplicaciones lineales $\{e_\alpha\}$ definidas anteriormente forman una base de $V^*$.
\end{exercise}

\begin{definition}
  Las aplicaciones lineales $\{e_\alpha\}$ definidas anteriormente las llamamos base dual de $\{e^\alpha\}$.
\end{definition}

Sea $f\in V^*$, en la base dual $f=f^\alpha e_\alpha$.
Dado $v=v_\beta e^\beta\in V$ tenemos los siguientes resultados:

\begin{equation*}
  f(v)=v_\beta f(e^\beta)=v_\beta f^\alpha e_\alpha(e^\beta)=
  v_\beta f^\alpha \delta_\alpha^\beta=v_\alpha f^\alpha
\end{equation*}

\begin{equation*}
  f(v)=v_\beta f(e^\beta)=e_\beta(v)f(e^\beta)=(f(e^\beta) e_\beta)(v)\so f=f(e^\beta) e_\beta\so f^\beta=f(e^\beta)
\end{equation*}

\begin{proposition}
  \label{res:coordenadas_duales}
  Sea $V$ un e-vectorial con $\{e^\alpha\}$ una base de $V$ y $\{e_\alpha\}$ su base dual.
  $\forall v\in V, \forall f\in V^*$ se cumple que:
  \begin{itemize}
    \item $v=e_\alpha(v)e^\alpha$.
    \item $f=f(e^\alpha)e_\alpha$.
    \item $f(v)=e_\alpha(v)f(e^\alpha)$.
  \end{itemize}
\end{proposition}


\section{Dual del cambio de base}\label{sec:dual-del-cambio-de-base}
Sea $V$ un e-vectorial con $\{e^\alpha\}$ y $\{e'^\alpha\}$ bases de $V$ y $A$ la matriz cambio de base.
Sea $\{f_\alpha\}$ y $\{f'_\alpha\}$ sus bases duales y $B$ la matriz cambio de base, entonces:
\[
  \delta_\alpha^\beta=f_\alpha(e^\beta)=B_\alpha^\mu f'_\mu(e^\beta)=B_\alpha^\mu A^\beta_\nu f'_\mu(e'^\nu)=B_\alpha^\mu A^\beta_\nu\delta^\nu_\mu=B_\alpha^\mu A^\beta_\mu=(BA)^\beta_\alpha.
\]
De forma análoga vemos que $\delta^\beta_\alpha=(AB)^\beta_\alpha$, y por tanto que $BA=AB=I_n$, obteniendo así el siguiente resultado.

\begin{proposition}
  \label{res:dual_cambio_base}
  Sea $V$ un e-vectorial con $\{e^\alpha\}$ y $\{e'^\alpha\}$ bases de $V$ y $A$ la matriz cambio de base.
  La matriz cambio de base de las bases duales $\{f_\alpha\}$ en $\{f'_\alpha\}$ es $A^{-1}$.

\end{proposition}


\chapter{Tensores}\label{ch:tensores}
\section{Tensores como vectores}\label{sec:tensores-como-vectores}
Sean $V_1,\ldots,V_n$ espacios vectoriales de dimensión $\dim(V_i)=n_i$ y
$\{x_i^\alpha\}_{\indexdots{\alpha}{1}{n_i}}$ base de $V_i$.
Si consideramos el producto tensorial, tenemos un nuevo espacio vectorial $V=V_1\otimes\cdots\otimes V_n$
de dimensión $\sum_{\indexdots{\alpha}{1}{n}} n_\alpha$ y con base
$\{x_1^{\alpha_1}\otimes\cdots\otimes x_n^{\alpha_n}\}_{\indexdots{\alpha_i}{1}{n_i}}$.

\begin{definition}
  Llamamos \define{tensor}{tensor} a los vectores del producto tensorial de espacios vectoriales.
\end{definition}

\begin{remark}
  Hay que comprobar que $\forall v\in V, \exists v_1\in V_1, \cdots v_n\in V_n\ |\ v=v_1\otimes\cdots\otimes v_n$.
\end{remark}

Por tanto un tensor se expresa en la base $B$ como $v=v_{\alpha_1,\cdots,\alpha_n} x_1^{\alpha_1}\otimes\cdots\otimes x_n^{\alpha_n}$.

\section{Cambio de base en tensores}\label{sec:cambio-de-base-en-tensores}
Sean $V=V_1\otimes\cdots\otimes V_n$ espacio tensorial y $\{x_i^\alpha\}$, $\{y_i^\alpha\}$ bases de $V_i$, llamemos $A_i$
la matriz cambio de base de $\{x_i^\alpha\}$, $\{y_i^\alpha\}$.
Sea $v=v_1\otimes\cdots\otimes v_n \in V$ que con respecto a la base
$B=\{x_1^{\alpha_1}\otimes\cdots\otimes x_n^{\alpha_n},\ \alpha_i=1,\cdots,n_\alpha \}$
se expresa como $v=v_{\alpha_1,\cdots,\alpha_n} x_1^{\alpha_1}\otimes\cdots\otimes x_n^{\alpha_n}$ y
que con respecto a la base $C=\{y_1^{\alpha_1}\otimes\cdots\otimes y_n^{\alpha_n},\ \alpha_i=1,\cdots,n_\alpha \}$
se expresa como $v=w_{\alpha_1,\cdots,\alpha_n} y_1^{\alpha_1}\otimes\cdots\otimes y_n^{\alpha_n}$.

La relación entre las coordenadas $v_{\alpha_1,\cdots,\alpha_n}$ y $w_{\alpha_1,\cdots,\alpha_n}$
vienen expresadas por la propiedad multilineal del producto tensorial, puesto
que $x_i^{\alpha_i}=(A_i)^{\alpha_i}_\beta y_i^{\beta}$ se tiene que

\begin{multline*}
  v=v_{\alpha_1,\cdots,\alpha_n} x_1^{\alpha_1}\otimes\cdots\otimes x_n^{\alpha_n}=
  v_{\alpha_1,\cdots,\alpha_n} ((A_1)^{\alpha_1}_{\beta_1}y_1^{\beta_1})\otimes\cdots\otimes ((A_n)^{\alpha_n}_{\beta_n}y_n^{\beta_n})=\\
  =v_{\alpha_1,\cdots,\alpha_n}(A_1)^{\alpha_1}_{\beta_1}\cdots (A_n)^{\alpha_n}_{\beta_n} y_1^{\beta_1}\otimes\cdots\otimes y_n^{\beta_n}
\end{multline*}

Por tanto se tiene la igualdad
\begin{equation}
  \label{eq:tensores_cambio_base}
  w_{\beta_1,\cdots,\beta_n}=v_{\alpha_1,\cdots,\alpha_n}(A_1)^{\alpha_1}_{\beta_1}\cdots (A_n)^{\alpha_n}_{\beta_n}
\end{equation}

\subsection{Tensores sobre un único espacio vectorial y su
dual}\label{subsec:tensores-sobre-un-unico-espacio-vectorial-y-su-dual}
Como caso especial, vamos a considerar la situación en la que tomamos el e-vectorial
$V^{\otimes r}\otimes (V^*)^{\otimes s}$, llamamos a este espacio vectorial el \textbf{$(r,s)$-espacio tensorial sobre $V$}
y se llama a los vectores de este espacio vectorial un \textbf{$(r,s)$-tensor}.
\begin{notation}
  \
  \begin{itemize}
    \item Se escribe $\mathcal{T}^r_s(V)$ al $(r,s)$-espacio tensorial sobre $V$.
    \item $\mathcal{T}^0_0(V)$ es el cuerpo de los escalares del espacio vectorial $V$.
    \item $\mathcal{T}^0_1(V) = V$.
    \item $\mathcal{T}^1_0(V)=V^*$.
  \end{itemize}
\end{notation}

Si tenemos $\{x^\alpha\}$ base de $V$ y $\{f_\beta\}$ base de $V^*$, un $(r,s)$-tensor se expresa en la base
\begin{equation}
  \label{eq:r-s-tensor-componentes}
  v=v_{\alpha_1,\cdots,\alpha_r}^{\beta_1,\cdots, \beta_s} x^{\alpha_1}\otimes\cdots\otimes x^{\alpha_r}\otimes f_{\beta_1}\otimes\cdots\otimes f_{\beta_s}
\end{equation}
Y si $\{y^\alpha\}$ base de $V$ y $\{g_\beta\}$ base de $V^*$, con $A$ y $B$ las matrices de cambio de base, la expresión~\ref{eq:tensores_cambio_base}
queda
\begin{equation}
  \label{eq:r-s-tensores_cambio_base}
  w_{\mu_1,\cdots,\mu_r}^{\nu_1,\cdots, \nu_s}=v_{\alpha_1,\cdots,\alpha_r}^{\beta_1,\cdots, \beta_s}A^{\alpha_1}_{\nu_1}\cdots A^{\alpha_r}_{\nu_r}B_{\beta_1}^{\mu_1}\cdots A_{\beta_s}^{\mu_s}
\end{equation}
Si además, tomamos en $V^*$ las respectivas base duales de $V$, por~\ref{res:dual_cambio_base} la expresión~\ref{eq:r-s-tensores_cambio_base}
queda
\begin{equation}
  \label{eq:r-s-tensores_cambio_base_dual}
  w_{\mu_1,\cdots,\mu_r}^{\nu_1,\cdots, \nu_s}=v_{\alpha_1,\cdots,\alpha_r}^{\beta_1,\cdots, \beta_s}A^{\alpha_1}_{\nu_1}\cdots A^{\alpha_r}_{\nu_r}(A^{-1})_{\beta_1}^{\mu_1}\cdots (A^{-1})_{\beta_s}^{\mu_s}
\end{equation}

\chapter{An\'alisis}\label{ch:analisis}

\section{Gradiente}\label{sec:gradiente}
\begin{definition}
  Dado un campo escalar $f$, se llama \textbf{gradiente}\label{def:gradiente} de $f$ y se escribe $\nabla
  f$, a la función $(n,n)$-suave que a cada punto $p$ le asigna el vector cuyas
  coordenadas cartesianas son las derivadas parciales de $f$ en $p$, $\nabla f(p)=\left({\frac
  {\partial f}{\partial x^\alpha}}(p)\right)_\alpha$.
\end{definition}

Considerando $\nabla$ como un operador, es lineal y cumple la regla del producto.

\begin{theorem}[Regla de la cadena]\label{th:regla-cadena}
  Si $f$ es un campo escalar en $\R^n$ y $\gamma$ una curva en $\R^n$, la derivada de la composición
es $(f\circ\gamma)^{'}(t)=\nabla f(\gamma(t))\gamma^{'}(t)$.
\end{theorem}

\section{Matriz Jacobiana}\label{sec:matriz-jacobiana}
\begin{definition}
  Sea $f$ una función $(n,m)$-suave, se llama \textbf{matriz
  Jacobiana}\label{def:matriz-jacobiana} de $f$ y se escribe $\mathbf{J}_f$ a la matriz
  definida por
  $\mathbf{J}_f=\left({\frac {\partial f^\beta}{\partial x^\alpha}}\right)_\alpha^\beta$.
  Llamamos \textbf{Jacobiano}\label{def:jacobiano} de $f$ al determinante de la matriz Jacobiana.
\end{definition}
Cuando $f$ es un campo escalar en $\R^n$, la matriz jacobiana es el gradiente.
Además, cuando esté definida $f^{-1}$, se cumple que $\mathbf{J}_{f^{-1}}=\mathbf{J}_f^{-1}$.


\chapter{Geometría diferencial}\label{ch:geometria-diferencial}
\section{Aplicaciones diferenciables}\label{sec:aplicaciones-diferenciables}

\begin{definition}
  Sea $\cartalocal$, una función $\maps{f}{M}{\R}$ se dice
  \define{diferenciable en p}{función diferenciable} si la función $f\circ\varphi^{-1}$ es
  diferenciable en $\varphi(p)$.
  La función se llama diferenciable si lo es en todos los puntos del dominio.
\end{definition}

\begin{notation}
  Llamamos $\cinfinity{M}$ al conjunto de todas las funciones diferenciables de $M$ y
  $\cinfinity{p}$ al conjunto de todas las funciones diferenciables en $p$.
\end{notation}

\begin{definition}
  Sean $(M, p, U, \varphi)$ y $(N, q, V, \phi)$ dos cartas locales, una aplicación
  $\maps{f}{M}{N}$ con $f(p)=q$ se dice
  \define{diferenciable en p}{aplicación diferenciable} si la aplicación
  $\phi\circ f\circ\varphi^{-1}$ es diferenciable en $\varphi(p)$.
  La aplicación se llama diferenciable si lo es en todos los puntos del dominio.
\end{definition}

\begin{notation}
  Llamamos $\cinfinity{M, N}$ al conjunto de todas las aplicaciones diferenciables de $M$ en $N$ y
  $\cinfinity{p, q}$ al conjunto de todas las aplicaciones diferenciables en $p$.
\end{notation}

\begin{definition}
  Sea $\cartalocal$, llamamos \define{coordenadas locales}{coordenadas-locales} de la carta, a las
  aplicaciones $\maps{\varphi^\alpha}{U}{\R}$ definido por $\varphi^\alpha(q)=\pi^\alpha(\varphi
  (q))$
  donde $\pi^\alpha$ es la proyección canónica de $\R^n$ en $\R$.
\end{definition}


\section{Espacio tangente}\label{sec:espacio-tangente}
\begin{definition}
  Sea $\cartalocal$, diremos que $\gamma$ es una \define{curva centrada en
  $p$}{Curva centrada} si $\exists\ \epsilon\in\R \ \mid \maps{\gamma}{(-\epsilon,\epsilon)}{U}\land
  \gamma(0)=p$.
  Se llama \define{derivada de $\gamma$ en $p$} al vector
  \begin{equation}
    \label{eq:derivada-curva}
    \left.\frac{d(\varphi\circ\gamma)
    }{dt}\right|_{t=0}=((\varphi^\alpha\circ\gamma)^{'}(0))_\alpha.
  \end{equation}
\end{definition}
\begin{notation}
  En un abuso de notación escribimos $\gamma^{'}(0)$ al vector tangente a $\gamma$ en $p$.
\end{notation}

Considerando una función real definida en $M$ vamos a definir el espacio tangente en un
punto $p$ usando la noción de derivada direccional del vector tangente (velocidad) de una curva
$\gamma$ en el punto $p$.

\begin{definition}
  Sea $\cartalocal$ y $\phi$ una función diferenciable en un entorno de $p$, llamamos
  \define{derivada direccional}{Derivada direccional} de $f$ en la dirección $v\in\R^n$ a
  \begin{equation}
    \label{eq:derivada-direccional}
    \left.\frac{d(f\circ\gamma)}{dt}\right|_{t=0}
  \end{equation}
  donde $\gamma$ es una curva centrada en $p$ con $\gamma^{'}(0)=v$.
\end{definition}

Al desarrollar la expresión \ref{eq:derivada-direccional} como
$f\circ\gamma=f\circ\varphi^{-1}\circ\varphi\circ\gamma$, tenemos por un lado
$\maps{f\circ\varphi^{-1}}{\R^n}{\R}$
y por otro
$\maps{\varphi\circ\gamma}{\R}{\R^n}$, $(\varphi\circ\gamma)(t)=
(x^\alpha=\varphi^\alpha\circ\gamma)_{\alpha}$,
por la regla de la cadena \ref{th:regla-cadena} se tiene
\begin{multline*}
  \left.\frac{d(f\circ\gamma)}{dt}\right|_{t=0}=
  \left.\frac{d(f\circ\varphi^{-1}\circ\varphi\circ\gamma)}{d\varphi^\alpha}\right|_{t=0}=
  \left.\frac{\partial(f\circ\varphi^{-1})}{\partial x^\alpha}\right|_{\varphi(p)}
  \left.\frac{dx^\alpha}{dt}\right|_{t=0}=\\
  =\left.\frac{\partial(f\circ\varphi^{-1})}{\partial x^\alpha}\right|_{\varphi(p)}
  (\varphi^\alpha\circ\gamma)^{'}(0)=
  \left.\frac{\partial(f\circ\varphi^{-1})}{\partial x^\alpha}\right|_{\varphi(p)}v_\alpha.
\end{multline*}

Como se ha comprobado, la definición de \ref{def:Derivada direccional} no depende de $\gamma$
sino sólo de $v$ y por tanto está bien definida y se puede hacer la siguiente construcción.

Sea $\cartalocal$ y $\gamma_1$ y $\gamma_2$ dos curvas centradas en $p$, se define la relación
$\gamma_1\sim\gamma_2$ cuando $\gamma_1^{'}(0)=\gamma_2^{'}(0)$.

\begin{exercise}
  \label{ex:relacion-equivalencia}
  La relación anterior es una relación de equivalencia.
\end{exercise}

Escribimos $\mathring{\gamma}$ a la clase de equivalencia de $\gamma$ en el conjunto cociente
$\cinfinity{p}/\sim$.

\begin{definition}
  Sea $\cartalocal$, $\gamma$ una función diferenciable y $\sim$ la relación de equivalencia
  descrita en~\ref{ex:relacion-equivalencia}.
  Se llama \define{vector tangente a la curva $\gamma$ en p}{Vector tangente a la curva} a la
  función
  \begin{equation}
    \label{eq:vector-tangente-curva}
      \mapsdef{\mathring{\gamma}}{\cinfinity{p}}{\R}{f}{\left.\frac{d(f\circ\gamma)}{dt}\right|_{t=0}}.
  \end{equation}
  Se llama \define{espacio tangente de $p$ en $M$}{Espacio tangente de una variedad}, $T_pM$ al
  conjunto de todas los vectores tangentes a todas las curvas centradas en $p$.
\end{definition}

\begin{exercise}
  Comprobar que con las operaciones $\mathring{\gamma_p}+\mathring{\gamma_p^\prime}=
  (\mathring{\gamma+\gamma^\prime})_p$ y $r\mathring{\gamma_p}=(\mathring{r\gamma})_p$, el espacio
  tangente tiene estructura de espacio vectorial.
\end{exercise}

Dados dos variedades diferenciables $M$ y $N$ y una aplicación diferenciable $\maps{f}{M}{N}$,
podemos definir una aplicación entre espacios tangentes a ambas variedades de forma natural, sin
más que considerar la composición de aplicaciones de la siguiente manera:
\begin{equation}
  \label{eq:diferencial-aplicacion}
    \mapsdef{d_pf}{Tp(M)}{T_{\varphi(p)}(N)}{\mathring{\gamma_p}}{\mathring{(f\circ\gamma)
    }_{\varphi
    (p)}}
\end{equation}

\begin{exercise}
  Comprobar que $d_pf$ está bien definido y que si $f$ es un difeomorfismo, entonces $d_pf$ es
  un isomorfismo de espacios vectoriales y su inversa es $(d_pf)^{-1}=d_{f(p)}f^{-1}$.
\end{exercise}

Como caso especial, si $N=\R$, entonces $\maps{d_pf}{T_pM}{\R}$, es decir $d_p f$ es un
elemento del dual del espacio tangente, $d_p f\in T_pM^*$


\section{Bases del espacio tangente}\label{sec:bases-del-espacio-tangente}
De forma natural tenemos una base del espacio tangente, usando las curvas definidas por las
coordenadas canónicas de $\R^n$, es decir, que si consideramos $\cartalocal$ y $\maps{x^\alpha}{(-\epsilon, \epsilon)\subset\R}{U\subset M}$ definido por $x^\alpha(t)=\varphi^{-1}(\varphi(p)+te^\alpha)$, las curvas en $M$
definidas por las coordenadas canónicas $e^\alpha$,
vamos a comprobar que $\{\mathring{x_p^\alpha}\}$ es una base de $T_pM$.

\begin{proposition}
  Sea $\cartalocal$ y $\maps{
  x^\alpha}{(-\epsilon, \epsilon)
  \subset\R}{U\subset M}$ definido por $x^\alpha(t)=\varphi^{-1}(\varphi(p)+te^\alpha)$, las
  curvas en
  $M$
  definidas por las coordenadas canónicas $e^\alpha$, entonces
  $\mathring{x_p^\alpha}=\mathring{x_p^\beta}\so\alpha=\beta$.
\end{proposition}
\begin{proof}
  Que $\mathring{x_p^\alpha}=\mathring{x_p^\beta}$ significa que
  $x^\alpha\sim x^\beta$, es decir, que las derivadas de
  $\varphi\circ x^\alpha$ y $\varphi\circ x^\beta$ en 0 son iguales.
  Las coordenadas de estas funciones en $\R^n$ se expresan como:
  \begin{equation*}
    (\pi^\gamma\circ\varphi\circ x^\alpha)(t)=\pi^\gamma(\varphi(\varphi^{-1}(\varphi(p)
    +te^\alpha)))=\pi^\gamma(\varphi(p)
    +te^\alpha)=\varphi^\gamma(p)+\delta^{\gamma\alpha} t
  \end{equation*}
  Por lo tanto, las coordenadas de la derivada de $\varphi\circ x^\alpha$ es
  $(\pi^\gamma\circ\varphi\circ x^\alpha)^\prime(t)=\delta^{\gamma\alpha}$ y de la igualdad dada
  por la relación de equivalencia, tenemos que
  $\delta^{\gamma\alpha}=\delta^{\gamma\alpha}\so\alpha=\beta$.
\end{proof}

\begin{proposition}
  Sea $\cartalocal$ y $\maps{
  x^\alpha}{(-\epsilon, \epsilon)
  \subset\R}{U\subset M}$ definido por $x^\alpha(t)=\varphi^{-1}(\varphi(p)+te^\alpha)$, las curvas en
  $M$
  definidas por las coordenadas canónicas $e^\alpha$, entonces, toda familia de
  $\{\lambda_\alpha\}\subset\R$, tal que
  $\lambda_\alpha\mathring{x_p^\alpha}=0\so\lambda_\alpha=0\ \forall\ \alpha$.
\end{proposition}
\begin{proof}
  Como hemos visto anteriormente, la coordenada $\gamma$ de la derivada en $0$ de $x^\alpha$ es
  $\delta^{\gamma\alpha}$ y por tanto la coordenada $\gamma$ de la derivada en $0$ de
  $\lambda_\alpha x^\alpha$ es $\lambda_\alpha\delta^{\gamma\alpha}=\lambda_\gamma$ y como este
  debe ser igual a $0$ en cada coordenada se tiene que $\lambda_\gamma=0\ \forall\ \gamma$.
\end{proof}

\begin{proposition}
  Sea $\cartalocal$ y $\maps{
  x^\alpha}{(-\epsilon, \epsilon)
  \subset\R}{U\subset M}$ definido por $x^\alpha(t)=\varphi^{-1}(\varphi(p)+te^\alpha)$, las curvas en
  $M$
  definidas por las coordenadas canónicas $e^\alpha$, entonces, para toda curva $\gamma$ en $M$ con
  $\gamma(0)=p$, existe una familia
  $\{\lambda_\alpha\}\subset\R$, tal que
  $\mathring{f}=\lambda_\alpha\mathring{x_p^\alpha}=0\so\lambda_\alpha$.
\end{proposition}
\begin{proof}
  La derivada de $f$ en $0$ es un elemento de $\R^n$ con coordenadas $(f_\alpha)$, como hemos visto
  anteriormente, las coordenadas de $\lambda_\alpha x^\alpha$ es $(\lambda_\alpha)$, por lo que
  solo es necesario definir $\lambda_\alpha=f_\alpha$.
\end{proof}

Con estos tres resultados, hemos comprobado que $T_pM$ es un espacio vectorial de dimensión $n$
y además hemos definido una base que usa las coordenadas canónicas de $\R^n$.
Por eso dejamos de usar la notación de la clase de equivalencia para referirnos a un elemento del
espacio tangente y lo denotaremos como vectores.

Sin embargo, esta forma de caracterizar el espacio vectorial no es útil, y resulta más práctico
obtener una descripción del espacio tangente y de una base canónica en función de las coordenadas
locales.
Esto es lo que vamos a ver más adelante.


\section{Fibrado tangente}\label{sec:fibrado-tangente}
\begin{definition}
  Sea $M$ una variedad, llamamos \define{fibrado tangente}{Fibrado tangente} al
  conjunto $TM=\{(p,v)\ \mid\ p\in M\ \land\ v\in T_pM\}$
\end{definition}

\begin{proposition}
  El fibrado tangente de una variedad $n$-dimensional es una variedad $2n$-dimensional.
\end{proposition}

Si consideramos la proyección canónica de $\maps{\pi}{T(M)}{M}$, tenemos que para cada punto $p$ de
$M$ el espacio tangente es $T_pM=\pi^{-1}(p)$, vamos a generalizar este concepto.

\begin{definition}
  Un \define{paquete de vectores}{Paquete de vectores} de rango $m$ sobre una variedad $M$
  n-dimensional es una variedad $E$ junto a una aplicación diferenciable de $\maps{\pi}{E}{M}$ tal
  que para cada punto $p$ de $M$, $E_p=\pi^{-1}(p)$ es un espacio vectorial, además existe $U$ un
  entorno de $p$ en $M$ y un difeomorfismo $\maps{\phi}{\pi^{-1}(U)}{U\times\R^m}$ con la condición
  de que:
  \begin{itemize}
    \item $pr_1\circ\phi=\pi$.
    \item $\forall\ p\in M, pr_2\circ\phi\mid_{E_p}$ es un isomorfismo lineal.
  \end{itemize}
\end{definition}


\section{Campo de vectores}\label{sec:campo-de-vectores}
\begin{definition}
  Llamamos \define{campo de vectores}{Campo de vectores} $X$ en $M$ al subconjunto de $T(M)$ formado por
  $X=\cup_{p\in M}(p,v)$.
\end{definition}

\begin{definition}
  Sea $X$ un campo de vectores sobre $M$ y $f\in\cinfinity{M}$, llamamos \define{derivada de Lie de
  f en X}{Derivada de Lie sobre una función} a:
  \begin{equation}
    \label{eq:derivada-lie-funcion-campo}
      \mapsdef{\mathscr{L}_X f}{M}{\R}{p}{d_p f(X(p))}
  \end{equation}
\end{definition}


\section{Espacio tangente como derivaciones}\label{sec:espacio-tangente-como-derivaciones}
\begin{definition}
  Sea $\cartalocal$ con coordenadas locales $(x^\alpha)$, sea $f\in\cinfinity{p}$, definimos la
  \define{derivada parcial}{Derivada parcial de una función} de $f$ con respecto a $x^\alpha$ en
  el punto $p$ por
  \begin{equation}
    \label{eq:parcial-funcion-punto}
    \frac{\partial f}{\partial x^\alpha}(p)=\frac{\partial (f\circ\varphi^{-1})}{\partial
    e^\alpha}(\varphi(p)).
  \end{equation}
  Donde $(e^\alpha)$ son las coordenadas canónicas de $\R^n$.
\end{definition}

\begin{definition}
  Sea $\cartalocal$ con coordenadas locales $(x^\alpha)$, definimos el
  \define{operador derivada parcial}{Operador derivada parcial de una función} con
  respecto a $x^\alpha$ en
  el punto $p$ al operador
  \begin{equation}
    \label{eq:operador-parcial-funcion-punto}
      \mapsdef{\left.\frac{\partial}{\partial x^\alpha}\right|_p}{\cinfinity{p}}{\R}{f}{\frac{\partial f}{\partial x^\alpha}(p)}
  \end{equation}
\end{definition}

\begin{notation}
  Denotamos al operador derivada parcial con respecto a $x^\alpha$ en $p$:
  \begin{itemize}
    \item Cuando no sea preciso indicar el punto $p$, por $\partial_{x^\alpha}$.
    \item Cuando tampoco sea preciso indicar las coordenadas, por $\partial^\alpha$ o $\partial_\alpha$.
  \end{itemize}
\end{notation}

\section{Diferencial de una función}\label{sec:diferencial-de-una-funcion}
\begin{definition}
  Sea $\cartalocal$ y $\mathring{\gamma_p}$ un elemento del espacio tangente en $p$, dada una
  función
  $f\in\cinfinity{M}$, definimos la \define{función diferencial de $f$ sobre
  $\mathring{\gamma_p}$}{diferencial de función en un punto} por
  \begin{equation}
    \label{eq:diferencial-funcion-punto}
    d_{\mathring{\gamma_p}}(f)=\frac{d}{dt}(f\circ\gamma)(t)\mid_{t=0}.
  \end{equation}
\end{definition}
