Por~\ref{sec:base-dual} el dual de la base local de $T_p M$ es una base de $T^*_p M$, además se cumple la igualdad $
(\partial^\alpha)^*(\partial^\beta)=\delta^{\alpha\beta}$.

\begin{definition}
  Sea $\cartalocal$ y $\mathring{\gamma}\in T_p M$, sea $\phi\in\cinfinity{M}$, se define \define{diferencial
  de $\phi$ sobre $\mathring{\gamma}$}{diferencial de función en un punto} por
  \begin{equation}
    \label{eq:diferencial-funcion-punto}
    \left.\frac{d}{dt}(\phi\circ\gamma)\right|_{t=0}
  \end{equation}
\end{definition}
\begin{notation}
  \
  \begin{itemize}
    \item Se escribe $d_{\phi}(\mathring{\gamma})$ la diferencial de $\phi$ sobre $\mathring{\gamma}$.
    \item Si $\phi=\varphi^\alpha$ se escribe $d_\alpha(\mathring{\gamma})=d_{\varphi^\alpha}(\mathring{\gamma})$.
  \end{itemize}
\end{notation}

Sea $\cartalocal$, $\varphi^\alpha$ una coordenada local y $x^\beta$ una curva local, si calculamos $d_{\alpha}
(\partial^\beta)$ se tiene que:
\[
  d_{\alpha}(\partial^\beta)=\left.\frac{d}{dt}(\varphi^\alpha\circ x^\beta)\right|_{t=0}=
  (\varphi^\alpha(\varphi^{-1}(\varphi(p)+te^\beta)))^{'}(0)=(\varphi^\alpha(p)
  +t\delta^{\alpha\beta})^{'}(0)=\delta^{\alpha\beta}
\]

Por lo tanto se tiene la siguiente igualdad $(\partial^\alpha)^*=d_\alpha$.

\begin{definition}
  Sea $\cartalocal$, la base formada por las diferenciales sobre
  las coordenadas locales se llama \define{base local}{def:base-local-dual-espacio-tangente} de
  $T^*_p M$.
\end{definition}

Dados dos variedades diferenciables $M$ y $N$ y una aplicación diferenciable $\maps{\phi}{M}{N}$,
podemos definir una aplicación entre espacios tangentes a ambas variedades de forma natural, sin
más que considerar la composición de aplicaciones de la siguiente manera:
\begin{equation}
  \label{eq:diferencial-aplicacion}
  \begin{alignedat}{1}
    \mapsdef{d_p\phi}{T_pM}{T_{\phi(p)}N}{\mathring{\gamma}}{\mathring{\phi\circ\gamma}}
  \end{alignedat}
\end{equation}

\begin{exercise}
  Comprobar que $d_p\phi$ está bien definido.
  Si $\phi$ es un difeomorfismo, entonces $d_p\phi$ es
  un isomorfismo de espacios vectoriales y su inversa es $(d_p\phi)^{-1}=d_{\phi(p)}\phi^{-1}$.
\end{exercise}
