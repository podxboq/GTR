\section{Base del dual del espacio tangente}\label{sec:base-dual-espacio-tangente}

Ahora vamos a ver como es el espacio dual del espacio tangente, ya sabemos por~\ref{sec:espacio-dual} que el dual de
una base de \gls{espacio-tangente} es una base de \glossarydef{espacio-tangente-dual}{$T^*_p\mathcal{M}$}{Dual del
espacio tangente} y que dado $\gamma^{'}\in$\gls{espacio-tangente} por~\ref{res:coordenadas_duales} se cumple que $\gamma^{'} = (\partial^\alpha)^*(\gamma^{'})
\partial^\alpha$ y por~\ref{eq:vector-tangente-coordenadas-locales} tenemos que $(\partial^\alpha)^*(\gamma^{'})=\frac{d
(\varphi^\alpha\circ\gamma)}{dt}(0)$, por esta igualdad, está justificada la siguiente notación.

\begin{notation}
  Llamamos \glossarydef{diferencial-local}{$d_\alpha$}{Dual de $\partial^\alpha$}$= (\partial^\alpha)^*$.
\end{notation}

\begin{definition}
  Sea $\cartalocal$, la base $\{d_\alpha\}$ formada por las duales de la base local de \gls{espacio-tangente} se llama
  \define{base local}{def:base-local-dual-espacio-tangente} de $T^*_p\mathcal{M}$.
\end{definition}

Así, por~\ref{sec:espacio-dual} todos los vectores $\gamma^{'}\in T_p\mathcal{M}$ y $\omega\in T_p^*\mathcal{M}$ tiene
la siguiente expresión en sus respectivas bases locales:
\begin{equation}
  \label{eq:vectores-tangentes-duales-coordenadas-locales}
  \omega = \omega(\partial^\alpha)d_{\alpha}\\
  \gamma^{'}=d_{\alpha}(\gamma^{'})\partial^{\alpha}
\end{equation}

Si $\omega=(\gamma^{'})^*$, como las coordenadas en sus respectivas bases son las mismas, entonces obtenemos la
siguiente igualdad
\begin{equation}
  \label{eq:vector-tangente-coordenadas-locales-iguales}
  \omega(\partial^\alpha)=d_{\alpha}(\gamma^{'})
\end{equation}


\section{La diferencial de una aplicación}\label{subsec:diferencial-aplicacion}
Sea $\doscartaslocales$, $\gamma$ una curva centrada en $p$ y $\maps{\phi}{U}{V}$ una aplicación
diferenciable con $\phi(p)=q$.
Como $\phi\circ\gamma$ es una curva centrada en $q$, tenemos que $(\phi\circ\gamma)^{'}(0)\in T_q\mathcal{N}$ y por
tanto podemos definir una aplicación entre espacios tangentes a ambas variedades de forma natural, sin
más que considerar dicha composición de aplicaciones de la siguiente manera:
\begin{definition}
  Sea $\maps{\phi}{\mathcal{M}}{\mathcal{N}}$ una aplicación diferenciable, definimos \define{diferencial de $\phi$
  en $p$}{diferencial-punto} a la aplicación
  \begin{equation*}
    \begin{alignedat}{1}
      \mapsdef{d_p\phi}{T_p\mathcal{M}}{T_q\mathcal{N}}{\gamma^{'}}{(\phi\circ\gamma)^{'}}
    \end{alignedat}
  \end{equation*}
\end{definition}

\begin{exercise}
  Comprobar que $d_p\phi$ está bien definida.
  Si $\phi$ es un difeomorfismo, entonces $d_p\phi$ es
  un isomorfismo de espacios vectoriales y su inversa es $(d_p\phi)^{-1}=d_{\phi(p)}\phi^{-1}$.
\end{exercise}
