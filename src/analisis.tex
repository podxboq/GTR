\begin{definition}
  Dado un campo escalar $f$, se llama \textbf{gradiente}\label{def:gradiente} de $f$ y se escribe $\nabla
  f$, a la función $(n,n)$-suave que a cada punto $p$ le asigna el vector cuyas
  coordenadas cartesianas son las derivadas parciales de $f$ en $p$, $\nabla f(p)=\left({\frac
  {\partial f}{\partial x^\alpha}}(p)\right)_\alpha$.
\end{definition}

Considerando $\nabla$ como un operador, es lineal y cumple la regla del producto.

\begin{theorem}[Regla de la cadena]\label{th:regla-cadena}
  Si $f$ es un campo escalar en $\R^n$ y $\gamma$ una curva en $\R^n$, la derivada de la composición
es $(f\circ\gamma)^{'}(t)=\nabla f(\gamma(t))\gamma^{'}(t)$.
\end{theorem}

\section{Matriz Jacobiana}\label{sec:matriz-jacobiana}
\begin{definition}
  Sea $f$ una función $(n,m)$-suave, se llama \textbf{matriz
  Jacobiana}\label{def:matriz-jacobiana} de $f$ y se escribe $\mathbf{J}_f$ a la matriz
  definida por
  $\mathbf{J}_f=\left({\frac {\partial f^\beta}{\partial x^\alpha}}\right)_\alpha^\beta$.
  Llamamos \textbf{Jacobiano}\label{def:jacobiano} de $f$ al determinante de la matriz Jacobiana.
\end{definition}
Cuando $f$ es un campo escalar en $\R^n$, la matriz jacobiana es el gradiente.
Además, cuando esté definida $f^{-1}$, se cumple que $\mathbf{J}_{f^{-1}}=\mathbf{J}_f^{-1}$.
