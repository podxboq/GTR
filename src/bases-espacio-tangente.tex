De forma natural tenemos una base del espacio tangente, usando las curvas definidas por las
coordenadas canónicas de $\R^n$, es decir, que si consideramos $\cartalocal$ y $\maps{x^\alpha}{(-\epsilon, \epsilon)\subset\R}{U\subset M}$ definido por $x^\alpha(t)=\varphi^{-1}(\varphi(p)+te^\alpha)$, las curvas en $M$
definidas por las coordenadas canónicas $e^\alpha$,
vamos a comprobar que $\{\mathring{x^\alpha}\}$ es una base de $T_pM$.

\begin{proposition}
  Sea $\cartalocal$ y $\maps{x^\alpha}{(-\epsilon, \epsilon)\subset\R}{U\subset M}$ definido por $x^\alpha(t)=\varphi^{-1}(\varphi(p)+te^\alpha)$, las
  curvas en
  $M$
  definidas por las coordenadas canónicas $e^\alpha$, entonces
  $\mathring{x^\alpha}=\mathring{x^\beta}\so\alpha=\beta$.
\end{proposition}
\begin{proof}
  Que $\mathring{x^\alpha}=\mathring{x^\beta}$ significa que
  $x^\alpha\sim x^\beta$, es decir, que las derivadas de
  $\varphi\circ x^\alpha$ y $\varphi\circ x^\beta$ en 0 son iguales.
  Las coordenadas de estas funciones en $\R^n$ se expresan como:
  \begin{equation*}
    (\pi^\gamma\circ\varphi\circ x^\alpha)(t)=\pi^\gamma(\varphi(\varphi^{-1}(\varphi(p)
    +te^\alpha)))=\pi^\gamma(\varphi(p)
    +te^\alpha)=\varphi^\gamma(p)+\delta^{\gamma\alpha} t
  \end{equation*}
  Por lo tanto, las coordenadas de la derivada de $\varphi\circ x^\alpha$ es
  $(\pi^\gamma\circ\varphi\circ x^\alpha)^\prime(t)=\delta^{\gamma\alpha}$ y de la igualdad dada
  por la relación de equivalencia, tenemos que
  $\delta^{\gamma\alpha}=\delta^{\gamma\alpha}\so\alpha=\beta$.
\end{proof}

\begin{proposition}
  Sea $\cartalocal$ y $\maps{x^\alpha}{(-\epsilon, \epsilon)\subset\R}{U\subset M}$ definido por $x^\alpha(t)=\varphi^{-1}(\varphi(p)+te^\alpha)$, las curvas en
  $M$
  definidas por las coordenadas canónicas $e^\alpha$, entonces, toda familia de
  $\{\lambda_\alpha\}\subset\R$, tal que
  $\lambda_\alpha\mathring{x^\alpha}=0\so\lambda_\alpha=0\ \forall\ \alpha$.
\end{proposition}
\begin{proof}
  Como hemos visto anteriormente, la coordenada $\gamma$ de la derivada en $0$ de $x^\alpha$ es
  $\delta^{\gamma\alpha}$ y por tanto la coordenada $\gamma$ de la derivada en $0$ de
  $\lambda_\alpha x^\alpha$ es $\lambda_\alpha\delta^{\gamma\alpha}=\lambda_\gamma$ y como este
  debe ser igual a $0$ en cada coordenada se tiene que $\lambda_\gamma=0\ \forall\ \gamma$.
\end{proof}

\begin{proposition}
  Sea $\cartalocal$ y $\maps{x^\alpha}{(-\epsilon, \epsilon)\subset\R}{U\subset M}$ definido por $x^\alpha(t)=\varphi^{-1}(\varphi(p)+te^\alpha)$, las curvas en
  $M$
  definidas por las coordenadas canónicas $e^\alpha$, entonces, para toda curva $\gamma$ en $M$ con
  $\gamma(0)=p$, existe una familia
  $\{\lambda_\alpha\}\subset\R$, tal que
  $\mathring{f}=\lambda_\alpha\mathring{x^\alpha}$.
\end{proposition}
\begin{proof}
  La derivada de $f$ en $0$ es un elemento de $\R^n$ con coordenadas $(f_\alpha)$, como hemos visto
  anteriormente, las coordenadas de $\lambda_\alpha x^\alpha$ es $(\lambda_\alpha)$, por lo que
  solo es necesario definir $\lambda_\alpha=f_\alpha$.
\end{proof}

Con estos tres resultados, se ha comprobado que $T_pM$ es un espacio vectorial de dimensión $n$
y además se ha definido una base que usa las coordenadas canónicas de $\R^n$, ahora se va calcular una expresión para
identificar quien es $\mathring{x^\alpha}$.

Sea $\phi\in\cinfinity{p}$, por el desarrollo realizado en~\ref{eq:desarrollo-vector-tangente} y para este caso
que $v_\alpha=(x^\alpha)^{'}(0)=1$,
\[
  \mathring{x^\alpha}(\phi)=\left.\frac{d(\phi\circ x^\alpha)
  }{dt}\right|_{t=0}=\frac{\partial\phi}{\partial x^\alpha}v_\alpha=\frac{\partial}{\partial x^\alpha}(\phi)
  \so\mathring{x^\alpha}=\frac{\partial}{\partial x^\alpha}
\]
\begin{definition}
  Sea $\cartalocal$, la base de $T_p M$ formada por la clase de equivalencia de las curvas
  $x^\alpha(t)=\varphi^{-1}(\varphi(p)+te^\alpha)$ se llama \define{base local}{def:base-local-espacio-tangente} de
  $T_p M$.
  Las curvas $x^\alpha$ se llaman \define{curvas locales}{curvas-locales} de M\@.
\end{definition}
\begin{notation}
  \
  \begin{itemize}
    \item A partir de este momento dejamos de usar la notación $\mathring{x^\alpha}$ para los elementos del espacio
    tangente.
    \item Se escribe $\frac{\partial}{\partial x^\alpha}$ a los elementos de la base local.
    \item Se escribe $\partial^\alpha=\frac{\partial}{\partial x^\alpha}$ cuando no sea necesario expresar las coordenadas
    $x^\alpha$.
  \end{itemize}

\end{notation}