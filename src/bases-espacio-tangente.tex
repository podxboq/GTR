\section{Base del espacio tangente}\label{sec:base-espacio-tangente}

Es una ejercicio fácil, comprobar que \gls{espacio-tangente} es un espacio vectorial, y sin entrar en mucho
detalle, vamos a ver que dicho espacio tiene la misma dimensión que la variedad diferencial y que ${\partial^\alpha}$ es una base.

\begin{proposition}
  Sea $\cartalocal$ y ${\gamma^\alpha}$ las $\alpha$-ésimas curvas locales, entonces, para toda curva $\gamma$
  centrada en $p$, existe una familia $\{\lambda_\alpha\}\subset\R$, tal que $\gamma=\lambda_\alpha\gamma^\alpha$.
\end{proposition}

\begin{proposition}
  Sea $\cartalocal$ y ${\gamma^\alpha}$ las $\alpha$-ésimas curvas locales, si existen dos familias
  $\{\lambda_\alpha\}$ y $\{\nu_\alpha\}\subset\R\tq$
  $\lambda_\alpha\gamma^\alpha=\nu_\beta\gamma^\beta\so\lambda_\alpha=\nu_\alpha$.
\end{proposition}

\begin{definition}
  Sea $\cartalocal$, llamamos \define{base local de \gls{espacio-tangente}}{base-local-espacio-tangente} a la base
  formada por las derivadas parciales con respecto a las coordenadas locales.
\end{definition}

Como hemos visto en~\ref{eq:imagen-vector-funcion} ya sabemos como se expresa la imagen de un vector de
\gls{espacio-tangente} para cualquier función, por lo tanto las coordenadas de $\gamma^{'}\in$\gls{espacio-tangente}
con respecto a la base local es
\begin{equation}\label{eq:vector-tangente-coordenadas-locales}
\gamma^{'}=\frac{d(\varphi^\alpha\circ\gamma)}{dt}(0)\partial^\alpha
\end{equation}

Hemos visto que cada sistema de coordenadas local en p proporciona una base del espacio tangente, por lo que tenemos
que conocer como se relacionan dos bases diferentes, para ello, lo que tenemos que conocer es cual es la matriz de
cambio de coordenadas.

\begin{proposition}
  Dadas dos bases de \gls{espacio-tangente} asociadas a dos sistemas de coordenadas, la matriz del cambio de base no es
  má́s que la matriz Jacobiana de la aplicación cambio de coordenadas.
\end{proposition}

\begin{notation}
  Denotaremos por $\gamma^{'}$ a un vector genérico del espacio tangente.
\end{notation}