%Para mapas de aplicaciones
\usetikzlibrary{cd}

\setlist{nolistsep} % Reduce spacing between bullet points and numbered lists

\definecolor{ocre}{RGB}{243,102,25} % Define the orange color used for highlighting throughout the book

%----------------------------------------------------------------------------------------
%	MARGINS
%----------------------------------------------------------------------------------------

\geometry{
	paper=a4paper, % Paper size, change to letterpaper for US letter size
	top=1.5cm, % Top margin
	bottom=1.5cm, % Bottom margin
	left=1cm, % Left margin
	right=1cm, % Right margin
	headheight=14pt, % Header height
	footskip=20pt, % Space from the bottom margin to the baseline of the footer
	headsep=10pt, % Space from the top margin to the baseline of the header
	%showframe, % Uncomment to show how the type block is set on the page
}

%----------------------------------------------------------------------------------------
%	FONTS
%----------------------------------------------------------------------------------------

\makeindex % Tells LaTeX to create the files required for indexing

%----------------------------------------------------------------------------------------
%	MAIN TABLE OF CONTENTS
%----------------------------------------------------------------------------------------

\contentsmargin{0cm} % Removes the default margin

%----------------------------------------------------------------------------------------
%	THEOREM STYLES
%----------------------------------------------------------------------------------------
\renewcommand{\qedsymbol}{$\blacksquare$}

% Defines the theorem text style for each type of theorem to one of the three styles above
\newcounter{dummy}
\numberwithin{dummy}{section}
\newtheorem{theoremeT}[dummy]{Theorem}
\newtheorem{exerciseT}[dummy]{Ejercicio}
\newtheorem{exampleT}[dummy]{Example}
\newtheorem{definitionT}[dummy]{Definición}
\newtheorem{proposition}[dummy]{Resultado}
\newtheorem{notation}[dummy]{Notación}
\newtheorem*{remark}{OJO}
\numberwithin{equation}{dummy}

% Theorem box
\newmdenv[skipabove=7pt,
skipbelow=7pt,
linewidth=1pt,
backgroundcolor=black!5,
linecolor=ocre]{tBox}

% Exercise box
\newmdenv[skipabove=7pt,
skipbelow=7pt,
rightline=false,
leftline=true,
topline=false,
bottomline=false,
backgroundcolor=ocre!10,
linecolor=ocre,
innerleftmargin=5pt,
innerrightmargin=5pt,
innertopmargin=5pt,
innerbottommargin=5pt,
leftmargin=0cm,
rightmargin=0cm,
linewidth=4pt]{eBox}

% Definition box
\newmdenv[skipabove=7pt,
skipbelow=7pt,
rightline=false,
leftline=true,
topline=false,
bottomline=false,
linecolor=ocre,
innerleftmargin=5pt,
innerrightmargin=5pt,
innertopmargin=0pt,
leftmargin=0cm,
rightmargin=0cm,
linewidth=4pt,
innerbottommargin=0pt]{dBox}

% Creates an environment for each type of theorem and assigns it a theorem text style from the "Theorem Styles" section above and a colored box from above
\newenvironment{theorem}{\begin{tBox}\begin{theoremeT}}{\end{theoremeT}\end{tBox}}
\newenvironment{definition}{\begin{eBox}\begin{definitionT}}{\hfill{\color{ocre}\tiny\ensuremath{\blacksquare}}\end{definitionT}\end{eBox}}
\newenvironment{exercise}{\begin{dBox}\begin{exerciseT}}{\end{exerciseT}\end{dBox}}
