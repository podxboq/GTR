Este libro quiere presentar matemáticamente los conceptos necesarios para el desarrollo de la Teoría General de la
Relatividad (GTR) de Einstein.

Recoger y unificar la terminología usada de forma diferente por diversos autores, ha sido el trabajo más importante
de este libro con respecto a otros libros sobre este tema.

Presentamos conceptos en geometría diferencial, variedades diferenciales, cálculo tensorial, etc.

Este documento puede ser usado libremente en las condiciones que establece la licencia GNU Free Documentation License
(www.gnu.org/copyleft/fdl.html).

Documento escrito en \LaTeX, con IntelliJ Idea como IDE y el plugin TeXiFy IDEA
de Hannah-Sten en un ordenador GNU/Linux (Manjaro).

Iniciado durante la pandemia del 2020.


\section{\'Indices}\label{sec:indices}
Principalmente para simplificar la notación al trabajar con índices, vamos
a introducir algunos conceptos y establecer una notación que haga más cómodo
trabajar con variables multi-indexadas.

\begin{notation}
  Denotaremos por:
  \begin{itemize}
    \item $I_n$ al conjunto de los naturales menores o iguales a $n$.
    \item $I_n^m$ al producto cartesiano $m$-veces de $I_n$.
    \item $P_n^m=\{(i_1,\cdots,i_m)\in I_n^m\ \mid \ i_k\neq 0\ \forall\ \indexdots{k}{1}{m}\}$
    \item $H_{kn}^m(l)=\{(i_1,\cdots,i_m)\in I_n^m\ \mid \ i_k=l\}$
  \end{itemize}
\end{notation}

\begin{notation}
  Si $\sigma=(1,\ldots,n)$, con la expresión $(x^\sigma)_\sigma$, estaremos escribiendo la $n$-tupla $(x^1,\ldots,
  x^n)$.
  Sin embargo con la expresión $x^\sigma$, estaremos escribiendo $x^{1,\ldots,n}$.
\end{notation}

\subsection{Criterio de índices}\label{subsec:criterio-de-indices}

Emplearemos continuamente el \textbf{convenio de sumas de Einstein}\index{convenio de sumas de Einstein} según el cual
índices repetidos arriba y abajo en una expresión están sumado en todos sus posibles valores,
así la expresión $y=\sum_{\alpha=1}^3 c_\alpha x^\alpha=c_1 x^1 + c_2 x^2 + c_3 x^3$
se simplifica por la convención a $y = c_\alpha x^\alpha$.
