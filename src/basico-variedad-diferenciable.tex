\section{Variedad diferenciable}\label{sec:variedad-diferenciable}
Vamos a repasar todos los conceptos que necesitamos y esto nos servirá para fijar la notación que vamos a usar.

\begin{definition}
  Sea $\mathcal{M}$ un conjunto.
  Una \define{carta n-dimensional}{carta-n-dimensional} (o simplemente carta) sobre $\mathcal{M}$ es un par $
  (U\coma\varphi)$ donde $U$ es un
  abierto de
  $\mathcal{M}$ y $\maps{\varphi}{U}{\R^n}$ es un homeomorfismo.
  Al conjunto $U$ se le denomina \define{entorno coordenado}{entorno-coordenado}.
\end{definition}

Vamos a definir sobre $\mathcal{M}$ algunos conceptos trasladados del espacio Euclídeo $\R^n$, restringiendo su
alcance al entorno coordenado y apoyándonos siempre sobre como se proyecta en $\R^n$ a través de las cartas.
Las coordenadas del espacio Euclídeo las denotaremos por $(u^\alpha)$ y por $\{e_\alpha\}$ la base canónica de $\R^n$,
$\pi^\alpha$ es la proyección canónica de la $\alpha$-ésima coordenada de $\R^n$ en $\R$ y $i^\alpha$ es la inclusión
canónica de la $\alpha$-ésima coordenada de $\R$ en $\R^n$.

\begin{definition}
Llamamos \define{$\alpha$-ésima función coordenada}{alpha-esima-funcion-coordenada} a
$\varphi^\alpha=\pi^\alpha\circ\varphi\in\cinfinity{p}$.
\end{definition}

Sobre un punto $p\in\mathcal{M}$ podemos definir muchas cartas distintas, para que las definiciones siguientes no
dependan de la elección de la carta, tenemos que imponer un criterio igualdad topológica.

\begin{definition}
Sea $\mathcal{M}$ un conjunto y $(U\coma\varphi)$ y $(V\coma\phi)$ dos cartas sobre $p\in\mathcal{M}$.
Diremos que son \define{cartas compatibles}{cartas-compatibles}, si $U\cap V=\emptyset$ o los conjuntos $\varphi
(U\cap V)$ y $\phi(U\cap V)$ son abiertos en $\R^n$ y las aplicaciones $\varphi\circ\phi^{-1}$ y
$\phi\circ\varphi^{-1}$ son difeomorfismo de clase $\mathscr{C}^\infty$.
  A las aplicaciones $\varphi\circ\phi^{-1}$ y $\phi\circ\varphi^{-1}$ se les llaman \define{aplicaciones cambio de
coordenadas}{aplicacion-cambio-coordenadas}.
\end{definition}

Si un conjunto está contenido en una única carta, dicho conjunto es difeomorfo al espacio euclídeo y su estudio no
ofrecería ninguna novedad, las estructuras interesantes son por tanto, aquellas que no pueden ser cubiertas por una
única carta, y por tanto necesitan al menos dos cartas para que todos sus puntos estén en algún entorno coordenado.

\begin{definition}
Sea $\mathcal{M}$ un conjunto, llamamos \define{atlas diferenciable n-dimensional}{atlas} (o simplemente atlas) a una
familia contable de cartas compatibles donde la unión de los entornos coordenados cubre a $\mathcal{M}$.
\end{definition}

En algunos textos, se diferencia entre atlas y atlas maximal, pero no es necesario ya que la estructura que dota al
conjunto un atlas es la misma que un atlas maximal, así que nos quedamos con la siguiente definición de variedad
diferenciable.

\begin{definition}
  Una \define{variedad diferenciable n-dimensional}{variedad-diferenciable} es un par $(\mathcal{M}\coma\mathcal{A})$
  donde \glossarydef{variedad-diferencial}{$\mathcal{M}$}{Variedad diferencial} es un conjunto y $\mathcal{A}$ es un atlas diferenciable n-dimensional.
\end{definition}
Para simplificar, diremos que tenemos \glossarydef{carta-local}{$\cartalocal$}{Carta local},
para indicar que $\mathcal{M}$ es una variedad diferenciable de dimensión $n$, que $p$ es un punto de
$\mathcal{M}$ y que $(U\coma\varphi)$ es una carta con $p\in
U$.

\begin{definition}
  Sea $\cartalocal$, diremos que $\gamma$ es una \define{curva centrada en
  $p$}{Curva centrada}\label{curva-centrada} si $\exists\ \epsilon\in\R \ \mid \maps{\gamma}{(-\epsilon\coma\epsilon)
  }{\varphi(U)}$ y
  $\gamma(0)=p$.
\end{definition}

\section{Aplicaciones diferenciables}
Vamos a extender todos los conceptos del cálculo diferencial en $\R^n$ a una variedad diferenciable, dicha extensión
de conceptos inicialmente lo haremos de forma local, es decir, fijado un punto, pues tendremos que apoyarnos siempre
en la existencia de una carta y por tanto de las coordenadas locales en dicho punto.

\begin{definition}
  Sean $\doscartaslocales$, una aplicación
  $\maps{\phi}{M}{N}$ con $\phi(p)=q$ se dice
  \define{diferenciable en p}{aplicación diferenciable} si la aplicación
  $\psi\circ \phi\circ\varphi^{-1}$ es diferenciable en $\varphi(p)$.
  La aplicación se llama diferenciable si lo es en todos los puntos del dominio.
\end{definition}

\begin{notation}
  \
  \begin{itemize}
    \item Se llama \glossarydef{c-infinity-m-n}{$\cinfinity{M, N}$}{conjunto de las aplicaciones diferenciables de
    $\mathcal{M}$ en $N$} al conjunto de todas las aplicaciones de $\mathcal{M}$ en $N$ diferenciables.
    \item Se llama \glossarydef{c-infinity-p-n}{$\cinfinity{p, N}$}{conjunto de las aplicaciones diferenciables en
    $p$ de $\mathcal{M}$ en $N$} al conjunto de todas las aplicaciones de $\mathcal{M}$ en $N$ diferenciables en $p$.
    \item Si $N=\R$, se llama \glossarydef{c-infinity-m-R}{$\cinfinity{M}$}{$\cinfinity{M, \R}$} $=\cinfinity{M,\R}$.
    \item Si $N=\R$, se llama \glossarydef{c-infinity-p-R}{$\cinfinity{p}$}{$\cinfinity{p, \R}$} $=\cinfinity{p,\R}$.
  \end{itemize}
\end{notation}

\subsection{Derivada parcial}\label{subsec:derivada-parcial}
Como sabemos, una aplicación es diferenciable si admite derivadas parciales en cada una de sus coordenadas, por eso
vamos a definir el concepto de derivada parcial para aplicaciones entre variedades diferenciables.

\begin{definition}
  Sean $\doscartaslocales$ y $\phi\in\cinfinity{p, N}$ con $\phi(p)=q$, definimos la
  \define{derivada parcial}{Derivada parcial de una función} de $\phi$ con respecto la coordenada local
  $\varphi^\alpha$ en $p$ por
  \begin{equation*}
    \label{eq:parcial-funcion-punto}
    \frac{\partial\phi}{\partial \varphi^\alpha}(p)=\frac{\partial(\psi\circ\phi\circ\varphi^{-1})}{\partial
    u^\alpha}(\varphi(p))
  \end{equation*}
\end{definition}

\begin{definition}
  Sean $\doscartaslocales$, se define el
  \define{operador derivada parcial}{Operador derivada parcial de una función} con
  respecto a la coordenada local $\varphi^\alpha$ en $p$ por
  \begin{equation*}
    \begin{alignedat}{1}
      \mapsdef{\partial^\alpha_p}{\cinfinity{p\coma N}}{\R^m}{\phi}{\frac{\partial
      \phi}{\partial\varphi^\alpha}(p)}
    \end{alignedat}
  \end{equation*}
\end{definition}

\begin{notation}
Cuando no sea preciso indicar el punto $p$, denotamos el operador derivada parcial con respecto a la coordenada
$\varphi^\alpha$ en $p$ por $\partial^\alpha$.
\end{notation}

Es fácil comprobar que el operador derivada parcial con respecto a la coordenada $\varphi^\alpha$ en $p$ cumple las
siguientes propiedades
\begin{itemize}\label{propiedad-derivada-parcial}
  \item \textbf{$\R$-lineal}: $\forall\ r,s\in\R$ y $\forall f,g\in\cinfinity{p}$ se cumple que $\partial^\alpha(rf+sg)
  =r\partial^\alpha(f)+s\partial^\alpha(g)$.
  \item \textbf{Regla de Leibniz}: $\forall f,g\in\cinfinity{p}$ se cumple que $\partial^\alpha(fg)=\partial^\alpha
  (f)g+f\partial^\alpha(g)$.
\end{itemize}

Como caso especial, vamos a calcular las derivadas parciales de las funciones coordenadas, como
$\forall\varphi^\beta\in\cinfinity{p}$, por definición tenemos
\begin{equation}\label{eq:derivada-parcial-coordenada-local}
  \frac{\partial\varphi^\beta}{\partial\varphi^\alpha}(p)=\frac{\partial (\varphi^\beta\circ\varphi^{-1})
  }{\partial u^\alpha}(\varphi(p))=\frac{\partial (\pi^\beta\circ\varphi\circ\varphi^{-1})
  }{\partial u^\alpha}(\varphi(p))=\frac{\partial\pi^\beta}{\partial u^\alpha}(\varphi(p))=\delta^{\alpha\beta}
\end{equation}

Para el caso especial de curvas $\phi=\gamma$ donde $\mathcal{M}=\R$, $\varphi=\text{Id}$ y donde la única función
coordenada se identifica con el parámetro, $\varphi^\alpha(t)=\varphi^1(t)=\text{Id}(t)=t$, la definición de derivada
parcial obtiene la siguiente forma
\begin{equation}\label{eq:derivada-parcial-curva}
\frac{\partial\gamma}{\partial t}(0)=\frac{d(\psi\circ\gamma)}{dt}(0)
\end{equation}

\begin{notation}
  Denotamos por $\gamma^{'}(\psi)$ a la diferencial de una curva con
  respecto a las coordenadas locales definidas por $\psi$.
\end{notation}