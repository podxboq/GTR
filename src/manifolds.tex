\section{Aplicaciones diferenciables}\label{sec:aplicaciones-diferenciables}

\begin{definition}
  Sea $\cartalocal$, se llama \define{coordenadas locales}{coordenadas-locales} de la carta, a las aplicaciones
  $\maps{\varphi^\alpha}{U}{\R}$ definido por $\varphi^\alpha=\pi^\alpha\circ\varphi$ donde $\pi^\alpha$ es la
  proyección canónica de $\R^n$ en $\R$.
\end{definition}

\begin{definition}
  Sean $\doscartalocal$, una aplicación
  $\maps{\phi}{M}{N}$ con $\phi(p)=q$ se dice
  \define{diferenciable en p}{aplicación diferenciable} si la aplicación
  $\psi\circ \phi\circ\varphi^{-1}$ es diferenciable en $\varphi(p)$.
  La aplicación se llama diferenciable si lo es en todos los puntos del dominio.
\end{definition}

\begin{notation}
  \
  \begin{itemize}
    \item Se llama $\cinfinity{M, N}$ al conjunto de todas las aplicaciones de $M$ en $N$ diferenciables.
    \item Se llama $\cinfinity{p, N}$ al conjunto de todas las aplicaciones de $M$ en $N$ diferenciables en $p$.
    \item Si $N=\R$, se llama $\cinfinity{M}=\cinfinity{M,\R}$.
    \item Si $N=\R$, se llama $\cinfinity{p}=\cinfinity{p,\R}$.
  \end{itemize}
\end{notation}

\begin{definition}
  Sean $\doscartalocal$ y $\phi\in\cinfinity{p, N}$, se define la
  \define{derivada direccional}{Derivada parcial de una función} de $\phi$ con respecto a $p$ en la dirección $e^\alpha$
  por
  \begin{equation}
    \label{eq:parcial-funcion-punto}
    \frac{\partial f}{\partial e^\alpha}(p)=\left.\frac{\partial (\psi\circ\phi\circ\varphi^{-1})}{\partial
    e^\alpha}\right|_{\varphi(p)}
  \end{equation}
  Donde $(e^\alpha)_\alpha$ son las coordenadas canónicas de $\R^n$.
\end{definition}

\begin{definition}
  Sean $\doscartalocal$, se define el
  \define{operador derivada direccional}{Operador derivada direccional de una función} con
  respecto a $p$ en la dirección $e^\alpha$ por
  \begin{equation}
    \begin{alignedat}{1}
      \label{eq:operador-parcial-funcion-punto}
      \mapsdef{\left.\frac{\partial}{\partial e^\alpha}\right|_p}{\cinfinity{p}}{\R}{\phi}{\frac{\partial
      \phi}{\partial e^\alpha}(p)}
    \end{alignedat}
  \end{equation}
\end{definition}
\begin{notation}
  \
  \begin{itemize}
    \item Se escribe $\partial^{x^\alpha}_p$ al operador derivada direccional con respecto a p en la dirección
    $e^\alpha$.
    \item Cuando no sea preciso indicar el punto $p$, por $\partial^{x^\alpha}$.
    \item Cuando tampoco sea preciso indicar las coordenadas, por $\partial^\alpha$.
  \end{itemize}
\end{notation}


\section{Espacio tangente}\label{sec:espacio-tangente}
\begin{definition}
  Sea $\cartalocal$, diremos que $\gamma$ es una \define{curva centrada en
  $p$}{Curva centrada} si $\exists\ \epsilon\in\R \ \mid \maps{\gamma}{(-\epsilon,\epsilon)}{U}\land
  \gamma(0)=p$.
  Se llama \define{derivada de $\gamma$ en $p$}{Derivada de una curva} al vector
  \begin{equation}
    \label{eq:derivada-curva}
    \left.\frac{d(\varphi\circ\gamma)
    }{dt}\right|_{t=0}=((\varphi^\alpha\circ\gamma)^{'}(0))_\alpha
  \end{equation}
\end{definition}
\begin{notation}
  Se escribe $\gamma^{'}(0)$ a la derivada de $\gamma$ en $p$.
\end{notation}

Considerando una función real definida en $M$ vamos a definir el espacio tangente en un
punto $p$ usando la noción de derivada direccional del vector tangente (velocidad) de una curva
$\gamma$ en el punto $p$.

\begin{definition}
  Sea $\cartalocal$ y $\phi\in\cinfinity{p}$, se llama
  \define{derivada direccional}{Derivada direccional} de $\phi$ en la dirección $v\in\R^n$ a
  \begin{equation}
    \label{eq:derivada-direccional}
    \left.\frac{d(\phi\circ\gamma)}{dt}\right|_{t=0}
  \end{equation}
  donde $\gamma$ es una curva centrada en $p$ con $\gamma^{'}(0)=v$.
\end{definition}
\begin{notation}
  Se escribe $D_v(\phi)$ a la derivada direccional de $\phi$ en la dirección $v$.
\end{notation}
Al desarrollar la expresión~\ref{eq:derivada-direccional} como
$\phi\circ\gamma=\phi\circ\varphi^{-1}\circ\varphi\circ\gamma$, tenemos por un lado
$\maps{\phi\circ\varphi^{-1}}{\R^n}{\R}$
y por otro
$\maps{\varphi\circ\gamma}{\R}{\R^n}$, $(\varphi\circ\gamma)(t)=
(x^\alpha=\varphi^\alpha\circ\gamma)_{\alpha}$,
por la regla de la cadena~\ref{th:regla-cadena} se tiene
\begin{multline}\label{eq:desarrollo-vector-tangente}
  \left.\frac{d(\phi\circ\gamma)}{dt}\right|_{t=0}=
  \left.\frac{d(\phi\circ\varphi^{-1}\circ\varphi\circ\gamma)}{dt}\right|_{t=0}=
  \left.\frac{\partial(\phi\circ\varphi^{-1})}{\partial x^\alpha}\right|_{\varphi(p)}
  \left.\frac{d(\varphi\circ\gamma)}{dt}\right|_{t=0}\by{\ref{def:Derivada de una curva}}\\
  =\left.\frac{\partial(\phi\circ\varphi^{-1})}{\partial x^\alpha}\right|_{\varphi(p)}
  \gamma^{'}(0)\by{\ref{def:Derivada parcial de una función}}
  \left.\frac{\partial\phi}{\partial x^\alpha}(p) v_\alpha.
\end{multline}

Como se ha comprobado, la definición de~\ref{def:Derivada direccional} no depende de $\gamma$
sino sólo de $v$ y por tanto está bien definida y se puede hacer la siguiente construcción.

Sea $\cartalocal$ y $\gamma_1$ y $\gamma_2$ dos curvas centradas en $p$, se define la relación
$\gamma_1\sim\gamma_2$ cuando $\gamma_1^{'}(0)=\gamma_2^{'}(0)$.

\begin{exercise}
  \label{ex:relacion-equivalencia}
  La relación anterior es una relación de equivalencia.
\end{exercise}
\begin{notation}
Se escribe $\mathring{\gamma}$ a la clase de equivalencia de $\gamma$ en el conjunto cociente
$\cinfinity{p}/\sim$.
\end{notation}

\begin{definition}
  Sea $\cartalocal$, $\gamma$ una curva centrada en p y $\sim$ la relación de equivalencia
  descrita en~\ref{ex:relacion-equivalencia}.
  Se llama \define{vector tangente a la curva $\gamma$ en p}{Vector tangente a la curva} a la
  función
  \begin{equation}
    \label{eq:vector-tangente-curva}
    \begin{alignat*}{1}
      \mapsdef{\mathring{\gamma}}{\cinfinity{p}}{\R}{f}{\left.\frac{d(f\circ\gamma)}{dt}\right|_{t=0}}
    \end{alignat*}
  \end{equation}
\end{definition}
\begin{definition}
  Se llama \define{espacio tangente de $p$ en $M$}{def:espacio-tangente-variedad}, al
  conjunto de todos los vectores tangentes a todas las curvas centradas en $p$.
\end{definition}
\begin{notation}
  Se escribe $T_pM$ al espacio tangente de p en M.
\end{notation}

\begin{exercise}
  Comprobar que con las operaciones $\mathring{\gamma}+\mathring{\gamma^\prime}=
  (\mathring{\gamma+\gamma^\prime})$ y $r\mathring{\gamma}=(\mathring{r\gamma})$, el espacio
  tangente tiene estructura de espacio vectorial.
\end{exercise}

\subsection{Borrar lo siguiente?}
Dados dos variedades diferenciables $M$ y $N$ y una aplicación diferenciable $\maps{f}{M}{N}$,
podemos definir una aplicación entre espacios tangentes a ambas variedades de forma natural, sin
más que considerar la composición de aplicaciones de la siguiente manera:
\begin{equation}
  \label{eq:diferencial-aplicacion}
  \begin{alignat*}{1}
    \mapsdef{d_pf}{Tp(M)}{T_{\varphi(p)}(N)}{\mathring{\gamma_p}}{\mathring{(f\circ\gamma)
    }_{\varphi
    (p)}}
  \end{alignat*}
\end{equation}

\begin{exercise}
  Comprobar que $d_pf$ está bien definido y que si $f$ es un difeomorfismo, entonces $d_pf$ es
  un isomorfismo de espacios vectoriales y su inversa es $(d_pf)^{-1}=d_{f(p)}f^{-1}$.
\end{exercise}

Como caso especial, si $N=\R$, entonces $\maps{d_pf}{T_pM}{\R}$, es decir $d_p f$ es un
elemento del dual del espacio tangente, $d_p f\in T_pM^*$


\section{Bases del espacio tangente}\label{sec:bases-espacio-tangente}
\section{Base del espacio tangente}\label{sec:base-espacio-tangente}

Es una ejercicio fácil, comprobar que \gls{espacio-tangente} es un espacio vectorial, y sin entrar en mucho
detalle, vamos a ver que dicho espacio tiene la misma dimensión que la variedad diferencial y que ${\partial^\alpha}$ es una base.

\begin{proposition}
  Sea $\cartalocal$ y ${\gamma^\alpha}$ las $\alpha$-ésimas curvas locales, entonces, para toda curva $\gamma$
  centrada en $p$, existe una familia $\{\lambda_\alpha\}\subset\R$, tal que $\gamma=\lambda_\alpha\gamma^\alpha$.
\end{proposition}

\begin{proposition}
  Sea $\cartalocal$ y ${\gamma^\alpha}$ las $\alpha$-ésimas curvas locales, si existen dos familias
  $\{\lambda_\alpha\}$ y $\{\nu_\alpha\}\subset\R\tq$
  $\lambda_\alpha\gamma^\alpha=\nu_\beta\gamma^\beta\so\lambda_\alpha=\nu_\alpha$.
\end{proposition}

\begin{definition}
  Sea $\cartalocal$, llamamos \define{base local de \gls{espacio-tangente}}{base-local-espacio-tangente} a la base
  formada por las derivadas parciales con respecto a las coordenadas locales.
\end{definition}

Como hemos visto en~\ref{eq:imagen-vector-funcion} ya sabemos como se expresa la imagen de un vector de
\gls{espacio-tangente} para cualquier función, por lo tanto las coordenadas de $\gamma^{'}\in$\gls{espacio-tangente}
con respecto a la base local es
\begin{equation}\label{eq:vector-tangente-coordenadas-locales}
\gamma^{'}=\frac{d(\varphi^\alpha\circ\gamma)}{dt}(0)\partial^\alpha
\end{equation}

Hemos visto que cada sistema de coordenadas local en p proporciona una base del espacio tangente, por lo que tenemos
que conocer como se relacionan dos bases diferentes, para ello, lo que tenemos que conocer es cual es la matriz de
cambio de coordenadas.

\begin{proposition}
  Dadas dos bases de \gls{espacio-tangente} asociadas a dos sistemas de coordenadas, la matriz del cambio de base no es
  más que la matriz Jacobiana de la aplicación cambio de coordenadas.
\end{proposition}

\begin{notation}
  Denotaremos por $\gamma^{'}$ a un vector genérico del espacio tangente.
\end{notation}

\section{Bases del dual del espacio tangente}\label{sec:bases-dual-espacio-tangente}
\section{Base del dual del espacio tangente}\label{sec:base-dual-espacio-tangente}

Ahora vamos a ver como es el espacio dual del espacio tangente, ya sabemos por~\ref{sec:espacio-dual} que el dual de
una base de \gls{espacio-tangente} es una base de \glossarydef{espacio-tangente-dual}{$T^*_p\mathcal{M}$}{Dual del
espacio tangente} y que dado $\gamma^{'}\in$\gls{espacio-tangente} por~\ref{res:coordenadas_duales} se cumple que $\gamma^{'} = (\partial^\alpha)^*(\gamma^{'})
\partial^\alpha$ y por~\ref{eq:vector-tangente-coordenadas-locales} tenemos que $(\partial^\alpha)^*(\gamma^{'})=\frac{d
(\varphi^\alpha\circ\gamma)}{dt}(0)$, por esta igualdad, está justificada la siguiente notación.

\begin{notation}
  Llamamos \glossarydef{diferencial-local}{$d_\alpha$}{Dual de $\partial^\alpha$}$= (\partial^\alpha)^*$.
\end{notation}

\begin{definition}
  Sea $\cartalocal$, la base $\{d_\alpha\}$ formada por las duales de la base local de \gls{espacio-tangente} se llama
  \define{base local}{def:base-local-dual-espacio-tangente} de $T^*_p\mathcal{M}$.
\end{definition}

Dados dos variedades diferenciables $\mathcal{M}$ y $N$ y una aplicación diferenciable $\maps{\phi}{M}{N}$,
podemos definir una aplicación entre espacios tangentes a ambas variedades de forma natural, sin
más que considerar la composición de aplicaciones de la siguiente manera:
\begin{equation}
  \label{eq:diferencial-aplicacion}
  \begin{alignedat}{1}
    \mapsdef{d_p\phi}{T_pM}{T_{\phi(p)}N}{\gamma^{'}}{\phi\circ\gamma^{'}}
  \end{alignedat}
\end{equation}

\begin{exercise}
  Comprobar que $d_p\phi$ está bien definido.
  Si $\phi$ es un difeomorfismo, entonces $d_p\phi$ es
  un isomorfismo de espacios vectoriales y su inversa es $(d_p\phi)^{-1}=d_{\phi(p)}\phi^{-1}$.
\end{exercise}



\section{Fibrado tangente}\label{sec:fibrado-tangente}
\begin{definition}
  Sea $M$ una variedad, se llama \define{s-fibrado vectorial}{s-fibrado-vectorial} en $M$ al par $(E, \pi)$ donde
  \begin{enumerate}
    \item $\maps{\pi}{E}{M}$ es sobreyectiva.
    \item $\forall p\in M\ \pi^{-1}(p)$ es un espacio vectorial.
    \item $\forall p\in M\ \exists U\subset M$ abierto y $\maps{\psi}{\pi^{-1}(U)}{U\times\R^s}$ tal que los siguientes diagramas son conmutativos:
  \end{enumerate}
  \[
    \begin{tikzcd}
      \pi^{-1}(U)\arrow{r}{\psi}\arrow[hook]{d} & U\times\R^s\arrow{d}{pr_1}\\
      E\arrow{r}{\pi} & U
    \end{tikzcd}
    \begin{tikzcd}
      \pi^{-1}(p)\arrow{r}{\psi}\arrow{dr}{\simeq} & U\times\R^s\arrow{d}{pr_2}\\
       & \R^s
    \end{tikzcd}
  \]
\end{definition}
\begin{notation}
  Se escribe $E_p=\pi^{-1}(p)$.
\end{notation}

\begin{definition}
Se llama \define{fibrado lineal canónico}{fibrado-lineal-canonico} al 1-fibrado vectorial $(E=M\times\R,\pi)$ donde
$\pi$
es la proyección canónica de $E$ en $M$.
\end{definition}

\begin{definition}
  Sea $M$ una variedad, se llama \define{fibrado tangente}{fibrado-tangente} al conjunto $\bigcup_{p\in M}\{p\}\times T_pM$.
  Se llama \define{fibrado cotangente}{fibrado-cotangente} al conjunto $\bigcup_{p\in M}\{p\}\times T^*_pM$.
\end{definition}
\begin{notation}
  Se escribe $TM$ al fibrado tangente de M. Se escribe $T^* M$ al fibrado cotangente de M\@.
\end{notation}

\begin{exercise}
  El fibrado tangente y el fibrado cotangente de una variedad $n$-dimensional son variedades $2n$-dimensionales.
\end{exercise}

\begin{proposition}
El fibrado tangente de una variedad $n$-dimensional es un $n$-fibrado tangente de $M$.
\end{proposition}
\begin{proof}
  Como los elementos de $TM$ son de la forma $(p, \mathring{\gamma})$ con $p\in M \land\mathring{\gamma}\in T_p M$,
  vamos a definir la aplicación $\maps{\pi}{TM}{M}$ por $\pi(p, \mathring{\gamma})=p$.

  Las dos primeras condiciones a cumplir por~\ref{def:s-fibrado-vectorial} son inmediatas pues $\forall\ p\in M\
  E_p=T_p M$, es decir, está definido y es un espacio vectorial.

  Para la tercera condición, dado $p\in M$, consideramos una carta local $(U, \varphi)$ y $\psi$ la aplicación:
  \[
    \begin{alignedat}{1}
    \mapsdef{\psi}{\pi^{-1}(U)}{U\times\R^n}{(p, \mathring{\gamma})}{(p, \gamma^{'}(0))}
    \end{alignedat}
  \]
  Como todas las curvas centradas en $p$ que pertenece a la clase de equivalencia de $\mathring{\gamma}$ cumplen que
  son iguales en su derivada en 0, $\psi$ está bien definido.

  El primer diagrama es conmutativo pues $(pr_1\circ\psi)(p, \mathring{\gamma})=pr_1(p,
\gamma^{'}(0))=p=\pi(p, \mathring{\gamma})\so pr_1\circ\psi=\pi$.

  En cuanto al segundo diagrama $\phi=\maps{\left.(pr_2\circ\psi)\right|_{E_p}}{E_p}{\R^n}$ definido por $\phi(p,
  \mathring{\gamma})=\gamma^{'}(0)$ es claramente un aplicación homomorfa sobreyectiva de espacios vectoriales, para
  demostrar que también es inyectiva, consideremos dos vectores $\mathring{\gamma_1},\mathring{\gamma_2}\in T_p M$
  cuya imagen por $\phi$ es la misma, es decir, que $\gamma_1^{'}(0)=\gamma_2^{'}(0)$, por la relación de
  equivalencia~\ref{ex:relacion-equivalencia} resulta que $\mathring{\gamma_1}=\mathring{\gamma_2}$.
  Por lo tanto $\phi$ es un isomorfismo.
\end{proof}

\begin{exercise}
  Demostrar que $T^* M$ es un $n$-fibrado tangente de $M$.
\end{exercise}

\begin{definition}
  Sea $(E,\pi)$ un $s$-fibrado vectorial sobre la variedad $M$, se llama \define{sección de M en
  E}{sección-fibrado-tangente} a la
  aplicación $\maps{\mathcal{S}}{M}{E}\ \mid\ \pi\circ\mathcal{S}=\operatorname{Id}$.
\end{definition}
\begin{notation}
  Se escribe $\Gamma(M,E)$ al conjunto de todas las secciones de $M$ en $E$.
\end{notation}


\section{El pullback}\label{sec:pullback}
Sea $\maps{\phi}{M}{N}$ una aplicación diferenciable entre dos variedades diferenciables.

\section{Campo de vectores y covectores}\label{sec:campo-vectores-covectores}
\begin{definition}
  Se llama \define{campo de vectores}{campo-vectores} en $M$ a una sección de $M$ en $TM$.
  Se llama \define{campo de covectores}{campo-covectores} en $M$ a una sección de $M$ en $T^* M$.
\end{definition}

\begin{notation}
  \
  \begin{itemize}
    \item Se escribe $\Gamma(TM) = \Gamma(M, TM)$.
    \item Se escribe $\Gamma(T^*M) = \Gamma(M, T^*M)$.
  \end{itemize}
\end{notation}

\section{Espacio tangente como derivaciones}\label{sec:espacio-tangente-como-derivaciones}

\begin{definition}
  Sea $X$ un campo de vectores sobre $M$ y $\phi\in\cinfinity{M}$, se llama \define{derivada de Lie de
  f en X}{Derivada de Lie sobre una función} a:
  \begin{equation}
    \label{eq:derivada-lie-funcion-campo}
    \begin{alignedat}{1}
      \mapsdef{\mathscr{L}_X \phi}{M}{\R}{p}{d_p \phi(X(p))}
    \end{alignedat}
  \end{equation}
\end{definition}

