\section{Aplicaciones diferenciables}\label{sec:aplicaciones-diferenciables}

\begin{definition}
  Sea $\cartalocal$, se llama \define{coordenadas locales}{coordenadas-locales} de la carta, a las aplicaciones
  $\maps{\varphi^\alpha}{U}{\R}$ definido por $\varphi^\alpha=\pi^\alpha\circ\varphi$ donde $\pi^\alpha$ es la
  proyección canónica de $\R^n$ en $\R$.
\end{definition}

\begin{definition}
  Sean $\doscartalocal$, una aplicación
  $\maps{\phi}{M}{N}$ con $\phi(p)=q$ se dice
  \define{diferenciable en p}{aplicación diferenciable} si la aplicación
  $\psi\circ \phi\circ\varphi^{-1}$ es diferenciable en $\varphi(p)$.
  La aplicación se llama diferenciable si lo es en todos los puntos del dominio.
\end{definition}

\begin{notation}
  \
  \begin{itemize}
    \item Se llama $\cinfinity{M, N}$ al conjunto de todas las aplicaciones de $M$ en $N$ diferenciables.
    \item Se llama $\cinfinity{p, N}$ al conjunto de todas las aplicaciones de $M$ en $N$ diferenciables en $p$.
    \item Si $N=\R$, se llama $\cinfinity{M}=\cinfinity{M,\R}$.
    \item Si $N=\R$, se llama $\cinfinity{p}=\cinfinity{p,\R}$.
  \end{itemize}
\end{notation}

\begin{definition}
  Sean $\doscartalocal$ y $\phi\in\cinfinity{p, N}$, se define la
  \define{derivada direccional}{Derivada parcial de una función} de $\phi$ con respecto a $p$ en la dirección $e^\alpha$
  por
  \begin{equation}
    \label{eq:parcial-funcion-punto}
    \frac{\partial f}{\partial e^\alpha}(p)=\left.\frac{\partial (\psi\circ\phi\circ\varphi^{-1})}{\partial
    e^\alpha}\right|_{\varphi(p)}
  \end{equation}
  Donde $(e^\alpha)_\alpha$ son las coordenadas canónicas de $\R^n$.
\end{definition}

\begin{definition}
  Sean $\doscartalocal$, se define el
  \define{operador derivada direccional}{Operador derivada direccional de una función} con
  respecto a $p$ en la dirección $e^\alpha$ por
  \begin{equation}
    \begin{alignedat}{1}
      \label{eq:operador-parcial-funcion-punto}
      \mapsdef{\left.\frac{\partial}{\partial e^\alpha}\right|_p}{\cinfinity{p}}{\R}{\phi}{\frac{\partial
      \phi}{\partial e^\alpha}(p)}
    \end{alignedat}
  \end{equation}
\end{definition}
\begin{notation}
  \
  \begin{itemize}
    \item Se escribe $\partial^{x^\alpha}_p$ al operador derivada direccional con respecto a p en la dirección
    $e^\alpha$.
    \item Cuando no sea preciso indicar el punto $p$, por $\partial^{x^\alpha}$.
    \item Cuando tampoco sea preciso indicar las coordenadas, por $\partial^\alpha$.
  \end{itemize}
\end{notation}


\section{Espacio tangente}\label{sec:espacio-tangente}
\begin{summary}

Recordemos que dada una función $f\in\cinfinity{\R^n\coma\R^m}$, le podemos asociar a cada punto $x\in\R^n$ la
derivada de la función en $x$ que denotamos por $Df_x$.
La estructura afín de $\R^n$ permite interpretar un vector $v\in\R^n$ como un vector tangente a $x$ asociando a cada
función real $g\in\cinfinity{\R^n}$ la derivada direccional $D_{v}g_x=Dg_x(v)$.
De este modo podemos interpretar cada vector tangente como una aplicación que asocia a cada función real de clase
$\cinfinity$ un valor real.

De forma análoga vamos a definir un vector tangente a un punto $p$ de una variedad diferenciable, es decir, que un
vector tangente será una aplicación que a cada función $\cinfinity{p}$ le aplica un número real y que además cumpla
con las propiedades que caracterizan a las derivaciones en $\R^n$, a saber, la linearidad y la regla de Leibniz.
\end{summary}

\section{Espacio tangente}\label{sec:espacio-tangente}

\begin{definition}
  Sea $\cartalocal$ y $\maps{\mathrm{v}}{\cinfinity{p}}{\R}$, diremos que $\mathrm{v}$ es un \define{vector tangente en
  $p$}{Vector tangente en p}, si es $\R$-lineal y cumple la regla de Leibniz.
  Llamamos \glossarydef{espacio-tangente}{$T_p\mathcal{M}$}{Espacio tangente de p en $\mathcal{M}$}
  \define{espacio tangente de
  $p$ en $\mathcal{M}$}{def:espacio-tangente}, al conjunto de todos los vectores tangentes en $p$.
\end{definition}

Por definición, y por lo comentado en~\ref{propiedad-derivada-parcial}, los operadores $\partial^\alpha$ derivada
parcial con respecto a la coordenada $\varphi^\alpha$ son vectores tangentes en $p$, es más, dada cualquier curva
$\gamma$ centrada en $p$ sobre $\mathcal{M}$ tiene asociado un vector tangente de forma natural, pues a cada función
$f\in\cinfinity{p}$ tenemos según~\ref{eq:derivada-parcial-curva} $\gamma^{'}(f)$ haciendo que podamos definir la
aplicación $\maps{\gamma^{'}}{\cinfinity{p}}{\R}$ que es $\R$-lineal y cumple la regla de Leibniz y por tanto por
definición $\gamma^{'}$ es un vector tangente en $p$.

Por otra parte fijado $\alpha$, y tomando $\epsilon\in\R\tq (-\epsilon, \epsilon)\subset\varphi^{-1}(U)$, podemos
definir la siguiente curva centrada en $p$, $\gamma_\alpha(t)=\varphi^{-1}(\varphi(p)+te^\alpha)$, vamos a calcular cual
es el vector tangente a $\gamma_\alpha$ en $p$.
Dado $f\in\cinfinity{p}$

\begin{multline}
\gamma_\alpha^{'}(f)\by{\ref{eq:derivada-parcial-curva}}
  \frac{d(f\circ\gamma_\alpha)}{dt}(0)=
  \frac{d(f\circ\varphi^{-1}\circ\varphi\circ\gamma_\alpha)}{dt}(0)=\\
  =\frac{\partial(f\circ\varphi^{-1})}{\partial u^\beta}(\varphi(p))
  \frac{d u^\beta(\varphi\circ\gamma_\alpha)}{dt}(0)\by{\ref{def:Derivada parcial de una función}}
  \frac{\partial f}{\partial \varphi^\beta}(p)\delta^{\alpha\beta}
=\partial^\alpha(f)
\end{multline}

Acabamos de ver, que $\partial^\alpha$ es un vector tangente a una curva en $p$, exactamente de la $\alpha$-ésima
curva local.
Ahora vamos a ver como se expresa cualquier vector.

\begin{multline}\label{eq:imagen-vector-funcion}
  \gamma^{'}(f)\by{\ref{eq:derivada-parcial-curva}}
  \frac{d(f\circ\gamma)}{dt}(0)=
  \frac{d(f\circ\varphi^{-1}\circ\varphi\circ\gamma)}{dt}(0)=\\
  =\frac{\partial(f\circ\varphi^{-1})}{\partial u^\beta}(\varphi(p))
  \frac{d u^\beta(\varphi\circ\gamma)}{dt}(0)\by{\ref{def:Derivada parcial de una función}}
  \frac{d(\varphi^\beta\circ\gamma)}{dt}(0)\frac{\partial f}{\partial \varphi^\beta}(p)
  =(\frac{d(\varphi^\beta\circ\gamma)}{dt}(0)\partial^\beta)(f)
\end{multline}



\section{Bases del espacio tangente}\label{sec:bases-del-espacio-tangente}
De forma natural tenemos una base del espacio tangente, usando las curvas definidas por las
coordenadas canónicas de $\R^n$, es decir, que si consideramos $\cartalocal$ y $\maps{x^\alpha}{(-\epsilon, \epsilon)\subset\R}{U\subset M}$ definido por $x^\alpha(t)=\varphi^{-1}(\varphi(p)+te^\alpha)$, las curvas en $M$
definidas por las coordenadas canónicas $e^\alpha$,
vamos a comprobar que $\{\mathring{x^\alpha}\}$ es una base de $T_pM$.

\begin{proposition}
  Sea $\cartalocal$ y $\maps{
  x^\alpha}{(-\epsilon, \epsilon)
  \subset\R}{U\subset M}$ definido por $x^\alpha(t)=\varphi^{-1}(\varphi(p)+te^\alpha)$, las
  curvas en
  $M$
  definidas por las coordenadas canónicas $e^\alpha$, entonces
  $\mathring{x^\alpha}=\mathring{x^\beta}\so\alpha=\beta$.
\end{proposition}
\begin{proof}
  Que $\mathring{x^\alpha}=\mathring{x^\beta}$ significa que
  $x^\alpha\sim x^\beta$, es decir, que las derivadas de
  $\varphi\circ x^\alpha$ y $\varphi\circ x^\beta$ en 0 son iguales.
  Las coordenadas de estas funciones en $\R^n$ se expresan como:
  \begin{equation*}
    (\pi^\gamma\circ\varphi\circ x^\alpha)(t)=\pi^\gamma(\varphi(\varphi^{-1}(\varphi(p)
    +te^\alpha)))=\pi^\gamma(\varphi(p)
    +te^\alpha)=\varphi^\gamma(p)+\delta^{\gamma\alpha} t
  \end{equation*}
  Por lo tanto, las coordenadas de la derivada de $\varphi\circ x^\alpha$ es
  $(\pi^\gamma\circ\varphi\circ x^\alpha)^\prime(t)=\delta^{\gamma\alpha}$ y de la igualdad dada
  por la relación de equivalencia, tenemos que
  $\delta^{\gamma\alpha}=\delta^{\gamma\alpha}\so\alpha=\beta$.
\end{proof}

\begin{proposition}
  Sea $\cartalocal$ y $\maps{
  x^\alpha}{(-\epsilon, \epsilon)
  \subset\R}{U\subset M}$ definido por $x^\alpha(t)=\varphi^{-1}(\varphi(p)+te^\alpha)$, las curvas en
  $M$
  definidas por las coordenadas canónicas $e^\alpha$, entonces, toda familia de
  $\{\lambda_\alpha\}\subset\R$, tal que
  $\lambda_\alpha\mathring{x^\alpha}=0\so\lambda_\alpha=0\ \forall\ \alpha$.
\end{proposition}
\begin{proof}
  Como hemos visto anteriormente, la coordenada $\gamma$ de la derivada en $0$ de $x^\alpha$ es
  $\delta^{\gamma\alpha}$ y por tanto la coordenada $\gamma$ de la derivada en $0$ de
  $\lambda_\alpha x^\alpha$ es $\lambda_\alpha\delta^{\gamma\alpha}=\lambda_\gamma$ y como este
  debe ser igual a $0$ en cada coordenada se tiene que $\lambda_\gamma=0\ \forall\ \gamma$.
\end{proof}

\begin{proposition}
  Sea $\cartalocal$ y $\maps{
  x^\alpha}{(-\epsilon, \epsilon)
  \subset\R}{U\subset M}$ definido por $x^\alpha(t)=\varphi^{-1}(\varphi(p)+te^\alpha)$, las curvas en
  $M$
  definidas por las coordenadas canónicas $e^\alpha$, entonces, para toda curva $\gamma$ en $M$ con
  $\gamma(0)=p$, existe una familia
  $\{\lambda_\alpha\}\subset\R$, tal que
  $\mathring{f}=\lambda_\alpha\mathring{x^\alpha}$.
\end{proposition}
\begin{proof}
  La derivada de $f$ en $0$ es un elemento de $\R^n$ con coordenadas $(f_\alpha)$, como hemos visto
  anteriormente, las coordenadas de $\lambda_\alpha x^\alpha$ es $(\lambda_\alpha)$, por lo que
  solo es necesario definir $\lambda_\alpha=f_\alpha$.
\end{proof}

Con estos tres resultados, se ha comprobado que $T_pM$ es un espacio vectorial de dimensión $n$
y además se ha definido una base que usa las coordenadas canónicas de $\R^n$, ahora se va calcular una expresión para
identificar quien es $\mathring{x^\alpha}$.

Sea $\phi\in\cinfinity{p}$, por el desarrollo realizado en~\ref{eq:desarrollo-vector-tangente} y para este caso
que $v_\alpha=(x^\alpha)^{'}(0)=1$,
\[
  \mathring{x^\alpha}(\phi)=\left.\frac{d(\phi\circ x^\alpha)
  }{dt}\right|_{t=0}=\frac{\partial\phi}{\partial x^\alpha}v_\alpha=\frac{\partial}{\partial x^\alpha}(\phi)
  \so\mathring{x^\alpha}=\frac{\partial}{\partial x^\alpha}
\]
\begin{definition}
  Sea $\cartalocal$, la base de $T_p M$ formada por la clase de equivalencia de las curvas
  $x^\alpha(t)=\varphi^{-1}(\varphi(p)+te^\alpha)$ se llama \define{base local}{def:base-local-espacio-tangente} de
  $T_p M$.
\end{definition}
\begin{notation}
  \
  \begin{itemize}
    \item A partir de este momento dejamos de usar la notación $\mathring{x^\alpha}$ para los elementos del espacio
    tangente.
    \item Se escribe $\frac{\partial}{\partial x^\alpha}$ a los elementos de la base local.
    \item Se escribe $\partial^\alpha=\frac{\partial}{\partial x^\alpha}$ cuando no sea necesario expresar las coordenadas
    $x^\alpha$.
  \end{itemize}

\end{notation}


\section{Fibrado tangente}\label{sec:fibrado-tangente}
\begin{definition}
  Sea $M$ una variedad, se llama \define{s-haz fibrado}{def:p-haz-fibrado} en $M$ al par $(E, \pi)$ donde $\forall
  p\in M\ \exists U\subset M$ abierto y $\maps{\psi}{\pi^{-1}(U)}{U\times\R^s}$ tal que los siguiente diagrama es
  conmutativo
  \[
    \begin{tikzcd}
      \pi^{-1}(U)\arrow[r]\arrow[d, hook] & U \\
      E\arrow[r, red] & U
      %%    \pi^{-1}(U)\arrow[r, "\psi"]\arrow[d, hook] & U\times\R^s\arrow[d, "pr_1"]\arrow[r, "pr_2" "\simeq"] & \R^s\\
      %%  E\arrow[r, "\pi" red] & U
    \end{tikzcd}
  \]
\end{definition}
\begin{definition}
  Sea $M$ una variedad, se llama \define{fibrado tangente}{Fibrado tangente} al
  conjunto $\bigcup_{p\in M}\{p\}\times T_pM$.
\end{definition}
\begin{notation}
  Se escribe $TM$ al fibrado tangente de M\@.
\end{notation}
\begin{proposition}
  El fibrado tangente de una variedad $n$-dimensional es una variedad $2n$-dimensional.
\end{proposition}

Si consideramos la proyección canónica de $\maps{\pi}{T(M)}{M}$, tenemos que para cada punto $p$ de
$M$ el espacio tangente es $T_pM=\pi^{-1}(p)$, vamos a generalizar este concepto.

\begin{definition}
  Un \define{paquete de vectores}{Paquete de vectores} de rango $m$ sobre una variedad $M$
  n-dimensional es una variedad $E$ junto a una aplicación diferenciable de $\maps{\pi}{E}{M}$ tal
  que para cada punto $p$ de $M$, $E_p=\pi^{-1}(p)$ es un espacio vectorial, además existe $U$ un
  entorno de $p$ en $M$ y un difeomorfismo $\maps{\phi}{\pi^{-1}(U)}{U\times\R^m}$ con la condición
  de que:
  \begin{itemize}
    \item $pr_1\circ\phi=\pi$.
    \item $\forall\ p\in M, pr_2\circ\phi\mid_{E_p}$ es un isomorfismo lineal.
  \end{itemize}
\end{definition}


\section{Campo de vectores}\label{sec:campo-de-vectores}
\begin{definition}
  se llama \define{campo de vectores}{Campo de vectores} $X$ en $M$ al subconjunto de $T(M)$ formado por
  $X=\cup_{p\in M}(p,v)$.
\end{definition}

\begin{definition}
  Sea $X$ un campo de vectores sobre $M$ y $f\in\cinfinity{M}$, se llama \define{derivada de Lie de
  f en X}{Derivada de Lie sobre una función} a:
  \begin{equation}
    \label{eq:derivada-lie-funcion-campo}
    \begin{alignedat}{1}
      \mapsdef{\mathscr{L}_X f}{M}{\R}{p}{d_p f(X(p))}
    \end{alignedat}
  \end{equation}
\end{definition}


\section{Espacio tangente como derivaciones}\label{sec:espacio-tangente-como-derivaciones}


\section{Diferencial de una función}\label{sec:diferencial-de-una-funcion}
\begin{definition}
  Sea $\cartalocal$ y $\mathring{\gamma_p}$ un elemento del espacio tangente en $p$, dada una
  función
  $f\in\cinfinity{M}$, se define la \define{función diferencial de $f$ sobre
  $\mathring{\gamma_p}$}{diferencial de función en un punto} por
  \begin{equation}
    \label{eq:diferencial-funcion-punto}
    d_{\mathring{\gamma_p}}(f)=\frac{d}{dt}(f\circ\gamma)(t)\mid_{t=0}.
  \end{equation}
\end{definition}
