\begin{summary}

Recordemos que dada una función $f\in\cinfinity{\R^n\coma\R^m}$, le podemos asociar a cada punto $x\in\R^n$ la
derivada de la función en $x$ que denotamos por $Df_x$.
La estructura afín de $\R^n$ permite interpretar un vector $v\in\R^n$ como un vector tangente a $x$ asociando a cada
función real $g\in\cinfinity{\R^n}$ la derivada direccional $D_{v}g_x=Dg_x(v)$.
De este modo podemos interpretar cada vector tangente como una aplicación que asocia a cada función real de clase
$\cinfinity$ un valor real.

De forma análoga vamos a definir un vector tangente a un punto $p$ de una variedad diferenciable, es decir, que un
vector tangente será una aplicación que a cada función $\cinfinity{p}$ le aplica un número real y que además cumpla
con las propiedades que caracterizan a las derivaciones en $\R^n$, a saber, la linearidad y la regla de Leibniz.
\end{summary}

\section{Espacio tangente}\label{sec:espacio-tangente}

\begin{definition}
  Sea $\cartalocal$ y $\maps{\mathrm{v}}{\cinfinity{p}}{\R}$, diremos que $\mathrm{v}$ es un \define{vector tangente en
  $p$}{Vector tangente en p}, si es $\R$-lineal y cumple la regla de Leibniz.
  Llamamos \glossarydef{espacio-tangente}{$T_p\mathcal{M}$}{Espacio tangente de p en $\mathcal{M}$}
  \define{espacio tangente de
  $p$ en $\mathcal{M}$}{def:espacio-tangente}, al conjunto de todos los vectores tangentes en $p$.
\end{definition}

Por definición, y por lo comentado en~\ref{propiedad-derivada-parcial}, los operadores $\partial^\alpha$ derivada
parcial con respecto a la coordenada $\varphi^\alpha$ son vectores tangentes en $p$, es más, dada cualquier curva
$\gamma$ centrada en $p$ sobre $\mathcal{M}$ tiene asociado un vector tangente de forma natural, pues a cada función
$f\in\cinfinity{p}$ tenemos según~\ref{eq:derivada-parcial-curva} $\gamma^{'}(f)$ haciendo que podamos definir la
aplicación $\maps{\gamma^{'}}{\cinfinity{p}}{\R}$ que es $\R$-lineal y cumple la regla de Leibniz y por tanto por
definición $\gamma^{'}$ es un vector tangente en $p$.

Por otra parte fijado $\alpha$, y tomando $\epsilon\in\R\tq (-\epsilon, \epsilon)\subset\varphi^{-1}(U)$, podemos
definir la siguiente curva centrada en $p$, $\gamma_\alpha(t)=\varphi^{-1}(\varphi(p)+te^\alpha)$, vamos a calcular cual
es el vector tangente a $\gamma_\alpha$ en $p$.
Dado $f\in\cinfinity{p}$

\begin{multline}
\gamma_\alpha^{'}(f)\by{\ref{eq:derivada-parcial-curva}}
  \frac{d(f\circ\gamma_\alpha)}{dt}(0)=
  \frac{d(f\circ\varphi^{-1}\circ\varphi\circ\gamma_\alpha)}{dt}(0)=\\
  =\frac{\partial(f\circ\varphi^{-1})}{\partial u^\beta}(\varphi(p))
  \frac{d u^\beta(\varphi\circ\gamma_\alpha)}{dt}(0)\by{\ref{def:Derivada parcial de una función}}
  \frac{\partial f}{\partial \varphi^\beta}(p)\delta^{\alpha\beta}
=\partial^\alpha(f)
\end{multline}

Acabamos de ver, que $\partial^\alpha$ es un vector tangente a una curva en $p$, exactamente de la $\alpha$-ésima
curva local.
Ahora vamos a ver como se expresa cualquier vector.

\begin{multline}\label{eq:imagen-vector-funcion}
  \gamma^{'}(f)\by{\ref{eq:derivada-parcial-curva}}
  \frac{d(f\circ\gamma)}{dt}(0)=
  \frac{d(f\circ\varphi^{-1}\circ\varphi\circ\gamma)}{dt}(0)=\\
  =\frac{\partial(f\circ\varphi^{-1})}{\partial u^\beta}(\varphi(p))
  \frac{d u^\beta(\varphi\circ\gamma)}{dt}(0)\by{\ref{def:Derivada parcial de una función}}
  \frac{d(\varphi^\beta\circ\gamma)}{dt}(0)\frac{\partial f}{\partial \varphi^\beta}(p)
  =(\frac{d(\varphi^\beta\circ\gamma)}{dt}(0)\partial^\beta)(f)
\end{multline}


\section{Base del espacio tangente}\label{sec:base-espacio-tangente}

Es una ejercicio fácil, comprobar que \gls{espacio-tangente} es un espacio vectorial, y sin entrar en mucho
detalle, vamos a ver que dicho espacio tiene la misma dimensión que la variedad diferencial y que ${\partial^\alpha}$ es una base.

\begin{result}
  Sea $\cartalocal$ y ${\gamma^\alpha}$ las $\alpha$-ésimas curvas locales, entonces, para toda curva $\gamma$
  centrada en $p$, existe una familia $\{\lambda_\alpha\}\subset\R$, tal que $\gamma=\lambda_\alpha\gamma^\alpha$.
\end{result}

\begin{result}
  Sea $\cartalocal$ y ${\gamma^\alpha}$ las $\alpha$-ésimas curvas locales, si existen dos familias
  $\{\lambda_\alpha\}$ y $\{\nu_\alpha\}\subset\R\tq$
  $\lambda_\alpha\gamma^\alpha=\nu_\beta\gamma^\beta\so\lambda_\alpha=\nu_\alpha$.
\end{result}

\begin{definition}
  Sea $\cartalocal$, llamamos \define{base local de \gls{espacio-tangente}}{base-local-espacio-tangente} a la base
  formada por las derivadas parciales con respecto a las coordenadas locales.
\end{definition}

Como hemos visto en~\ref{eq:imagen-vector-funcion} ya sabemos como se expresa la imagen de un vector de
\gls{espacio-tangente} para cualquier función, por lo tanto las coordenadas de $\gamma^{'}\in$\gls{espacio-tangente}
con respecto a la base local es
\begin{equation}\label{eq:vector-tangente-coordenadas-locales}
\gamma^{'}=\frac{d(\varphi^\alpha\circ\gamma)}{dt}(0)\partial^\alpha
\end{equation}

Hemos visto que cada sistema de coordenadas local en p proporciona una base del espacio tangente, por lo que tenemos
que conocer como se relacionan dos bases diferentes, para ello, lo que tenemos que conocer es cual es la matriz de
cambio de coordenadas.

\begin{result}
  Dadas dos bases de \gls{espacio-tangente} asociadas a dos sistemas de coordenadas, la matriz del cambio de base no es
  más que la matriz Jacobiana de la aplicación cambio de coordenadas.
\end{result}

\begin{notation}
  Denotaremos por $\gamma^{'}$ a un vector genérico del espacio tangente.
\end{notation}

\section{Base del dual del espacio tangente}\label{sec:base-dual-espacio-tangente}

Ahora vamos a ver como es el espacio dual del espacio tangente, ya sabemos por~\ref{sec:espacio-dual} que el dual de
una base de \gls{espacio-tangente} es una base de \glossarydef{espacio-tangente-dual}{$T^*_p\mathcal{M}$}{Dual del
espacio tangente} y que dado $\gamma^{'}\in$\gls{espacio-tangente} por~\ref{res:coordenadas_duales} se cumple que $\gamma^{'} = (\partial^\alpha)^*(\gamma^{'})
\partial^\alpha$ y por~\ref{eq:vector-tangente-coordenadas-locales} tenemos que $(\partial^\alpha)^*(\gamma^{'})=\frac{d
(\varphi^\alpha\circ\gamma)}{dt}(0)$, por esta igualdad, está justificada la siguiente notación.

\begin{notation}
  Llamamos \glossarydef{diferencial-local}{$d_\alpha$}{Dual de $\partial^\alpha$}$= (\partial^\alpha)^*$.
\end{notation}

\begin{definition}
  Sea $\cartalocal$, la base $\{d_\alpha\}$ formada por las duales de la base local de \gls{espacio-tangente} se llama
  \define{base local}{def:base-local-dual-espacio-tangente} de $T^*_p\mathcal{M}$.
\end{definition}

Así, por~\ref{sec:espacio-dual} todos los vectores $\gamma^{'}\in T_p\mathcal{M}$ y $\omega\in T_p^*\mathcal{M}$ tiene
la siguiente expresión en sus respectivas bases locales:
\begin{equation}
  \label{eq:vectores-tangentes-duales-coordenadas-locales}
  \omega = \omega(\partial^\alpha)d_{\alpha}\\
  \gamma^{'}=d_{\alpha}(\gamma^{'})\partial^{\alpha}
\end{equation}

Si $\omega=(\gamma^{'})^*$, como las coordenadas en sus respectivas bases son las mismas, entonces obtenemos la
siguiente igualdad
\begin{equation}
  \label{eq:vector-tangente-coordenadas-locales-iguales}
  \omega(\partial^\alpha)=d_{\alpha}(\gamma^{'})
\end{equation}


\section{La diferencial de una aplicación}\label{subsec:diferencial-aplicacion}
Sea $\doscartaslocales$, $\gamma$ una curva centrada en $p$ y $\maps{\phi}{U}{V}$ una aplicación
diferenciable con $\phi(p)=q$.
Como $\phi\circ\gamma$ es una curva centrada en $q$, tenemos que $(\phi\circ\gamma)^{'}(0)\in T_q\mathcal{N}$ y por
tanto podemos definir una aplicación entre espacios tangentes a ambas variedades de forma natural, sin
más que considerar dicha composición de aplicaciones de la siguiente manera:
\begin{definition}
  Sea $\maps{\phi}{\mathcal{M}}{\mathcal{N}}$ una aplicación diferenciable, definimos \define{diferencial de $\phi$
  en $p$}{diferencial-punto} a la aplicación
  \begin{equation*}
    \begin{alignedat}{1}
      \mapsdef{d_p\phi}{T_p\mathcal{M}}{T_q\mathcal{N}}{\gamma^{'}}{(\phi\circ\gamma)^{'}}
    \end{alignedat}
  \end{equation*}
\end{definition}

\begin{exercise}
  Comprobar que $d_p\phi$ está bien definida.
  Si $\phi$ es un difeomorfismo, entonces $d_p\phi$ es
  un isomorfismo de espacios vectoriales y su inversa es $(d_p\phi)^{-1}=d_{\phi(p)}\phi^{-1}$.
\end{exercise}



\section{Fibrado tangente}\label{sec:fibrado-tangente}
\begin{definition}
  Sea $\mathcal{M}$ una variedad diferenciable, se llama \define{s-fibrado vectorial}{s-fibrado-vectorial} en $\mathcal{M}$ al
  par $(E\coma \pi)$ donde
  \begin{enumerate}
    \item $\maps{\pi}{E}{\mathcal{M}}$ es sobreyectiva.
    \item $\forall p\in \mathcal{M}\coma\pi^{-1}(p)$ es un espacio vectorial.
    \item $\forall p\in \mathcal{M}\coma\exists\ U\subset \mathcal{M}$ abierto y $\maps{\psi}{\pi^{-1}(U)}{U\times\R^s}$
    tal que los siguientes diagramas son conmutativos:
  \end{enumerate}
  \[
    \begin{tikzcd}
      \pi^{-1}(U)\arrow{r}{\psi}\arrow[hook]{d} & U\times\R^s\arrow{d}{pr_1}\\
      E\arrow{r}{\pi} & U
    \end{tikzcd}
    \begin{tikzcd}
      \pi^{-1}(p)\arrow{r}{\psi}\arrow{dr}{\simeq} & U\times\R^s\arrow{d}{pr_2}\\
       & \R^s
    \end{tikzcd}
  \]
\end{definition}
\begin{notation}
  Se escribe \glossarydef{p-fibrado}{$E_p$}{Fibrado en p}$=\pi^{-1}(p)$.
\end{notation}

\begin{definition}
Se llama \define{fibrado lineal canónico}{fibrado-lineal-canonico} al 1-fibrado vectorial $
(E=\mathcal{M}\times\R\coma\pi)$ donde
$\pi$
es la proyección canónica de $E$ en $\mathcal{M}$.
\end{definition}

\begin{definition}
  Sea $\mathcal{M}$ una variedad, se llama \define{fibrado tangente}{fibrado-tangente} al conjunto $\bigcup_{p\in \mathcal{M}}\{p\}\times T_p\mathcal{M}$.
  Se llama \define{dual del fibrado tangente}{dual-fibrado-tangente} al conjunto $\bigcup_{p\in \mathcal{M}}\{p\}\times T^*_p\mathcal{M}$.
\end{definition}
\begin{notation}
  Se escribe \glossarydef{fibrado-tangente}{$T\mathcal{M}$}{Fibrado tangente de $\mathcal{M}$} al fibrado tangente de $\mathcal{M}$.
  Se escribe \glossarydef{fibrado-dual}{$T^* \mathcal{M}$}{Dual del fibrado de $\mathcal{M}$} al dual del fibrado
  tangente de $\mathcal{M}$.
\end{notation}

\begin{exercise}
  El fibrado tangente y el dual del fibrado tangente de una variedad $n$-dimensional son variedades
  $2n$-dimensionales.
\end{exercise}

\begin{result}
El fibrado tangente de una variedad $n$-dimensional es un $n$-fibrado tangente de $\mathcal{M}$.
\end{result}

\begin{exercise}
  Demostrar que $T^*\mathcal{M}$ es un $n$-fibrado tangente de $\mathcal{M}$.
\end{exercise}

\begin{definition}
  Sea $(E\coma\pi)$ un $s$-fibrado vectorial sobre la variedad $\mathcal{M}$, se llama \define{sección de $\mathcal{M}$ en
  E}{sección-fibrado-tangente} a la
  aplicación $\maps{\mathcal{S}}{\mathcal{M}}{E}\ \mid\ \pi\circ\mathcal{S}=\operatorname{Id}$.
\end{definition}
\begin{notation}
  Se escribe $\Gamma(\mathcal{M},E)$ al conjunto de todas las secciones de $\mathcal{M}$ en $E$.
\end{notation}



\section{Campo de vectores}\label{sec:campo-vectores}
\begin{definition}
	Llamamos \define{campo de vectores}{campo-vectores} en $\mathcal{M}$ a una sección de $\mathcal{M}$ en $T\mathcal{M}$.
	Llamamos \define{campo de vectores duales}{campo-covectores} en $\mathcal{M}$ a una sección de $\mathcal{M}$ en $T^*
	\mathcal{M}$.
\end{definition}

\begin{notation}
	\
	\begin{itemize}
		\item Escribimos $\Gamma(T\mathcal{M}) = \Gamma(\mathcal{M}\coma T\mathcal{M})$.
		\item Escribimos $\Gamma(T^*\mathcal{M}) = \Gamma(\mathcal{M}\coma T^*\mathcal{M})$.
	\end{itemize}
\end{notation}

Sea $\cartalocal$ y $X$ un campo de vectores en $\mathcal{M}$, la restricción de $X$ a $U$
nos permite expresar con respecto a la base local en cada punto de $q\in U$ $X(p)
=X_{\alpha}(p)\partial^\alpha$, donde $\maps{X_\alpha}{U}{\R}$ son diferenciables, esta expresión nos permite ver el
campo de vectores localmente como una aplicación entre las funciones diferenciables en $U$ y las funciones en $U$ con
la asignación:
\begin{equation}
	\label{eq:campo-vectores-homomorfismo}
	\begin{alignedat}{2}
		X\colon\cinfinity{U} & \longrightarrow & \cinfinity{U} \\
		f & \longmapsto & X(f)\colon U & \longrightarrow\R \\
		& & q &\longmapsto X_\alpha(p)\partial^\alpha(f)
	\end{alignedat}
\end{equation}

Con la visión de $X$ como un endomorfismo, se puede comprobar que además, $X$ es una derivación en $\cinfinity{U}$.
Se tiene además que toda derivación en $\cinfinity{U}$ es en realidad un campo de vectores y obtenemos por tanto el
siguiente resultado.

\begin{proposition}
	\label{pro:campo-vectores-derivacion}
	Sea $\mathcal{M}$ una variedad. $X$ es un campo de vectores sobre $\mathcal{M}$ si y sólo si es una derivación
	sobre $\cinfinity{\mathcal{M}}$.
\end{proposition}

Con esta nueva visión de los campos de vectores, podemos considerar la composición de campos como si fueran funciones,
si tenemos $X\coma Y$ dos campos sobre una variedad $\mathcal{M}$, consideramos $XY$ un nuevo campo definido por $XY
(f)=X(Y(f))$.
Sin embargo, como veremos más adelante, esta composición de campos vectoriales no es un campo vectorial.

\begin{exercise}
	Sea $X\coma Y$ dos campos de vectores sobre una variedad $\mathcal{M}$. Demostrar que $XY$ no es un campo vectorial.
\end{exercise}
\begin{proof}
	Sean $f\coma g\in\cinfinity{M}$, se tienen las siguientes igualdades:
	\begin{eqnarray*}
		(XY)(fg) & = & X(Y(fg)) = X(Y(f)g+fY(g)) = X(Y(f)g)+X(fY(g))=\\
		& = & X(Y(f))g+Y(f)X(g)+X(f)Y(g)+fX(Y(g)) = \\
		& = & (XY)(f)g+f(XY)(g)+Y(f)X(g)+X(f)Y(g)
	\end{eqnarray*}
\end{proof}

Como vemos en el desarrollo de la expresión $(XY)(fg)$, tenemos los términos extras $Y(f)X(g)+X(f)Y(g)$ que en
general no es nulo.
Sin embargo, si consideramos la expresión $(XY-YX)$, vemos que la expresión se desarrolla de la siguiente manera:
\begin{eqnarray*}
	(XY-YX)(fg) & = & (XY)(f)g+f(XY)(g)+Y(f)X(g)+X(f)Y(g) - \\
	& - & (YX)(f)g-f(YX)(g)-X(f)Y(g)-Y(f)X(g) = \\
	& = & (XY)(f)g+f(XY)(g) - (YX)(f)g-f(YX)(g) = (XY-YX)(f)g+ f(XY-YX)(g)
\end{eqnarray*}


\begin{definition}
	Sea $\cartalocal$ y $X\coma Y$ dos campos de vectores en $\mathcal{M}$, definimos el \define{corchete de
	Lie}{corchete-lie} al nuevo campo de vectores definido por $[X\coma Y]=XY-YX$.
\end{definition}

El corchete de Lie cumple las siguientes propiedades:
\begin{itemize}
	\item \textbf{Anticonmutativo}: $\lie{X}{Y}=-\lie{Y}{X}$.
	\item \textbf{Lineal}: $\lie{aX+bY}{Z}=a\lie{X}{Z}+b\lie{Y}{Z}$.
	\item \textbf{Identidad de Jacobi}: $\lie{\lie{X}{Y}}{Z}+\lie{\lie{Y}{Z}}{X}+\lie{\lie{Z}{X}}{Y}=0$.
	\item $\lie{fX}{gY}=fg\lie{X}{Y}+fX(g)Y-gY(f)X$.
\end{itemize}

Sea $\cartalocal$ y $X\coma Y$ dos campos de vectores sobre $\mathcal{M}$, sobre $U$ podemos expresar en coordenadas
locales para cada punto $p\in U$ los campos por $X(p)=X_\alpha(p)\partial^\alpha$ y $Y(p)=Y_\alpha(p)\partial^\alpha$,
nos preguntamos que expresión tiene el corchete de Lie.

\begin{eqnarray*}
[X\coma Y]
	(p) & = & (XY-YX)(p) = X(Y(p))-Y(X(p))=X(Y_\alpha(p)\partial^\alpha)-Y(X_\alpha(p)
	\partial^\alpha) = \\
	& = & Y_\alpha(p)X(\partial^\alpha)-X_\alpha(p)Y(\partial^\alpha) = (X(Y_\alpha(p))-Y(X_\alpha(p)))\partial^\alpha
\end{eqnarray*}

\begin{definition}
	Sea $X$ un campo de vectores sobre $\mathcal{M}$ y $\phi\in\cinfinity{M}$, se llama \define{derivada de Lie de
	$\phi$ en X}{Derivada de Lie sobre una función} a:
	\begin{equation}
		\label{eq:derivada-lie-funcion-campo}
		\begin{alignedat}{1}
			\mapsdef{\mathscr{L}_X \phi}{M}{\R}{p}{d_p \phi(X(p))}
		\end{alignedat}
	\end{equation}
\end{definition}

\begin{proposition}
	$\mathscr{L}$ es un isomorfismo entre las derivaciones de $\mathcal{M}$ y $\Gamma(T\mathcal{M})$.
\end{proposition}

\begin{proposition}
	$\lie{\mathscr{L}_X}{\mathscr{L}_Y}=\mathscr{L}_{\lie{X}{Y}}$.
\end{proposition}

\begin{proposition}
	$\lie{X}{Y}=\mathscr{L}_X Y$.
\end{proposition}

\section{Curvas integrales}\label{sec:curvas-integrales}
\begin{definition}
	Sea $X$ un campo vectorial sobre $\mathcal{M}$.
	Una curva $\maps{\gamma}{I}{M}$ es una \define{curva integral de X}{curva-integral} si $\gamma^{'}(t)=X(\gamma
	(t))\
	\forall\
	t\in I$.
\end{definition}

Por el teorema de Picard–Lindelöf, if $X$ es un campo vectorial diferenciable, existe para cada punto $p\in\mathcal{M}$ una
única curva integral de $X$ centrada en $p$.

\begin{notation}
	Sea $X$ un campo vectorial sobre $\mathcal{M}$, denotaremos \glossarydef{curva-x-p}{$\gamma_X^p$}{Curva integral de X
	centrada en p} a la única curva integral de $X$ centrada en $p$.
\end{notation}

\begin{definition}
  Sea $X$ un campo vectorial sobre $\mathcal{M}$, llamaremos \define{flujo}{flujo} generado por $X$ a la aplicación
  $\maps{\sigma}{\R\times\mathcal{M}}{\mathcal{M}}$ definido por $\sigma(t, p)=\gamma_X^p(t)$.
\end{definition}

Un flujo satisface la siguiente propiedad
\begin{equation*}
	\sigma(t, \sigma(s,p))=\sigma(t+s,p)
\end{equation*}