\begin{itemize}
  \item Diremos que $V$ es un e-vectorial para indicar que es un espacio vectorial
  de dimensión finita y con un producto escalar $\pescalar{}{}$ sobre un cuerpo $K$.
  \item Si $V$ es un espacio vectorial, llamamos $V^*$ a su espacio dual.
  \item Diremos que una función de $\R^n$ en $\R^m$ es $(n,m)$-suave si se puede diferencial indefinidamente.
  \item Diremos que $M$ es una variedad diferenciable de dimensión $n$ con una métrica.
  \item Se llama a $\cartalocal$ si $M$ es un variedad diferenciable, $p\in U$ con $U$ abierto en $M$ y $\maps{\varphi}{U}{\R^n}$ es un homeomorfismo.
  \item Usaremos la función delta de kronecker con los índices en sus variantes
  $\delta_{\alpha\beta}=\delta_\alpha^\beta=\delta^{\alpha\beta}$.
  \item Dada una matriz $A$ describiremos a sus entradas con los índices en sus variantes
  $A=(A_{\alpha\beta})_{\alpha\beta}=(A_\alpha^\beta)_\alpha^\beta=(A^{\alpha\beta})^{\alpha\beta}$.
  \item Si $V$ es un espacio vectorial, denotaremos
  $V^{\otimes r}=V\otimes\overbrace{\cdots}^\text{r\ veces}\otimes V$.
\end{itemize}
